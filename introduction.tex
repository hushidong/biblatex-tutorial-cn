
\chapter{What is \package{Biblatex}?}\label{ch:introduction}
\marginpar{\begin{CJK}{UTF8}{gbsn}\footnotesize Biblatex 是个什么包?
\end{CJK}}

This chapter has two different sections, intended for different
audiences. If you are new to automated bibliography tools in \LaTeX,
then you should read the first section.\intref{p~\pageref{newbie}} If
you already have experience of using automated bibliography tools in
\LaTeX, such as \bibtex\ and \package{Natbib}, then you can probably
turn straight to the second section.\intref{p~\pageref{expert}} If you
are prefer to jump in and start getting your hands dirty without much
explanation, there is a quick start section at the back of this
guide,\intref{p~\pageref{ch:quickstart}} along with an aide memoire to
the frequently used commands and
options.\intref{p~\pageref{ch:cheatsheet}}

\section{For the true neophyte}

\subsection{The basic idea}
\label{newbie}
\indexstart{Biblatex!very basic introduction}
\marginpar{\begin{CJK}{UTF8}{gbsn}\footnotesize 给新手的基本概念
\end{CJK}}

Academic writing usually cites sources. 
\marginpar{\begin{CJK}{UTF8}{gbsn}\footnotesize 学术写作需要引用来源,不同的出版商、期刊、个人的写作对于引用的形式要求都是不同的。但文献引用底下的原理是类似的。在正文或脚注等地方引用文献以标注来源,并在合适的地方常用列表的形式给出来源的详细信息。

可以手动给出这些信息,但耗时、费力、容易出错,并且难以应付随时需要的改变。
\end{CJK}}
The exact form of these
citations varies: different disciplines have different practices, and
different publishers and journals too. But the underlying idea is
similar: there are references in the text (or footnotes) that enable
the reader to identify a source for a particular statement or
quotation. This may be a number in a list of references (like [1]),
or a combination of author and year (like Author 2014), or even a
footnote which identifies the source.\footnote{Like: Author, `Paper
  Title' in \emph{Learned Journal}, Vol.\ 10 (2014), p.\ 111.}
Alongside this there is a bibliography: a list of all the sources
which have been cited, which will nearly always enable the reader to
look up the source in a library, and sometimes provide (in itself)
useful information about the source, such as its date, or where
and how it was published.

It is, of course, possible to produce all these indications entirely
`by hand': typing out references, bibliographies and so forth. But
there are disadvantages to doing that:
\begin{itemize}
\item It is time-consuming. If you write lots of papers, you will
  probably find yourself citing the same sources again and again. It's
  tedious to have to keep looking up and typing out the same data. You
  have to spend time doing things like organising your bibliography
  into the proper order.
\item It is error-prone. Every time you type data up, you risk making
  a mistake: either a mistake of substance (typing `Stanely' for
  `Stanley'), or in the way the data is formatted.
\item It can be frustrating. Each citation system has its own picky
  requirements. One wants the titles of articles in double quotes,
  ``Like this''. Another wants them in single quotes, `Like
  this'. Another wants them in italics \emph{Like this}. There are
  specialists who revel in such details: but for the average person
  they are just a nuisance, and it is nice to have a system which
  takes care of these details for you.
\item The details sometimes change as you write. For instance, if you
  use a system where references are numbered, adding a reference may
  change all the numbering. Or if you follow a system which uses
  abbreviations such as `ibid', `op.\ cit.' or references back to
  earlier notes, adding a single footnote can throw everything out. In
  such systems it's not possible to finalise references until the
  document is complete: and every change can easily introduce errors.
\end{itemize}

The idea of a citation system is to \emph{separate content from
  presentation}. 
\marginpar{\begin{CJK}{UTF8}{gbsn}\footnotesize 参考文献引用系统的逻辑是,把文献来源的重要细节比如作者、题名等放到一个数据库文件中。写作时对这些文献做一个简短的引用,然后由计算机从数据库中抽取信息,然后插入文档中。基于这种方式,作者完全可以找到并利用很多已有的文献数据库。
\end{CJK}}  
You record the important details about a source in a
database file: the author(s), title, date, publisher and so forth. As
you write, you use a short reference to indicate the work you want to
cite at that point in your paper. And then you let the computer take
care of extracting the information from your database, then inserting
and formatting it into your paper. And in fact, in many fields, you
can find existing databases which you can borrow from which will
contain data you need.

This is what \biblatex\ does. 
\marginpar{\begin{CJK}{UTF8}{gbsn}\footnotesize 这就是 biblatex 做的事情。用户可以维护好一个数据库。然后在写作需要的时候引用其中的文献,在文档编译是,有biblatex 来获取文献的信息并格式化,最后形成引用的标注标签和文献表。
\end{CJK}}  
You maintain a database which contains
the works you may want to cite. The database can be large: it needn't
include only the works that you want to cite in a particular paper ---
it can be as large as you like. It can be in any order. All that
matters is that it contains the information that is needed to
`construct' the citations.

As you write your paper, you insert commands to tell \biblatex\ that
you want to cite particular sources drawn from that database.

Then, when you are done, as part of the compilation of your \LaTeX\
source, you get \biblatex\ to extract the relevant information, format
it, and construct the citations and bibliography in proper format.

\subsection{In practice}
\marginpar{\begin{CJK}{UTF8}{gbsn}\footnotesize 实践中需要几个工具以一定的流程来实现目标。首先至少需要两个文件一个是tex源文件(可能用include和input包含多个文件),一个是文献数据库文件(可能也有多个数据库)。
\end{CJK}} 

In practice, getting all this to happen involves a number of different
tools and programs, which all have to work together.

You are going to start with two files:\footnote{At least two, since
  your \LaTeX\ source could \cs{include} or \cs{input} more than one
  file, and your database may consist of more than one file too.} a
database file and your \LaTeX\ source file.

\marginpar{\begin{CJK}{UTF8}{gbsn}\footnotesize 数据库文件后缀为.bib,内容包含文献来源的详细信息,并以特定的格式(见\ref{ch:database}章)描述。它是一个普通的文本文件,可以利用任何编辑器编写,或利用一些辅助工具来维护
(见\ref{ch:tools}章)。
\end{CJK}} 
The database file conventionally has the suffix \texttt{.bib}, and
consists of a set of records containing information about the sources
you may wish to cite. The format of such a file is described in
chapter \ref{ch:database}. It is a plain text file, which you can
either produce by hand (using a text editor, such as the one you use
to write \LaTeX\ source code) or can be maintained by one of a number
of available `helper' programs.\intref{Chapter \ref{ch:tools}}

\marginpar{\begin{CJK}{UTF8}{gbsn}\footnotesize \LaTeX\ 源文件后缀为.tex,内容为\LaTeX\ 格式的文本。为利用biblatex引用文献,包含如下内容:

1. biblatex宏包加载语句

2. 加载文献数据源的语句

3. 引用文献的语句

4. 打印文献表的语句
\end{CJK}} 
The \LaTeX\ source file conventionally has the suffix \texttt{.tex}
and contains all your text, marked up in \LaTeX\ format. It also
contains:
\begin{itemize}
\item A\marginnote{\cs{usepackage[style=...]\{biblatex\}}} line that
  loads \biblatex\ and tells it what `style' to use for citations and
  bibliography.
\item One\marginnote{\cs{addbibresource\{file.bib\}}} or more lines
  which tell \biblatex\ what file(s) to use as the database from which
  information is to be extracted.
\item One\marginnote{\cs{cite\{...\}}} or more commands which tell
  \biblatex\ that you want to \cs{cite} a particular source at that
  point in your text.
\item (Usually)\marginnote{\cs{printbibliography}} one or more
  commands that tell \biblatex\ to print a bibliography at that point
  in the document.
\end{itemize}

The next task is to get the various tools to cooperate together to get
to the end result that you need. 
\marginpar{\begin{CJK}{UTF8}{gbsn}\footnotesize 下一步就是利用不同工具来配合得到需要的结果,一般需要如下步骤:

1. \LaTeX\ 第一遍编译tex源文档

2. 利用工具分析文献数据库,抽取数据。可以使用多种工具比如 bibtex,biber 等,对于 biblatex 主要使用 biber。

3. 再次利用\LaTeX\ 编译源文档

4. 有时需要多运行一次\LaTeX\ 来完善文档内的超连接
\end{CJK}} 
In general that requires at least the
following things to happen:
\begin{itemize}
\item Run \LaTeX\ first on the source file. At this point no citations
  are produced. Instead \biblatex\ \emph{records} a list of the
  sources that you have cited.
\item Run another program which reads the list that has been produced,
  analyses the database, and extracts the relevant information into a
  format that \biblatex\ can work with. There is a choice of programs
  that you might use for this, but in this handbook we will be
  assuming that you will use a program called \package{Biber}.
\item Run \LaTeX\ again, to `read in' the data that has now been put
  in digestible form, and produce the citations and bibliography.
\item (Sometimes), run \LaTeX\ yet \emph{again} to finalise
  cross-references.
\end{itemize}

This `pattern' (\LaTeX, \package{Biber}, \LaTeX, [\LaTeX]) is repeated
whenever you need to compile your document.\footnote{Not quite true:
  it is often unnecessary to re-run \package{Biber} if no new
  citations have been added. But it is never wrong to do so.}
\indexstop{Biblatex!very basic introduction}

\subsection{A quick demonstration\label{neophyte:example}}
\marginpar{\begin{CJK}{UTF8}{gbsn}\footnotesize 何不给出一个新手示例?假设读者已经安装一套tex发行版比如texlive,包含了所有编译需要的工具。下面第一步就是构建一个文献数据库文件。
\end{CJK}} 
Why not try a quick example? I assume for these purposes that you have
a fully-equipped and functional \TeX\ system, and have installed the
pre-requisites for \biblatex. This will be true on any of the standard
modern \TeX\ installations; but it's not a bad idea to update to at
least the most recent versions of \biblatex\ and \package{Biber},
since both are under active development.

First, let's set up our `database'. It's not going to be much: just a
single book.

Open a text editor, and produce a new file
\texttt{handbook.bib}:\intref{Much more about how to write
  \texttt{.bib} files is explained in chapter~\ref{ch:database}.}

\begin{verbatim}
@book{nussbaum:95,
  author = "Nussbaum, Martha C.",
  title = "Poetic Justice",
  subtitle = "The Literary Imagination and Public Life",
  publisher = "Beacon Press",
  location = "Boston",
  date = "1995",
}
\end{verbatim}

\marginpar{\begin{CJK}{UTF8}{gbsn}\footnotesize 然后利用文本编辑器写出一个tex源文档。并在其中引用文献,引用的命令详见第\ref{ch:citationcommands}章。
\end{CJK}}
Now open a text editor or your \LaTeX\ editor, and create a small test
file (I'll call mine \texttt{test.tex}):\intref[1.5in]{Much more about
  \cs{cite} commands is explained in
  chapter~\ref{ch:citationcommands}.}

\begin{verbatim}
\documentclass{article}
\usepackage{csquotes}
\usepackage[style=numeric]{biblatex}
\addbibresource{handbook.bib}
\begin{document}

As Nussbaum comments \cite[17]{nussbaum:95}:
\enquote{The utilitarian picture of human beings and
of rationality is familiar enough in theory}.

\printbibliography
\end{document}
\end{verbatim}

\marginpar{\begin{CJK}{UTF8}{gbsn}\footnotesize 接着开始根据前面介绍的步骤进行编译,注意每一步的输出结果。
\end{CJK}}
Now, run \LaTeX\ on the file, once. If you look at the resulting
typeset document, it should look something like figure
\ref{nussbaum1}.

\begin{figure}
\fbox{\includegraphics{./examples/nussbaum1u.pdf}}
\caption{Before running \package{Biber}}\label{nussbaum1}
\end{figure}

The reference `[\textbf{sebum:95}]' appears in square brackets
because although \LaTeX\ can `see' that there is going to be a
citation, it doesn't yet have the data that will enable it to
construct that citation, since \package{Biber} has not been run.

Now, from a terminal window opened in the same directory, run
\framebox{\parbox{\linewidth}{\texttt biber test}}

Hopefully, you will see a number of messages marked as `INFO', ending
with \texttt{INFO -- Output to test.bbl}.

Now run \LaTeX\ again on the source file |test.tex|. And if all is
going well, you should now see output like figure \ref{nussbaum2}.

\begin{figure}
\fbox{\includegraphics{./examples/nussbaum2u.pdf}}
\caption{After running \package{Biber} and \LaTeX}\label{nussbaum2}
\end{figure}

\marginpar{\begin{CJK}{UTF8}{gbsn}\footnotesize 最后可以看到,引用文献的信息已经在文献表中了。如果没有得到该结果,有可能是biber没有成功运行,这是可能在哪存在问题,可以按如下问题来检查:
1.加入文献源了么?2.编译每一步都完成了么?3.biber结果是否提示错误?4.引用来源的时候文献的关键字是否正确?5. bib文件是否包含数据?6.bib中的文献条目是否有效?具体检查方法详见第\ref{ch:troubleshooting}章。
\end{CJK}}
As you can see, the citation data has been pulled into the
bibliography. If the output remains unchanged (it looks like figure
\ref{nussbaum1}) that is because \package{Biber} has not successfully
run. That is \emph{probably} because there is some error in your
|.bib| file, so go back and check that. Any time you see this, you
should ask yourself these
questions:\marginnote{\textit{Debugging}\\1. \cs{addbibresource}?\\2. Run
  \LaTeX\, \package{Biber}, \LaTeX?\\3. \package{Biber} was
  error-free?\\4. Cite key is correct?\\5. \texttt{.bib} file contains
  source?\\6. Entry in \texttt{.bib} file valid?\\See
  chapter~\ref{ch:troubleshooting}.} (1) have I told \biblatex\ where
to find the bibliography file (using |\addbibresource{}|? (2) Have I
run \LaTeX\ then \package{Biber} then \LaTeX\ again? (3) Did
\package{Biber} report any errors? (If you like, you can check the log file at
\angled{jobname}|.blg|) (4) Is my citation \emph{key} correct (e.g.,
you haven't typed |nusbaum| for |nussbaum| in the \LaTeX\ source? (5)
Does the |.bib| file include that citation? (6) Is the syntax of the
citation correct?

\subsection{What next}

There is still quite a bit to learn, of course. To some extent it's up
to you how you go from here, but the following chapters take what I
think is a reasonably logical order.

\section{For the more experienced}
\label{expert}
\marginpar{\begin{CJK}{UTF8}{gbsn}\footnotesize 给有经验的用户。
\end{CJK}}
\indexstart{Biblatex!compared to bibtex}

\marginpar{\begin{CJK}{UTF8}{gbsn}\footnotesize 从用户的角度使用 biblatex 的模式类似于以前习惯的用\bibtex 加\package{Natbib}的过程。也是把文献数据放到bib文件中,然后用工具按步骤编译。但需要注意源文档中使用的命令方面的差异。
\end{CJK}}
From the user's perspective, the basic pattern followed by \biblatex\
is similar to the one you will be accustomed to if you use \bibtex\
to produce \verb|.bst| files, or citation packages such as \package{Natbib}:
\begin{itemize}
\item Your bibliographical data is still stored in \verb|.bib| files,
  which have largely the same format as \bibtex\ files. (But
  \biblatex\ recognises additional entry types and fields.)
\item You still generate citation and bibliography data by running
  \LaTeX\, then an external program (either \bibtex\ itself or a more
  powerful replacement, \package{Biber}, and then \LaTeX\ again
  (sometimes twice).
\end{itemize}

There are a few, largely cosmetic, changes to the commands that you
need to include in your document to `set up' the bibliography. (See
page \pageref{bibtex:simple:eg}.)

\subsection{Differences}
\marginpar{\begin{CJK}{UTF8}{gbsn}\footnotesize 两者存在的差异主要包括:
1. 格式化文献数据的差异。biblatex用的是tex语言,而bibtex是右bibtex工具格式化。

2. biblatex可以做传统方法不能做的事,比如动态修改。这意味着它更强大,更灵活。其文献数据编译程序biber也更强大,能突破bibtex的局限。
\end{CJK}}
So, what are the main differences?\footnote{See also \url{http://tex.stackexchange.com/questions/25701/bibtex-vs-biber-and-biblatex-vs-natbib}}
\begin{itemize}
\item There is a big difference in the essential way in which
  \biblatex\ and traditional \bibtex-based systems work. In
  particular, in \biblatex\ the external program is used only to
  prepare data: the formatting and output of that data is largely
  handled using \LaTeX\ code, whereas in \bibtex, most of the
  formatting is done by \bibtex, producing a bibliography which is
  more-or-less ready to be typeset as it is.
\item The practical consequence of that is that \biblatex\ can do
  things that traditional \bibtex\ cannot: in particular it can
  respond \emph{dynamically} to context in a way that traditional
  \bibtex\ cannot, or cannot easily.
\item This means that \biblatex\ can be used for citation systems
  (such as traditional humanities-style systems) for which \bibtex\
  alone is unsuited. It is a more powerful and flexible system.
\item In addition, the \package{Biber} program -- which is recommended
  to replace \bibtex\ in sorting and handling the bibliographical data
  itself -- is more flexible and powerful than \bibtex\ is, and
  doesn't suffer from some of the \bibtex's historical limitations.
\end{itemize}
\indexstop{Biblatex!compared to bibtex}

You may well be asking: should I use \biblatex? Most users probably
fall into one of three groups.
\marginpar{\begin{CJK}{UTF8}{gbsn}\footnotesize 那么我是否要用biblatex呢?
如果要提交一些有固定格式要求且提供bibtex样式和文档模板的文章,那么不需要使用biblatex,因为要将转到biblatex是件麻烦事。如果拥有传统的bibtex样式,并且满足需求,也没有改变的需要,那么就不需要换到biblatex。但如果你需要生成精致复杂的参考文献,需要引用非传统标准的文献,需要进行非ASC码的排序,需要使用多语言,那么就需要使用biblatex。
\end{CJK}}

Some people \emph{need} to stick with traditional \bibtex. For
instance, if you are submitting work to a journal which has a \bibtex\
style which it requires you to use, then you should \emph{not} switch
to \biblatex, at least for that paper. Many journals are in just that
position, and very few (if any) have adopted \biblatex. And if you
frequently or sometimes need to use \bibtex, you should try to keep
your |.bib| files compatible with it (for instance, continue to use
|year|, |journal| and |address| rather than |date|, |journaltitle| and
|location| fields).

Some people \emph{don't need} to switch. If you have existing \bibtex\
styles which work for you, and do everything you need, then there is
no reason to change. You can. But you don't need to. The existing
programs have the advantage of stability in such cases. On the other
hand, switching will accustom you to \biblatex\ and put its more
powerful features at your disposal. It's really a matter of
taste. Those who don't have an existing heavy investment in \bibtex\
should probably prefer \biblatex. But don't make such a change a week
before you are due to submit your doctoral thesis.

Some people \emph{need} \biblatex. This is probably true if you work
in the humanities and need sophisticated and complex styles such as
traditional footnote-based citations systems, if you want to cite
non-standard sources, if you need to handle large volumes of non-ASCII
characters or have complex sorting requirements, if you need to work
in multiple languages.

\subsection{The required changes}

\indexstart{Biblatex!compared to bibtex}
The main differences between using \biblatex\ and `standard' \bibtex\
are as follows:\label{bibtex:simple:eg}

\marginpar{\begin{CJK}{UTF8}{gbsn}\footnotesize 改用biblatex需要注意一下几点:
1. 需要加载biblatex包
2. 需要在加载时制定样式
3. 需要用命令加载参考文献数据
4. 需要使用biblatex自己的文献表打印命令
5. 最重要的是要理解传统的bibtex的样式bst文件在biblatex中无法使用。要得到与标准bst样式一致的文献表,那么可以使用biblatex提供的标准样式替代。
\end{CJK}}

\begin{itemize}
\item You need to load \biblatex\ as a package using \cs{usepackage}
  (generally with various options). It's generally wise to load
  \package{csquotes} as well.
\item Instead of \cs{bibliographystyle}, you specify the style to be
  used as an option when loading \biblatex:
\begin{pseudoverb}\centering
  \cs{usepackage}[style=\ldots]\{biblatex\}
\end{pseudoverb}
\item Instead of specifying the \verb|.bib| file as an argument to the
  \cs{bibliography} command, you use the command
  \cs{addbibresource\{\}} to identify the file(s). You specify the
  file(s) to be used completely, including their \verb|.bib|
  suffixes. So if your bibliography file is called \verb|mybib.bib|,
  you have
\begin{pseudoverb}
  \centering \cs[mybib\textbf{.bib}]{addbibresource}
\end{pseudoverb}
\item Instead of \cs{bibliography}, you use \cs{printbibliography} at
  the point where the bibliography should be printed.
\end{itemize}

One important point to understand: existing \bibtex\ \emph{styles}
cannot be directly used by \biblatex.\sidenote{There is a project to
  provide \biblatex\ styles that are identical to the standard
  \bibtex\ ones: the \package{biblatex-trad} package. Development
  seems to have been sporadic.}  The `standard' styles in \biblatex\
do not precisely correspond to the standard styles in \bibtex.
\indexstop{Biblatex!compared to bibtex}

\subsection{An example}

Find an existing \verb|.bib| file of your own (or, if you are pushed
to find one, make use of a standard file such as
\verb|biblatex-examples.bib|, which will be installed with \biblatex).

\marginpar{\begin{CJK}{UTF8}{gbsn}\footnotesize 给出一个示例
\end{CJK}}


Try the following sample document: \clearpage
\marginnote{\tikz[overlay]{\draw (0,-0.5em) edge [out=180, in=25,
    -stealth] ([shift={(0.5ex,2.3ex)}] pic cs:csquote)}Although not
  essential, \package{csquotes} should normally be loaded.}
\marginnote[2ex]{\tikz[overlay]{\draw (0,-0.5em) edge [out=180, in=90,
    -stealth] ([shift={(-0.5ex,2.3ex)}] pic cs:style)}Style is
  specified when loading the package, not using
  \cs{bibliographystyle}.}  \marginnote[0.2ex]{\tikz[overlay]{\draw
    (0,-0.5 em) edge [out=180, in=0, -stealth] ([shift={(1.2ex,
      +1ex)}] pic cs:resource)}Instead of identifying the
  \texttt{.bib} files in the \cs{bibliography} command, they are
  identified this way.}  \marginnote[0.2ex]{\tikz[overlay]{\draw
    (0,-0.5 em) edge [out=195, in=45, -stealth] ([shift={(1.2ex,
      +1ex)}] pic cs:printbib)}Instead of \cs{bibliography},
  \cs{printbibliography} is used to print the reference list.}
\begin{pseudoverb}
\cs[article]{documentclass}\\
\colorbox{green!30}{\cs[csquotes]{usepackage}\tikzmark{csquote}}\\
\cs{usepackage}[backend=bibtex, \colorbox{red!30}{style=\tikzmark{style}numeric}]\{biblatex\}\\
\colorbox{blue!25}{\cs[biblatex-examples.bib]{addbibresource}\tikzmark{resource}}\\
\cs[document]{begin}

\cs[*]{nocite}

\colorbox{green!30}{\cs{printbibliography}\tikzmark{printbib}}

\cs[document]{end}
\end{pseudoverb}

If you run \LaTeX, \bibtex, and \LaTeX\ again, you should find that a
numbered bibliography is produced. Now see if you can get it to work
with \package{Biber}. Replace line 2 with
\begin{center}
\cs{usepackage[backend=biber, style=numeric]\{biblatex\}}
\end{center}
and delete \verb|.aux|, \verb|.bbl| and \verb|.blg| files. Now run
\LaTeX, \package{Biber} and \LaTeX\ again. You should generate the
same document. (You might also like to try the sample document
suggested for complete neophytes, which is at page
\pageref{neophyte:example}.) Although \biblatex\ can use \bibtex\ to
prepare its data, it's generally better to use \package{Biber}, which
is a more modern program with many advantages.

\subsection{The \package{Natbib} package}

\index{Natbib}

\marginpar{\begin{CJK}{UTF8}{gbsn}\footnotesize \package{Natbib}宏包的问题。biblatex提供了\package{Natbib}宏包的替代方案,它提供一个\package{Natbib}模块,使用户可以无缝衔接使用类似Natbib提供的命令。但实际上最好的思路是习惯使用biblatex自身提供的命令,因为它基于样式可以灵活地定制。
\end{CJK}}
Many \LaTeX\ users use the \package{Natbib} package. This is a sort of
half-way house between \bibtex\ and \biblatex. It's somewhat more
flexible than a pure \bibtex\ solution, and has (in particular) a
wider range of citation commands to deal with author\slash year
citation systems. But it still uses \bibtex\ under the hood, and it
doesn't have \biblatex's flexibility.

Biblatex has a \package{Natbib} `compatibility mode'.
If you load \biblatex\ with the option \texttt{natbib} (or
\texttt{natbib\allowbreak =\allowbreak true}), then it will let you
use some \package{Natbib}-like citation commands, like \cs{citet} and
\cs{citep}. However, the compatibility is really only skin-deep;
it hardly\footnote{It does modify the punctuation used to separate an
author's name from the year to match the \package{Natbib} default.}
extends beyond the citation commands, and the actual formatting of the
citations (which will depend on the style you specify) will be
determined by \biblatex. In general, it's probably a better idea to
get used to the \biblatex\ citation commands.
%%% Local Variables:
%%% coding: utf-8
%%% mode: LaTeX
%%% TeX-master: "biblatex-tutorial"
%%% End:
