\chapter{Languages}\label{ch:languages}

Dealing with different languages is not straightforward; but
\biblatex\ is fairly well set up for the task. This section aims to
provide basic information; there is actually considerably more on
offer in \biblatex\ for sophisticated language use than this section
deals with. This section basically assumes the `normal' user who
produces documents in a single language, but may occasionally need to
deal with bibliographic sources in other languages.

\index{languages!possible changes}
There are a number of different ways that a bibliographic system may
need to adjust to different languages. Most basically, a number of
words and abbreviations may need to change: `ed' in English becomes
`éd' in French, or `Hrsg' in German. There may be differences in
punctuation (for instance in how quotations are typeset). There may be
differences in sorting. (For instance, is `Aardvark' a word you would
expect to appear at the beginning or the end of a dictionary? An
English person and a Swede would disagree!) At a still more basic
level -- unseen by the user until it goes wrong -- the hyphenation of
different languages varies. And, even further `under the skin' there
are issues about the `encoding' of an input file: how human-readable
glyphs are turned into machine-readable sequences of numbers.

\section{A dogmatic approach}

\index{languages!recommendations}
Later sections of this chapter try to provide a better explanation of
how all these various things can be done. But here is a dogmatic `just
do this' approach, which will work in many cases.

\begin{enumerate}
\item If you are writing in a language which requires accents and
  diacritics, prefer to use \smallcaps{utf-8} encoding in your source
  and |.bib| files. Set your editor to use that encoding and, unless
  you are using Xe\TeX\ or Lua\TeX\ (which handle unicode natively),
  use the \package{inputenc} package in your source file.
 \begin{pseudoverb}
 \cs{usepackage}[utf8]\{inputenc\}
 \end{pseudoverb}
\item Load \package{babel}, or, if you use it,
    \package{polyglossia}, with your preferred
  language.\marginnote{\newcommand{\english}{{\normalfont
        (English)}}The language options are: \ttfamily american
    \english, australian \english, austrian, brazil, british \english,
    canadian \english, canadien {\normalfont(French)}, catalan, czech,
    danish, dutch, finnish, french, german, greek, italian, naustrian,
    ngerman, norsk, nynorsk, portugues, russian, spanish, swedish.}
 \begin{pseudoverb}
   \cs{usepackage}[\angled{language}]\{babel\}
 \end{pseudoverb}
 This should be done, and any language selection made, \emph{before} loading \biblatex.
\item Load \package{csquotes} with an appropriate quotation style:
 \begin{pseudoverb}
   \cs{usepackage}[style=\angled{language}]\{csquotes\}
 \end{pseudoverb}
\item Load \biblatex\ with the style |autolang=hyphen|, which will
  make sure that appropriate hyphenation patterns are used based on
  the |langid| field of the entry, if it has been set. Set that field
  where you think it will be helpful.
\item If you are using anything other than the standard bibliography
  style, consult that style's documentation to find out whether you
  need to `connect' your language to a particular style-specific
  language style. If so, do that using
 \begin{pseudoverb}
   \cs{DeclareLanguageMapping}\{\angled{language}\}\{\angled{language-definition}\}
 \end{pseudoverb}
 For instance, in the \package{APA} style, I might have
 \begin{pseudoverb}
   \cs{DeclareLanguageMapping}\{british\}\{english-apa\}
 \end{pseudoverb}
\end{enumerate}

\section{Some more detail}

\subsection{Encodings}

\index{encoding}
\biblatex\ (or, more accurately, \package{Biber}) is perfectly
happy working with unicode (indeed, internally this is what it always
tries to do). It is generally best, unless your use of accents and
diacritics is minimal, to use a \smallcaps{utf-8} encoded |.bib|
file. This need not be the encoding used in your |.tex| file: the
software will attempt to detect that, and will output a |.bbl| file
accordingly. But in general it makes sense, nowadays, to use a
\smallcaps{utf-8} encoded |.tex| source. For this, you should either
use Xe\TeX\ or Lua\TeX, which handle unicode natively, or use the
\package{inputenc} package to set the encoding to |utf8|.

Occasionally, notwithstanding such precautions, there can be
difficulties caused by the way \TeX\ and \package{Biber} deal with
encodings. In such cases it may be necessary to set input and output
encodings explicitly; on this, you should consult the \package{Biber}
documentation.

\subsection{Language}

In general, in the \LaTeX\ world, language features are controlled
using the (older) \package{babel} or (newer) \package{polyglossia}
packages to set languages. The \biblatex\ package generally tries to
leverage these packages to work out what language is being used. It
then attempts to load a file which contains suitable bibliographical
features in that language.

This is in many cases enough. But internally the position has to be a
bit more complex, because \biblatex\ has to associate a given basic
language with a set of files which define suitable abbreviations,
formatting of dates, and so forth for that language. In the standard
styles, this is entirely automatic. In non-standard styles, there may
be cases where the style is language-specific, or in which you are
otherwise required to tell \biblatex\ explicitly what language
definitions to use.

This is the purpose of the \cs{DeclareLanguageMapping} command. What
it does is tell \biblatex\ what definitions to use \emph{for a
  particular language}. So, for instance
\begin{pseudoverb}
  \cs{DeclareLanguageMapping}\{british\}\{british-apa\}
\end{pseudoverb}
tells \biblatex\ that if the language being used for typesetting is
|british|-English, then it should look for the relevant definitions in
the |british-apa| langugage definition. Sometimes this has been done
for you by the writer of the bibliography style; sometimes it needs to
be explicitly done (especially if the style provides a limited range
of language definitions).

\indexstart{languages!multiple}
It is perfectly permissible to have a document in more than one
language: the citation and bibliography commands will use whichever
language is active at the relevant time. So, for instance (to take a
thoroughly feeble example) suppose we had the following:

\begin{Verbatim}[frame=single]
...
\usepackage[french,british]{babel}
\usepackage[style=numeric]{biblatex}
\addbibresource{biblatex-examples.bib}

\begin{document}
We make reference to Aristotle \cite{aristotle:anima}.

\printbibliography

\selectlanguage{french}

\printbibliography
...
\end{Verbatim}

\begin{figure*}
\fbox{\includegraphics{./examples/duallanguage.pdf}}
\vspace{3pt}%
\caption{A document using two languages\label{dual:language}}
\end{figure*}

That code will produce something which looks like figure
\ref{dual:language}. The first time the bibliography is printed,
English conventions are used, because English is the active
language. Then French is selected as the language, and this time the
bibliography, when printed, is formatting according to French
conventions and terminology.

In that example, the language is being selected explicitly in the
source file. But what about the |.bib| entries themselves? Can they
select languages? The short answer is, Yes. But it's really quite
important to give a longer answer.

A |.bib| entry may have the field |language| set. But this will not
affect the language conventions used to typeset it. The purpose of the
|language| and |origlanguage| fields is to record
\emph{bibliographical} data, which may (depending on the style) be
printed.\sidenote[][-4ex]{For instance, a style might print:
  Aristotle, \emph{De Anima} (Greek).} So |language| and
|origlanguage| relate to the \emph{source} not the
\emph{entry}. However ther is a field
|langid|\footnote{\texttt{langid} was previously called
  \texttt{hyphenation}.} which is specifically designed for this
purpose. You set the |langid| field to indicate that the language of
the entry should (potentially) be fixed in a particular way. For
instance, if you are citing an author and a title in a foreign
language, it may be natural to specify an appropriate language ID:

\begin{verbatim}
@book{swann,
  author     = {Proust, Marcel},
  maintitle  = {À la recherche du temps perdu},
  volume     = 1,
  title      = {Du côté de chez Swann},
  date       = {1913},
  langid     = {french},
}
\end{verbatim}

\index{database!langid field@\texttt{langid} field}
However, simply specifying the language ID does not directly make any
difference: the entry simply gets printed in the ordinary way (see
figure \ref{proust1}).
\begin{figure}
\framebox{\includegraphics{./examples/proust1}}
\caption{The language ID: not necessarily any difference\label{proust1}}
\end{figure}

However, you can make it make a difference by using an option when
\biblatex\ is loaded:
\begin{itemize}
\item |autolang=hyphen| will activate hyphenation rules for the
  specified language. This may or may not make any difference
  (depending on line breaks and so forth). In our example it would
  not. But it's generally a good idea to allow hyphenation of titles
  and so forth correctly.
\item |autolang=other| will apply the \emph{full} set of rules for the
  language in question when printing that particular entry, as can be
  seen in figure \ref{proust2}. Note in this case that the
  bibliography in general is in English format (the title is
  `References' not `Références', and so forth), but the particular
  entry takes a French format.
\end{itemize}

\begin{figure}
\fbox{\includegraphics{./examples/proust2}}
\caption{\texttt{autolang=other}\label{proust2}}
\end{figure}

In general terms, it seems to me, activating language-specific
hyphenation is normally a sensible idea, but activating full language
support seems to me generally a rather peculiar thing to do in general
(though there might be some multilingual works where it makes sense).

There are further and deeper complexities for users of
\package{polyglossia}, for whom still more complex features are
available. But those are beyond the scope of this book.
\indexstop{languages!multiple}

\subsection{Sorting}

\index{sorting!language specific}
There is a tendency to consider alphabetical order, drilled into us as
children, as if it were fixed and immutable. But in fact alphabetical
order can vary from language to language. For instance, in Czech and
Slovak accented letters are conventionally sorted after unaccented
letters (so \'{E}mil comes after Emily), in Norwegian `Aa' is
considered the same as \r{A}, and placed at the end of the alphabet
(so Aardvark comes after Zebra!).

This sort of detail is under the control of \package{Biber}, not of
\biblatex. Normally you can assume that an appropriate locale will be
set based on environment variables in your computer. But you may
sometimes need to specify the locale. You can do this by specifying a
\texttt{sortlocale} as an option when loading \biblatex. So, for
instance, to have a Norwegian sorting scheme, you would specify
\begin{verbatim}
\usepackage[sortlocale=nb_NO]{biblatex}
\end{verbatim}

You should\footnote{Though at the time of writing \package{Biber}
  seems to be broken.} see the effect of this in figures
\ref{zebra:en} and \ref{zebra:no}. The bibliography used is the same
in each case, but figure \ref{zebra:en} is sorted using
\texttt{sortlocale=en\_GB}, whereas figure \ref{zebra:no} is sorted
using \texttt{sortlocale=nb\_NO}.

\begin{figure}
\fbox{\includegraphics{./examples/sorting-eg1.pdf}}
\caption{Sorted with \texttt{sortlocale=en\_GB}\label{zebra:en}}
\end{figure}

\begin{figure}
\fbox{\includegraphics{./examples/sorting-eg2.pdf}}
\caption{Sorted with \texttt{sortlocale=nb\_NO}\label{zebra:no}}
\end{figure}

\section{Limitations}
Although \biblatex\ has many features designed to deal with different
languages, there are still limits. The limits are, as things stand,
reached when different languages have fundamentally different ideas
about the sort of information that should be output, or the order in
which it shoudl be output. To cope with this it is necessary to carry
out deep changes to the bibliography drivers, and the existing styles
do not support this.

%%% Local Variables:
%%% coding: utf-8
%%% mode: LaTeX
%%% TeX-master: "biblatex-tutorial"
%%% End:
