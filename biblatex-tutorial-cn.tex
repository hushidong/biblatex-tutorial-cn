\documentclass[nobib,hyphens]{tufte-book}
\PassOptionsToPackage{cmyk}{xcolor}
\usepackage[utf8]{inputenc}
\usepackage{CJKutf8}
%\usepackage[encapsulated]{CJK}%可以与下一句替换
\input{zhwinfonts}
\usepackage{tikz}
\usepackage{graphicx}
\usepackage{enumitem}
\usepackage{rotating}
\usepackage{etoolbox}
\usepackage{pdfpages}
\usepackage{xparse}
\usepackage{standalone}
\usepackage{imakeidx}
\usepackage[geometry]{ifsym}
\usepackage{amssymb}
\usepackage{booktabs}
\usepackage{tabularx}

\usetikzlibrary{trees}
\usetikzlibrary{tikzmark}
\tikzstyle{every picture}+=[remember picture]

\usepackage{fancyvrb}
\DefineShortVerb{\|}

\newcommand{\package}[1]{{\sffamily #1}}
\newcommand{\biblatex}{\package{Biblatex}}
\newcommand{\bibtex}{\package{{\scshape bib}\TeX}}
\newcommand{\prefacedate}{\today}
\newcommand{\gives}{\ensuremath{\rightarrow}}
\newcommand{\angled}[1]{%
  \ensuremath{\langle}%
  {\normalfont\itshape #1\/}%
  \ensuremath{\rangle}}
\newenvironment{pseudoverb}{\begin{list}{}{}\ttfamily\item}{\end{list}}
\DeclareDocumentCommand{\cs}{O {} O {} m}
{\begingroup
  \ttfamily
\textbackslash
  #3%
  \ifblank{#2}%
    {\ifblank{#1}
      {}%
      {\{#1\}}}%
    {[#1]}%
  \ifblank{#2}%
    {}%
    {\{#2\}}%
  \endgroup}
\newcommand{\URL}{\smallcaps{url}}
\newcommand{\braced}[1]{\texttt{\{}#1\texttt{\}}}
\newcommand{\intref}[2][0pt]{%
  \marginnote[#1]{
    \raisebox{-1.5pt}{\color{BrickRed}\FilledTriangleRight}%
    \,\parbox[t]{\marginparwidth - 1em}{#2}}}
\newcommand{\manref}[1]{%
  \intref{\emph{Manual}~#1}}

\newcommand{\packindex}[1]{%
  \index{#1@\package{#1}}}
\makeatletter
\newrobustcmd{\indexstart}[1]{%
  \def\@tempa{#1\string|(}%
  \protected@edef\theindexentry{%
    \unexpanded{\index}
    {\@tempa}}%
  \theindexentry}
\newrobustcmd{\indexstop}[1]{%
  \def\@tempa{#1\string|)}%
  \protected@edef\theindexentry{%
    \unexpanded{\index}
    {\@tempa}}%
  \theindexentry}
\newrobustcmd{\csindex}[1]{%
  \def\@tempa{#1@\cs{#1}}%
  \protected@edef\theindexentry{%
    \unexpanded{\index}
    {\@tempa}}
  \theindexentry}
\makeatother

\setcounter{secnumdepth}{0}

\makeindex[columns=2]

\newcommand{\releasedate}{2017}

%%%%%%%%%%%%%%%%%%%%%%%%%%%%%%%%%%%%%%%%%%%%%%%%%%%%%%%%%%%%%%%%%%%%
% INCLUSIONS


 \includeonly{quickstart,
             frontmatter,
             introduction,
             database,
             customize1,
             bibstyles,
             citationcommands,
             bibliographyformat,
             subdivisions,
             sorting,
             languages,
             recipes,
             tools,
             troubleshooting,
             styleeg,
             cheatsheet,
             indexes,
              }
%%%%%%%%%%%%%%%%%%%%%%%%%%%%%%%%%%%%%%%%%%%%%%%%%%%%%%%%%%%%%%%%%%%%%

\title{Biblatex}

\author{Paul Stanley}

\date{2017}

\begin{document}
\begin{CJK}{UTF8}{zhsong}

% The following code to deal with subtitles is adapted from
% Ryan Spencer's post at
% https://groups.google.com/forum/#!topic/tufte-latex/fPP20IOzPP0
\makeatletter
\newcommand{\subtitle}[1]{
  \gdef\@subtitle{#1}}

\renewcommand{\maketitlepage}[0]{%
  \cleardoublepage%
  {%
   \sffamily%
   \begin{fullwidth}%
     \fontsize{18}{20}\selectfont\par\noindent\textcolor{darkgray}%
     {\allcaps{\thanklessauthor}}%
   \vspace{11.5pc}%
   \fontsize{24}{40}\selectfont
   \par\noindent\textcolor{darkgray}{\allcaps{\thanklesstitle}}%
    \ifdefined\@subtitle
     {\fontsize{14}{38}\selectfont
      \par\noindent\textcolor{darkgray}{\allcaps{\@subtitle}}}%
    \fi
      \vfill%
    \fontsize{14}{16}\selectfont\par\noindent\allcaps{\releasedate}%
    \end{fullwidth}%
    }
    \thispagestyle{empty}%
    \clearpage}
\makeatother

\subtitle{简易读本}%An Easier Read

\frontmatter
\maketitle

\strut\vspace{10cm}

Copyright \textcopyright\ Paul Stanley \releasedate. The moral right of the author is asserted.

\vspace{2pc}

{\small This handbook is released under a CC Attribution-ShareAlike
  license CC BY-SA:
  \url{http://creativecommons.org/licenses/by-sa/3.0/}. In summary,
  you are free to
\begin{itemize}
\item Share --- copy and redistribute the material in any medium or
  format
\item Adapt --- remix, transform, and build upon the material for any
  purpose, even commercially.
\end{itemize}
The licensor cannot revoke these freedoms as long as you follow the
license terms.

Under the following terms:
\begin{itemize}
\item Attribution --- You must give appropriate credit, provide a link
  to the license, and indicate if changes were made. You may do so in
  any reasonable manner, but not in any way that suggests the licensor
  endorses you or your use.
\item ShareAlike --- If you remix, transform, or build upon the
  material, you must distribute your contributions under the same
  license as the original.
\end{itemize}
}

\vspace{2pc} Typeset using \LaTeX\ in the \package{tufte-book} class,
using Palatino, Helvetica, and Bera Mono fonts.  \cleardoublepage

\tableofcontents

%\chapter{Preface}
\chapter{序}

The \biblatex\ package is a tour de force by its originator (Philip
Lehmann) and its current maintainer(s) (Phil Kime --- who is also
responsible for \package{Biber} --- assisted by others). It is
powerful. But with power comes complexity. The manual is a mine of
information, but sometimes rather overwhelming.

\marginpar{\begin{CJK}{UTF8}{gbsn}
\footnotesize 介绍作者的动机,相比于全面系统的biblatex手册,希望提供一个介绍性的、实用性的文档,重点解释 biblatex 怎么运作,发掘可能的选项和技巧,来帮助一般用户完成相当\textit{标准的}任务。主要面向一般用户而不是样式作者,便于读者获得一些解决常见问题的答案。
\end{CJK}}
My aim here has been to write something that is a bit more than a mere
introduction, but certainly not a systematic manual. That, after all,
\biblatex\ already has. It is supposed to be, above all, practical:
focussed on explaining not how \biblatex\ works, or exploring all its
possible options and wrinkles, but trying to show how ordinary users
can use it to accomplish reasonably `standard' tasks. It is not
intended to replace the manual; indeed, I have assumed that the reader
will have that to hand. It is not intended as an advanced book for
bibliography-style writers. It is aimed at the ordinary user, who is
looking for practical advice about everyday issues. I hope it will be
useful.

It is a pleasure to acknowledge the help of all those, particularly
those who participate in \TeX-StackExchange, who have directly or
indirectly helped in so many ways. I am also grateful to all those who
have commented on this guide's Github project page. I welcome all
comments and suggestions.

At the time of writing, the version of \biblatex\ on my system is 3.7.

\hfill\smallcaps{PMS}

\hfill{London, \prefacedate}

\vspace{1ex}

\hfill{\texttt{pstanley@essexcourt.com}}

%%% Local Variables:
%%% coding: utf-8
%%% mode: LaTeX
%%% TeX-master: "biblatex-tutorial"
%%% End:


\mainmatter




%\chapter{What is \package{Biblatex}?}
\chapter{Biblatex 是什么?}
\label{ch:introduction}
\marginpar{\begin{CJK}{UTF8}{gbsn}\footnotesize Biblatex 是个什么包?
\end{CJK}}

This chapter has two different sections, intended for different
audiences. If you are new to automated bibliography tools in \LaTeX,
then you should read the first section.\intref{p~\pageref{newbie}} If
you already have experience of using automated bibliography tools in
\LaTeX, such as \bibtex\ and \package{Natbib}, then you can probably
turn straight to the second section.\intref{p~\pageref{expert}} If you
are prefer to jump in and start getting your hands dirty without much
explanation, there is a quick start section at the back of this
guide,\intref{p~\pageref{ch:quickstart}} along with an aide memoire to
the frequently used commands and
options.\intref{p~\pageref{ch:cheatsheet}}

%\section{For the true neophyte}
\section{给纯新手}

\subsection{The basic idea}
\label{newbie}
\indexstart{Biblatex!very basic introduction}
\marginpar{\begin{CJK}{UTF8}{gbsn}\footnotesize 给新手的基本概念
\end{CJK}}

Academic writing usually cites sources.
\marginpar{\begin{CJK}{UTF8}{gbsn}\footnotesize 学术写作需要引用来源,不同的出版商、期刊、个人的写作对于引用的形式要求都是不同的。但文献引用底下的原理是类似的。在正文或脚注等地方引用文献以标注来源,并在合适的地方常用列表的形式给出来源的详细信息。

可以手动给出这些信息,但耗时、费力、容易出错,并且难以应付随时需要的改变。
\end{CJK}}
The exact form of these
citations varies: different disciplines have different practices, and
different publishers and journals too. But the underlying idea is
similar: there are references in the text (or footnotes) that enable
the reader to identify a source for a particular statement or
quotation. This may be a number in a list of references (like [1]),
or a combination of author and year (like Author 2014), or even a
footnote which identifies the source.\footnote{Like: Author, `Paper
  Title' in \emph{Learned Journal}, Vol.\ 10 (2014), p.\ 111.}
Alongside this there is a bibliography: a list of all the sources
which have been cited, which will nearly always enable the reader to
look up the source in a library, and sometimes provide (in itself)
useful information about the source, such as its date, or where
and how it was published.

It is, of course, possible to produce all these indications entirely
`by hand': typing out references, bibliographies and so forth. But
there are disadvantages to doing that:
\begin{itemize}
\item It is time-consuming. If you write lots of papers, you will
  probably find yourself citing the same sources again and again. It's
  tedious to have to keep looking up and typing out the same data. You
  have to spend time doing things like organising your bibliography
  into the proper order.
\item It is error-prone. Every time you type data up, you risk making
  a mistake: either a mistake of substance (typing `Stanely' for
  `Stanley'), or in the way the data is formatted.
\item It can be frustrating. Each citation system has its own picky
  requirements. One wants the titles of articles in double quotes,
  ``Like this''. Another wants them in single quotes, `Like
  this'. Another wants them in italics \emph{Like this}. There are
  specialists who revel in such details: but for the average person
  they are just a nuisance, and it is nice to have a system which
  takes care of these details for you.
\item The details sometimes change as you write. For instance, if you
  use a system where references are numbered, adding a reference may
  change all the numbering. Or if you follow a system which uses
  abbreviations such as `ibid', `op.\ cit.' or references back to
  earlier notes, adding a single footnote can throw everything out. In
  such systems it's not possible to finalise references until the
  document is complete: and every change can easily introduce errors.
\end{itemize}

The idea of a citation system is to \emph{separate content from
  presentation}.
\marginpar{\begin{CJK}{UTF8}{gbsn}\footnotesize 参考文献引用系统的逻辑是,把文献来源的重要细节比如作者、题名等放到一个数据库文件中。写作时对这些文献做一个简短的引用,然后由计算机从数据库中抽取信息,然后插入文档中。基于这种方式,作者完全可以找到并利用很多已有的文献数据库。
\end{CJK}}
You record the important details about a source in a
database file: the author(s), title, date, publisher and so forth. As
you write, you use a short reference to indicate the work you want to
cite at that point in your paper. And then you let the computer take
care of extracting the information from your database, then inserting
and formatting it into your paper. And in fact, in many fields, you
can find existing databases which you can borrow from which will
contain data you need.

This is what \biblatex\ does.
\marginpar{\begin{CJK}{UTF8}{gbsn}\footnotesize 这就是 biblatex 做的事情。用户可以维护好一个数据库。然后在写作需要的时候引用其中的文献,在文档编译是,有biblatex 来获取文献的信息并格式化,最后形成引用的标注标签和文献表。
\end{CJK}}
You maintain a database which contains
the works you may want to cite. The database can be large: it needn't
include only the works that you want to cite in a particular paper ---
it can be as large as you like. It can be in any order. All that
matters is that it contains the information that is needed to
`construct' the citations.

As you write your paper, you insert commands to tell \biblatex\ that
you want to cite particular sources drawn from that database.

Then, when you are done, as part of the compilation of your \LaTeX\
source, you get \biblatex\ to extract the relevant information, format
it, and construct the citations and bibliography in proper format.

\subsection{In practice}
\marginpar{\begin{CJK}{UTF8}{gbsn}\footnotesize 实践中需要几个工具以一定的流程来实现目标。首先至少需要两个文件一个是tex源文件(可能用include和input包含多个文件),一个是文献数据库文件(可能也有多个数据库)。
\end{CJK}}

In practice, getting all this to happen involves a number of different
tools and programs, which all have to work together.

You are going to start with two files:\footnote{At least two, since
  your \LaTeX\ source could \cs{include} or \cs{input} more than one
  file, and your database may consist of more than one file too.} a
database file and your \LaTeX\ source file.

\marginpar{\begin{CJK}{UTF8}{gbsn}\footnotesize 数据库文件后缀为.bib,内容包含文献来源的详细信息,并以特定的格式(见\ref{ch:database}章)描述。它是一个普通的文本文件,可以利用任何编辑器编写,或利用一些辅助工具来维护
(见\ref{ch:tools}章)。
\end{CJK}}
The database file conventionally has the suffix \texttt{.bib}, and
consists of a set of records containing information about the sources
you may wish to cite. The format of such a file is described in
chapter \ref{ch:database}. It is a plain text file, which you can
either produce by hand (using a text editor, such as the one you use
to write \LaTeX\ source code) or can be maintained by one of a number
of available `helper' programs.\intref{Chapter \ref{ch:tools}}

\marginpar{\begin{CJK}{UTF8}{gbsn}\footnotesize \LaTeX\ 源文件后缀为.tex,内容为\LaTeX\ 格式的文本。为利用biblatex引用文献,包含如下内容:

1. biblatex宏包加载语句

2. 加载文献数据源的语句

3. 引用文献的语句

4. 打印文献表的语句
\end{CJK}}
The \LaTeX\ source file conventionally has the suffix \texttt{.tex}
and contains all your text, marked up in \LaTeX\ format. It also
contains:
\begin{itemize}
\item A\marginnote{\cs{usepackage[style=...]\{biblatex\}}} line that
  loads \biblatex\ and tells it what `style' to use for citations and
  bibliography.
\item One\marginnote{\cs{addbibresource\{file.bib\}}} or more lines
  which tell \biblatex\ what file(s) to use as the database from which
  information is to be extracted.
\item One\marginnote{\cs{cite\{...\}}} or more commands which tell
  \biblatex\ that you want to \cs{cite} a particular source at that
  point in your text.
\item (Usually)\marginnote{\cs{printbibliography}} one or more
  commands that tell \biblatex\ to print a bibliography at that point
  in the document.
\end{itemize}

The next task is to get the various tools to cooperate together to get
to the end result that you need.
\marginpar{\begin{CJK}{UTF8}{gbsn}\footnotesize 下一步就是利用不同工具来配合得到需要的结果,一般需要如下步骤:

1. \LaTeX\ 第一遍编译tex源文档

2. 利用工具分析文献数据库,抽取数据。可以使用多种工具比如 bibtex,biber 等,对于 biblatex 主要使用 biber。

3. 再次利用\LaTeX\ 编译源文档

4. 有时需要多运行一次\LaTeX\ 来完善文档内的超连接
\end{CJK}}
In general that requires at least the
following things to happen:
\begin{itemize}
\item Run \LaTeX\ first on the source file. At this point no citations
  are produced. Instead \biblatex\ \emph{records} a list of the
  sources that you have cited.
\item Run another program which reads the list that has been produced,
  analyses the database, and extracts the relevant information into a
  format that \biblatex\ can work with. There is a choice of programs
  that you might use for this, but in this handbook we will be
  assuming that you will use a program called \package{Biber}.
\item Run \LaTeX\ again, to `read in' the data that has now been put
  in digestible form, and produce the citations and bibliography.
\item (Sometimes), run \LaTeX\ yet \emph{again} to finalise
  cross-references.
\end{itemize}

This `pattern' (\LaTeX, \package{Biber}, \LaTeX, [\LaTeX]) is repeated
whenever you need to compile your document.\footnote{Not quite true:
  it is often unnecessary to re-run \package{Biber} if no new
  citations have been added. But it is never wrong to do so.}
\indexstop{Biblatex!very basic introduction}

\subsection{A quick demonstration\label{neophyte:example}}
\marginpar{\begin{CJK}{UTF8}{gbsn}\footnotesize 何不给出一个新手示例?假设读者已经安装一套tex发行版比如texlive,包含了所有编译需要的工具。下面第一步就是构建一个文献数据库文件。
\end{CJK}}
Why not try a quick example? I assume for these purposes that you have
a fully-equipped and functional \TeX\ system, and have installed the
pre-requisites for \biblatex. This will be true on any of the standard
modern \TeX\ installations; but it's not a bad idea to update to at
least the most recent versions of \biblatex\ and \package{Biber},
since both are under active development.

First, let's set up our `database'. It's not going to be much: just a
single book.

Open a text editor, and produce a new file
\texttt{handbook.bib}:\intref{Much more about how to write
  \texttt{.bib} files is explained in chapter~\ref{ch:database}.}

\begin{verbatim}
@book{nussbaum:95,
  author = "Nussbaum, Martha C.",
  title = "Poetic Justice",
  subtitle = "The Literary Imagination and Public Life",
  publisher = "Beacon Press",
  location = "Boston",
  date = "1995",
}
\end{verbatim}

\marginpar{\begin{CJK}{UTF8}{gbsn}\footnotesize 然后利用文本编辑器写出一个tex源文档。并在其中引用文献,引用的命令详见第\ref{ch:citationcommands}章。
\end{CJK}}
Now open a text editor or your \LaTeX\ editor, and create a small test
file (I'll call mine \texttt{test.tex}):\intref[1.5in]{Much more about
  \cs{cite} commands is explained in
  chapter~\ref{ch:citationcommands}.}

\begin{verbatim}
\documentclass{article}
\usepackage{csquotes}
\usepackage[style=numeric]{biblatex}
\addbibresource{handbook.bib}
\begin{document}

As Nussbaum comments \cite[17]{nussbaum:95}:
\enquote{The utilitarian picture of human beings and
of rationality is familiar enough in theory}.

\printbibliography
\end{document}
\end{verbatim}

\marginpar{\begin{CJK}{UTF8}{gbsn}\footnotesize 接着开始根据前面介绍的步骤进行编译,注意每一步的输出结果。
\end{CJK}}
Now, run \LaTeX\ on the file, once. If you look at the resulting
typeset document, it should look something like figure
\ref{nussbaum1}.

\begin{figure}
\fbox{\includegraphics{./examples/nussbaum1u.pdf}}
\caption{Before running \package{Biber}}\label{nussbaum1}
\end{figure}

The reference `[\textbf{sebum:95}]' appears in square brackets
because although \LaTeX\ can `see' that there is going to be a
citation, it doesn't yet have the data that will enable it to
construct that citation, since \package{Biber} has not been run.

Now, from a terminal window opened in the same directory, run
\framebox{\parbox{\linewidth}{\texttt biber test}}

Hopefully, you will see a number of messages marked as `INFO', ending
with \texttt{INFO -- Output to test.bbl}.

Now run \LaTeX\ again on the source file |test.tex|. And if all is
going well, you should now see output like figure \ref{nussbaum2}.

\begin{figure}
\fbox{\includegraphics{./examples/nussbaum2u.pdf}}
\caption{After running \package{Biber} and \LaTeX}\label{nussbaum2}
\end{figure}

\marginpar{\begin{CJK}{UTF8}{gbsn}\footnotesize 最后可以看到,引用文献的信息已经在文献表中了。如果没有得到该结果,有可能是biber没有成功运行,这是可能在哪存在问题,可以按如下问题来检查:
1.加入文献源了么?2.编译每一步都完成了么?3.biber结果是否提示错误?4.引用来源的时候文献的关键字是否正确?5. bib文件是否包含数据?6.bib中的文献条目是否有效?具体检查方法详见第\ref{ch:troubleshooting}章。
\end{CJK}}
As you can see, the citation data has been pulled into the
bibliography. If the output remains unchanged (it looks like figure
\ref{nussbaum1}) that is because \package{Biber} has not successfully
run. That is \emph{probably} because there is some error in your
|.bib| file, so go back and check that. Any time you see this, you
should ask yourself these
questions:\marginnote{\textit{Debugging}\\1. \cs{addbibresource}?\\2. Run
  \LaTeX\, \package{Biber}, \LaTeX?\\3. \package{Biber} was
  error-free?\\4. Cite key is correct?\\5. \texttt{.bib} file contains
  source?\\6. Entry in \texttt{.bib} file valid?\\See
  chapter~\ref{ch:troubleshooting}.} (1) have I told \biblatex\ where
to find the bibliography file (using |\addbibresource{}|? (2) Have I
run \LaTeX\ then \package{Biber} then \LaTeX\ again? (3) Did
\package{Biber} report any errors? (If you like, you can check the log file at
\angled{jobname}|.blg|) (4) Is my citation \emph{key} correct (e.g.,
you haven't typed |nusbaum| for |nussbaum| in the \LaTeX\ source? (5)
Does the |.bib| file include that citation? (6) Is the syntax of the
citation correct?

\subsection{What next}

There is still quite a bit to learn, of course. To some extent it's up
to you how you go from here, but the following chapters take what I
think is a reasonably logical order.

%\section{For the more experienced}
\section{给有检验的用户}
\label{expert}
\marginpar{\begin{CJK}{UTF8}{gbsn}\footnotesize 给有经验的用户。
\end{CJK}}
\indexstart{Biblatex!compared to bibtex}

\marginpar{\begin{CJK}{UTF8}{gbsn}\footnotesize 从用户的角度使用 biblatex 的模式类似于以前习惯的用\bibtex 加\package{Natbib}的过程。也是把文献数据放到bib文件中,然后用工具按步骤编译。但需要注意源文档中使用的命令方面的差异。
\end{CJK}}
From the user's perspective, the basic pattern followed by \biblatex\
is similar to the one you will be accustomed to if you use \bibtex\
to produce \verb|.bst| files, or citation packages such as \package{Natbib}:
\begin{itemize}
\item Your bibliographical data is still stored in \verb|.bib| files,
  which have largely the same format as \bibtex\ files. (But
  \biblatex\ recognises additional entry types and fields.)
\item You still generate citation and bibliography data by running
  \LaTeX\, then an external program (either \bibtex\ itself or a more
  powerful replacement, \package{Biber}, and then \LaTeX\ again
  (sometimes twice).
\end{itemize}

There are a few, largely cosmetic, changes to the commands that you
need to include in your document to `set up' the bibliography. (See
page \pageref{bibtex:simple:eg}.)

\subsection{Differences}
\marginpar{\begin{CJK}{UTF8}{gbsn}\footnotesize 两者存在的差异主要包括:
1. 格式化文献数据的差异。biblatex用的是tex语言,而bibtex是右bibtex工具格式化。

2. biblatex可以做传统方法不能做的事,比如动态修改。这意味着它更强大,更灵活。其文献数据编译程序biber也更强大,能突破bibtex的局限。
\end{CJK}}
So, what are the main differences?\footnote{See also \url{http://tex.stackexchange.com/questions/25701/bibtex-vs-biber-and-biblatex-vs-natbib}}
\begin{itemize}
\item There is a big difference in the essential way in which
  \biblatex\ and traditional \bibtex-based systems work. In
  particular, in \biblatex\ the external program is used only to
  prepare data: the formatting and output of that data is largely
  handled using \LaTeX\ code, whereas in \bibtex, most of the
  formatting is done by \bibtex, producing a bibliography which is
  more-or-less ready to be typeset as it is.
\item The practical consequence of that is that \biblatex\ can do
  things that traditional \bibtex\ cannot: in particular it can
  respond \emph{dynamically} to context in a way that traditional
  \bibtex\ cannot, or cannot easily.
\item This means that \biblatex\ can be used for citation systems
  (such as traditional humanities-style systems) for which \bibtex\
  alone is unsuited. It is a more powerful and flexible system.
\item In addition, the \package{Biber} program -- which is recommended
  to replace \bibtex\ in sorting and handling the bibliographical data
  itself -- is more flexible and powerful than \bibtex\ is, and
  doesn't suffer from some of the \bibtex's historical limitations.
\end{itemize}
\indexstop{Biblatex!compared to bibtex}

You may well be asking: should I use \biblatex? Most users probably
fall into one of three groups.
\marginpar{\begin{CJK}{UTF8}{gbsn}\footnotesize 那么我是否要用biblatex呢?
如果要提交一些有固定格式要求且提供bibtex样式和文档模板的文章,那么不需要使用biblatex,因为要将转到biblatex是件麻烦事。如果拥有传统的bibtex样式,并且满足需求,也没有改变的需要,那么就不需要换到biblatex。但如果你需要生成精致复杂的参考文献,需要引用非传统标准的文献,需要进行非ASC码的排序,需要使用多语言,那么就需要使用biblatex。
\end{CJK}}

Some people \emph{need} to stick with traditional \bibtex. For
instance, if you are submitting work to a journal which has a \bibtex\
style which it requires you to use, then you should \emph{not} switch
to \biblatex, at least for that paper. Many journals are in just that
position, and very few (if any) have adopted \biblatex. And if you
frequently or sometimes need to use \bibtex, you should try to keep
your |.bib| files compatible with it (for instance, continue to use
|year|, |journal| and |address| rather than |date|, |journaltitle| and
|location| fields).

Some people \emph{don't need} to switch. If you have existing \bibtex\
styles which work for you, and do everything you need, then there is
no reason to change. You can. But you don't need to. The existing
programs have the advantage of stability in such cases. On the other
hand, switching will accustom you to \biblatex\ and put its more
powerful features at your disposal. It's really a matter of
taste. Those who don't have an existing heavy investment in \bibtex\
should probably prefer \biblatex. But don't make such a change a week
before you are due to submit your doctoral thesis.

Some people \emph{need} \biblatex. This is probably true if you work
in the humanities and need sophisticated and complex styles such as
traditional footnote-based citations systems, if you want to cite
non-standard sources, if you need to handle large volumes of non-ASCII
characters or have complex sorting requirements, if you need to work
in multiple languages.

\subsection{The required changes}

\indexstart{Biblatex!compared to bibtex}
The main differences between using \biblatex\ and `standard' \bibtex\
are as follows:\label{bibtex:simple:eg}

\marginpar{\begin{CJK}{UTF8}{gbsn}\footnotesize 改用biblatex需要注意一下几点:
1. 需要加载biblatex包
2. 需要在加载时制定样式
3. 需要用命令加载参考文献数据
4. 需要使用biblatex自己的文献表打印命令
5. 最重要的是要理解传统的bibtex的样式bst文件在biblatex中无法使用。要得到与标准bst样式一致的文献表,那么可以使用biblatex提供的标准样式替代。
\end{CJK}}

\begin{itemize}
\item You need to load \biblatex\ as a package using \cs{usepackage}
  (generally with various options). It's generally wise to load
  \package{csquotes} as well.
\item Instead of \cs{bibliographystyle}, you specify the style to be
  used as an option when loading \biblatex:
\begin{pseudoverb}\centering
  \cs{usepackage}[style=\ldots]\{biblatex\}
\end{pseudoverb}
\item Instead of specifying the \verb|.bib| file as an argument to the
  \cs{bibliography} command, you use the command
  \cs{addbibresource\{\}} to identify the file(s). You specify the
  file(s) to be used completely, including their \verb|.bib|
  suffixes. So if your bibliography file is called \verb|mybib.bib|,
  you have
\begin{pseudoverb}
  \centering \cs[mybib\textbf{.bib}]{addbibresource}
\end{pseudoverb}
\item Instead of \cs{bibliography}, you use \cs{printbibliography} at
  the point where the bibliography should be printed.
\end{itemize}

One important point to understand: existing \bibtex\ \emph{styles}
cannot be directly used by \biblatex.\sidenote{There is a project to
  provide \biblatex\ styles that are identical to the standard
  \bibtex\ ones: the \package{biblatex-trad} package. Development
  seems to have been sporadic.}  The `standard' styles in \biblatex\
do not precisely correspond to the standard styles in \bibtex.
\indexstop{Biblatex!compared to bibtex}

\subsection{An example}

Find an existing \verb|.bib| file of your own (or, if you are pushed
to find one, make use of a standard file such as
\verb|biblatex-examples.bib|, which will be installed with \biblatex).

\marginpar{\begin{CJK}{UTF8}{gbsn}\footnotesize 给出一个示例
\end{CJK}}


Try the following sample document: \clearpage
\marginnote{\tikz[overlay]{\draw (0,-0.5em) edge [out=180, in=25,
    -stealth] ([shift={(0.5ex,2.3ex)}] pic cs:csquote)}Although not
  essential, \package{csquotes} should normally be loaded.}
\marginnote[2ex]{\tikz[overlay]{\draw (0,-0.5em) edge [out=180, in=90,
    -stealth] ([shift={(-0.5ex,2.3ex)}] pic cs:style)}Style is
  specified when loading the package, not using
  \cs{bibliographystyle}.}  \marginnote[0.2ex]{\tikz[overlay]{\draw
    (0,-0.5 em) edge [out=180, in=0, -stealth] ([shift={(1.2ex,
      +1ex)}] pic cs:resource)}Instead of identifying the
  \texttt{.bib} files in the \cs{bibliography} command, they are
  identified this way.}  \marginnote[0.2ex]{\tikz[overlay]{\draw
    (0,-0.5 em) edge [out=195, in=45, -stealth] ([shift={(1.2ex,
      +1ex)}] pic cs:printbib)}Instead of \cs{bibliography},
  \cs{printbibliography} is used to print the reference list.}
\begin{pseudoverb}
\cs[article]{documentclass}\\
\colorbox{green!30}{\cs[csquotes]{usepackage}\tikzmark{csquote}}\\
\cs{usepackage}[backend=bibtex, \colorbox{red!30}{style=\tikzmark{style}numeric}]\{biblatex\}\\
\colorbox{blue!25}{\cs[biblatex-examples.bib]{addbibresource}\tikzmark{resource}}\\
\cs[document]{begin}

\cs[*]{nocite}

\colorbox{green!30}{\cs{printbibliography}\tikzmark{printbib}}

\cs[document]{end}
\end{pseudoverb}

If you run \LaTeX, \bibtex, and \LaTeX\ again, you should find that a
numbered bibliography is produced. Now see if you can get it to work
with \package{Biber}. Replace line 2 with
\begin{center}
\cs{usepackage[backend=biber, style=numeric]\{biblatex\}}
\end{center}
and delete \verb|.aux|, \verb|.bbl| and \verb|.blg| files. Now run
\LaTeX, \package{Biber} and \LaTeX\ again. You should generate the
same document. (You might also like to try the sample document
suggested for complete neophytes, which is at page
\pageref{neophyte:example}.) Although \biblatex\ can use \bibtex\ to
prepare its data, it's generally better to use \package{Biber}, which
is a more modern program with many advantages.

\subsection{The \package{Natbib} package}

\index{Natbib}

\marginpar{\begin{CJK}{UTF8}{gbsn}\footnotesize \package{Natbib}宏包的问题。biblatex提供了\package{Natbib}宏包的替代方案,它提供一个\package{Natbib}模块,使用户可以无缝衔接使用类似Natbib提供的命令。但实际上最好的思路是习惯使用biblatex自身提供的命令,因为它基于样式可以灵活地定制。
\end{CJK}}
Many \LaTeX\ users use the \package{Natbib} package. This is a sort of
half-way house between \bibtex\ and \biblatex. It's somewhat more
flexible than a pure \bibtex\ solution, and has (in particular) a
wider range of citation commands to deal with author\slash year
citation systems. But it still uses \bibtex\ under the hood, and it
doesn't have \biblatex's flexibility.

Biblatex has a \package{Natbib} `compatibility mode'.
If you load \biblatex\ with the option \texttt{natbib} (or
\texttt{natbib\allowbreak =\allowbreak true}), then it will let you
use some \package{Natbib}-like citation commands, like \cs{citet} and
\cs{citep}. However, the compatibility is really only skin-deep;
it hardly\footnote{It does modify the punctuation used to separate an
author's name from the year to match the \package{Natbib} default.}
extends beyond the citation commands, and the actual formatting of the
citations (which will depend on the style you specify) will be
determined by \biblatex. In general, it's probably a better idea to
get used to the \biblatex\ citation commands.
%%% Local Variables:
%%% coding: utf-8
%%% mode: LaTeX
%%% TeX-master: "biblatex-tutorial"
%%% End:


%\chapter{The Database File}
\chapter{数据库文件}
\label{ch:database}

This chapter is intended as a basic introduction to database files.
\marginpar{\begin{CJK}{UTF8}{gbsn}\footnotesize 本章介绍数据库文件。仅讨论常用的文献条目类型。给出一些bib文件示例及其产生的效果。要记住文献输出的格式取决于所选择的参考文献样式。而不同的样式会有不同的差异。后续章节主要介绍这种输出的格式,本章主要介绍文献信息的输入。
\end{CJK}}
It deals only with commonly used types of source, leaving the more
obscure corners to the manual. It starts from the ground up, for those
who have no \bibtex\ or \biblatex\ experience. Those who are familiar
with \bibtex\ already can probably skim the first part, but should
read the later parts fairly carefully because \biblatex\, with
\package{Biber} is slightly different and considerably more powerful
than \bibtex.

The chapter gives examples of some |.bib| files and output they will
produce. Do bear in mind that the precise form of the output is
\emph{dependent on the style chosen}: the examples are intended to
help illustrate the general principles, but different styles may
select from and format the data in different ways. The rest of the
book is all about how to control such output. This chapter is about
the input, and the examples are only there to give colour.

%\section{From ground up}
\section{从零开始}
\index{.bib file@\texttt{.bib} file|see{database}}%
\index{database!general description}
The \package{Biber} program, which prepares data for use by \biblatex,
can read a number of different formats; but only one is (at the
moment) really fully supported, and that is the format you should
use.
\marginpar{\begin{CJK}{UTF8}{gbsn}\footnotesize biber支持多种输入,但只有一种格式是完全支持的。就是为 \bibtex 开发的bibtex格式。它简单,易写,可以很容易用专门软件生成。bibtex格式文件通常有后缀名.bib,它是可以包含任意unicode字符的文本文件。
\end{CJK}}
It is a format originally developed for \bibtex. It is simple. It
can be written by hand, or produced with the assistance of specialised
software.


The \bibtex\ format database file conventionally has the suffix
\verb|.bib|. It is a `plain text' file which (with \package{Biber})
can use any unicode characters.

It consists of a number of `entries', each of which relates to a
particular bibliographical source.
\marginpar{\begin{CJK}{UTF8}{gbsn}\footnotesize 它由一定数量的“条目”构成,每一条目对应一篇文献源。每个条目都有自己的条目类型,比如 book(专著), article (期刊析出的文章)等,每个条目都有一定的域构成,这些域保存有参考文献的各项详细信息。一个条目的结构如图\ref{basic:source:eg}所示。
\end{CJK}}
Each entry consists of an
\emph{entrytype specifier} which explains what sort of source it is:
like book, or article; a \emph{key} which uniquely identifies the
source and which you will use for citations; and a set of
\emph{fields} which contain bibliographical data about the work. The
basic structure of such a record is shown in figure
\ref{basic:source:eg}.

\begin{figure*}
\strut\vspace{2ex}

\begin{minipage}[t]{1.5in}
\sffamily
\noindent
\tikz{\node(typekey){type};}

\vspace{10ex}

\noindent
\tikz{\node(fieldkey){fields};}

\end{minipage}
\begin{minipage}[t]{3in}
\ttfamily
\tikz{
  \node(entrytype)
       [text height=10pt, text depth=2pt, fill=red!30]{@book};
   \node(brace)[text height=10pt, text depth=2pt, xshift=5ex]{\{};
   \node(entrykey)
        [text height=10pt, text depth=2pt, fill=green!30,xshift=13ex]{nussbaum95};
   \node(keycomma)
        [circle, draw, xshift=23ex, yshift=-5pt, line width=1pt]{,};}
\quad\tikz{\node(authornode)[text height=10pt, text depth=2pt, fill=blue!25]{author = \{Nussbaum, Martha C.\}} ;
  \node(authorcomma)[circle, draw, xshift=22ex, yshift=-5pt, line width=1pt]{,} ;}\\
\quad\tikz{\node(titlenode)[text height=10pt, text depth=2pt, fill=blue!25]{title = \{Poetic Justice\}} ;
  \node(titlecomma)[circle, draw, xshift=18ex, yshift=-5pt, line width=1pt]{,} ;}\\
\quad\tikz{\node(publishernode)[text height=10pt, text depth=2pt, fill=blue!25]{publisher = \{Beacon Press\}} ;
  \node(publishercomma)[circle, draw, xshift=20ex, yshift=-5pt, line width=1pt]{,} ;}\\
\quad\tikz{\node(locationnode)[text height=10pt, text depth=2pt, fill=blue!25]{location = \{Boston\}} ;
  \node(locationcomma)[circle, draw, xshift=15ex, yshift=-5pt, line width=1pt]{,} ;}\\
\quad\tikz{\node(datenode)[text height=10pt, text depth=2pt, fill=blue!25]{date = \{1995\}} ;
  \node(datecomma)[circle, draw, xshift=11.5ex, yshift=-5pt, line width=1pt]{,} ;}\\
\tikz{\node(finalbrace)[text height=10pt, text depth=2pt]{\}};}
\end{minipage}
\begin{minipage}[t]{1in}
\raggedright
\sffamily
\noindent
\tikz{\node(keykey){key};}

\vspace{10ex}
\noindent
fields separated by commas
\end{minipage}
\begin{tikzpicture}[overlay]
\path[-stealth] (typekey.south) edge [out=-90, in=180] (entrytype.west) ;
\path[-stealth] (fieldkey.east) edge [out=0, in=180] (authornode.west) ;
\path[-stealth] (fieldkey.east) edge [out=0, in=180] (titlenode.west) ;
\path[-stealth] (fieldkey.east) edge [out=0, in=180] (publishernode.west) ;
\path[-stealth] (fieldkey.east) edge [out=0, in=180] (locationnode.west);
\path[-stealth] (fieldkey.east) edge [out=0, in=180] (datenode.west) ;
\path[-stealth] (keykey.west) edge [out=180, in=45] (entrykey.north east) ;
\end{tikzpicture}
\caption{A basic source record\label{basic:source:eg}}
\end{figure*}

\index{database!entry type}
The \emph{entrytype} specifier (\verb|@book|) in figure
\ref{basic:source:eg}) says what type of source this is.
\marginpar{\begin{CJK}{UTF8}{gbsn}\footnotesize 图中的标识 (\verb|@book|) 就是文献的类型。biblatex及其样式可能支持很多类型,但本章只关注4种类型:
book,article,inbook,Incollection。
\end{CJK}}
A large number of different types are supported, which we will look at in due
course; indeed, in the end, the question is really what types are
supported by the particular style you are using, since styles can
define any type they like. But in this chapter we are going to look at
just four types:
\begin{description}
\item[Book] is used for an entry that consists of a complete physical
  book.
\item[Article] is used for journal articles.
\item[Inbook] is used for a self-contained chapter in a
  book.\intref{See p~\pageref{inbook:vs:incollection} for the
    difference between \texttt{inbook} and \texttt{incollection}.}
\item[Incollection] is used for a self-contained paper in a collection
  of papers.
\end{description}

The entry type specifier is not case sensitive: \verb|@book|,
\verb|@Book| and \verb|@BOOK| all mean the same thing.
\marginpar{\begin{CJK}{UTF8}{gbsn}\footnotesize 条目类型之后第一个内容是参考文献的引用关键字(\verb|key|)。其内的字符要避免出现\{ \}等特殊符号。可以采用一些容易记忆,使用的关键字,比如\texttt{nussbaum:1995}这样的。
\end{CJK}}
\index{database!general description}
A source record, if you squint at it, takes the basic form
\begin{center}\verb|@entrytype{...}|\end{center}The very first thing
within the braces is the \verb|key|. This does not have to take any
particular form, but it does need to be \emph{unique} to the
particular source, and it should be something memorable enough and
short enough to be practical for you, since you will be using it in
documents you write.

Although a \verb|key| can include quite a range of characters, there
are some that you should avoid;\footnote{{\ttfamily
      \textquotedbl\ @ \textquotesingle\
      \textbackslash\ \# \{ \}
      \textasciitilde\ \%\ \textunderscore\
      \&\ \$\ , \textasciicircum}} nor may a
key include spaces. It's a good idea to be more-or-less consistent
in the form your keys take, and to use common sense. For instance,
\texttt{nussbaum:1995} or \texttt{nussbaum:poetic} might be quite
good keys, but \texttt{nb} could easily clash with others keys,
\texttt{nussbaum:1995:poetic-justice} is arguably too long, and
\texttt{nbpoj95} may be hard to remember or work with. Use your
common sense.

Immediately after the \verb|key|, place a comma.
The rest of the source record consists of a set of fields and values entered
(usually)\footnote{Occasionally neither braces nor quote marks are
    needed.} in either the form
\begin{center}\texttt{field = \{value\},}\end{center}
or, if you prefer, in the form
\begin{center}\texttt{field = "value",}\end{center}
Fields must be separated by commas. It is permissible (though not
required) to have a comma after the last one. I always make sure a
comma is added to every field: it makes editing easier, because if you
add another field you won't forget the comma.
\marginpar{\begin{CJK}{UTF8}{gbsn}\footnotesize 引用关键字后面是一个逗号,然后是以项=值形式表示的域信息。各个域之间必须要用逗号分隔。域顺序无关紧要,也可以在=号周边等位置增添一些空格使内容对齐,是信息看起来更美观。
\end{CJK}}

The order in which you place the fields (apart from the |key|, which
must come first) doesn't matter. You can use whitespace freely between
fields and labels to keep things neat: as far as \package{Biber} is
concerned
\begin{Verbatim}
   author = {Allen, Sidney},
   title  = {Vox Latina},
\end{Verbatim}
is just the same as
\begin{Verbatim}
author={Allen, Sidney},title={Vox Latina}
\end{Verbatim}
but for human consumption sensible spacing helps to keep things tidy.

\index{unicode!database, in}
Within a field you can use any unicode characters, and you can include
\LaTeX\ code too, if you need to (though it's sometimes a bad idea,
especially in name fields, for reasons we will come to).
\marginpar{\begin{CJK}{UTF8}{gbsn}\footnotesize 域内可以包含任意unicode字符,也可以包含\LaTeX\ 代码,但这不是一个很好的方式,特别是在像姓名(name)类的域中。域信息的输入应尽可能的全面,比如姓名尽可能的写全,如果采用缩写,当采用要使用全名的样式时,就会缺失信息。
\end{CJK}}
Similarly,
just as when you are writing \LaTeX, you can break lines and end up
with just a single space, which also helps you to arrange things
neatly.

As figure \ref{basic:source:eg} shows one
can see that they are mostly self-explanatory. It's important to
appreciate that they are intended as data sources: the raw material
from which \biblatex\ will construct citations, not the citations
themselves. So generally you should enter as much information as you
have, because while \biblatex\ can ignore (or alter) information it
has, it cannot invent information it doesn't have. For example, if
entering a name you should, if you can, give the full name. Perhaps
your currently-preferred citation system uses initials only. That
doesn't matter, because \biblatex\ can extract the initials if it
knows the full name; but it can never guess the full name from
initials, and if you ever decided to use a system which needed the
full name, you would be lost. Similarly, you shouldn't attempt any
sort of formatting in these fields: they are just data, and if you
want names in small capitals or titles in italics you deal with that
by giving instructions when the |.bib| file is \emph{used}, not within
it.

%\section{Field types}
\section{域类型}

\index{database!fields}
When you read the \biblatex\ manual,\intref{\emph{Manual} \S~2.2.1}
\marginpar{\begin{CJK}{UTF8}{gbsn}\footnotesize 域主要有5种类型:
姓名域,列表域,文本域,日期域和其它域。
\end{CJK}}

you will see that it distinguishes between five main types of field:
name fields, list fields, verbatim fields, date fields and other
fields, with various others for good measure. This is not actually a
very helpful distinction from the user's point of view, and I'd
suggest that you think of fields in this way:
\begin{itemize}
\item \emph{Names} These \emph{are} special, and worth thinking about
  separately. Typical name fields are \texttt{author} and
  \texttt{editor} fields.
\item \emph{Dates} Again, the \emph{are} special, and worth
  considering specially. Typical name fields are \texttt{date} and
  \texttt{urldate}.
\item \emph{Other bibliographical fields}, like \texttt{title},
  \texttt{journaltitle}, \texttt{url} or \texttt{isbn}. These are all
  fields that will contain material which might (depending on your
  style) find its way into your printed bibliography.
\item \emph{Meta-data} fields, like \texttt{keywords} and
  \texttt{options}, in which you provide information about your source
  which may be used to help format it, but is not expected to find its
  way directly into print.
\end{itemize}

\subsection{Names}

\indexstart{database!names}\index{names!database, in|see{database}}
Names turn out to be rather complex things.
\marginpar{\begin{CJK}{UTF8}{gbsn}\footnotesize 姓名域。
\end{CJK}}
Suppose I have the
advantage of being called Quentin William Ffortescue von
Rumplestiltskin, Jr. I probably care about all the parts of this name;
but bibliographical software is particularly interested in the last
name which will be used for sorting (Rumplestiltskin) and the first
names which will, if necessary, be used for sorting (Quentin William
Ffortescue), and from which the initials (Q.\,W.\,F.) will be
constructed; but it may also need to know how to print my surname
including its `von' part, and to include the `Jr.' after my name.

All this can be done. It can even (usually) be inferred. It
\emph{usually} doesn't matter whether a name is entered as
\verb|John Smith| or \verb|Smith, John|. If in doubt, the following
rules will keep you straight. (Not all of them is strictly necessary;
but they are quite simple.)

\begin{enumerate}
\item Always enter initials with full stops. If it sees
  \verb|A. Author|, or \verb|Author, A.|,
  \biblatex\footnote{\package{Biber}, actually.} knows that \verb|A|
  is an initial. If it sees \verb|A Smith| it might think that `A' is
  a (very short) name. Since this \emph{can} matter, usually when you
  least expect it, always include the dot. This doesn't mean that
  \biblatex\ will necessarily \emph{print} any dot when it outputs
  the bibliography. That is controlled by the bibliography style, so
  you can safely include the dots in your \verb|.bib| file and lose
  them later.
\item Always put any \emph{von} (or equivalent, such as \emph{van} or
  \emph{de la}) \emph{before} the last name, as shown above.
\item For names \emph{without Jr.\ or numbers} use \emph{either}
  `ordinary' name order \emph{or} \angled{last name}, \angled{first
    name or initials}. So \verb|von Author, A.| or
  \verb|A. von Author|. I highly recommend consistency in this. And in
  fact I strongly recommend that you use the Last, First format, which
  avoids mistakes with \ldots
\item If there is a `junior part' (like Jr.\ or III), you \emph{must}
  use the `backwards' form:\begin{center}von Rumplestiltskin, Jr.,
    Quentin William Ffortescue\end{center}
\item If what looks like two words should actually be treated as one,
  then enclose them in braces. For instance, if a last name is
  double-barrelled but not hyphenated, it needs to go in braces. The
  hyphenated \verb|Homburg-Williams| will be fine, but we don't want
  Mr.~J.\ Homburg Williams to find himself as J.\ H.~Williams, so make
  it \verb|{Homburg Williams}, J.|\index{name, double-barrelled}
\item Similarly some names are not made to be broken at all, notably
  institutions, which sometimes claim authorship or something close to
  it. If the University of Oxford `writes' a report, \biblatex\ will
  be inclined to think of it as `of Oxford, U'. Avoid this by
  enclosing the whole name in (extra) braces:\intref{See
    p~\pageref{sorting:sortname} for advice on how to make sure such
    names are properly sorted.}\index{institutions!as authors}\index{author!institutional}
\begin{center}
\verb|author = {{University of Oxford}}|
\end{center}
\item For accented characters, either use unicode or enclose the
  accented letters in braces: \verb|{Victor, Paul {\'E}mile}|. Of
  these techniques, the use of unicode accented characters---which
  \package{Biber} can handle, unlike \bibtex---is much to be
  preferred. If you do have to use \TeX\ accents, enclose the
  character in question (but only that character) in braces.\index{accented letters}
\end{enumerate}

Often academic works have multiple authors or editors. In that case,
enter all the names separated by `and' (\emph{not} commas).
\begin{center}
\ttfamily
Ardman, A. \textbf{and} Baptiste, B. \textbf{and} Carruthers, C.
\end{center}

You can also put \verb|and others| yourself. But be careful. How many
authors' names get printed is heavily style-dependent.\intref{See
  p~\pageref{recipes:maxnames} for more information on how many names
  will be printed.} Some styles only want one or two authors before
printing `et al'; others may want four or five, or always print every
name. It's easy for \biblatex\ to truncate names if it has more than
it needs, but impossible for it to guess who wrote a paper if it
hasn't been told, so as always, enter all the information you have
available.
\indexstop{database!names, in}

\subsection{Dates}

\index{database!dates, in}\index{date!database, in|see{database}}
Enter dates in the form |YYYY-MM-DD|.
\marginpar{\begin{CJK}{UTF8}{gbsn}\footnotesize 日期域。
\end{CJK}}
So the 28th day in
of February 2012 is |2012-02-28|. In general, most \biblatex\ styles
will work with less-than-complete dates, and of course in many cases
you won't know the full date (for instance, the date when a book was
published): so |2012| is a valid date, as is |2012-02| (February
2012), as is |2012-02-28|. (Of course, that's not how dates will be
\emph{printed}, unless you want them to be: it's just how they are
entered.)

Occasionally you need to enter a range of dates. In that case, use the
a forward slash to separate the range: February to March 2012 is
|2012-02/2012-03|, and so on.

\subsection{Data fields}

Data fields other than name fields do not normally present such
difficulties, but there are a few points to bear in mind.
\marginpar{\begin{CJK}{UTF8}{gbsn}\footnotesize 数据域。
\end{CJK}}

Some such fields (for instance the |publisher| and |location| fields)
are in fact \emph{lists}: in such cases, if you need more than one
entity mentioned, use \emph{and} between them, as with names.

Data fields may be important for sorting. For example, although in
most cases the primary sorting is done using the |author| or |editor|
field, the |title| field may sometimes come into play. This can pose
additional challenges, because it is possible, especially if the field
includes some \LaTeX\ command, to bamboozle the sorting algorithm. In
general, enclose \LaTeX\ commands (other than accented letters) in
\emph{two} sets of braces, and use unicode to deal with accented
letters wherever possible. So, for instance, we would
enter:\intref{See p~\pageref{sorting:sorttitle} for more information
  about how to control the way titles are sorted.} As with names,
accents and diacritics are probably best handled with unicode.
\index{accented letters}
\begin{center}
|title = {The {{\LaTeX}} Companion}|

or

|title = {{\`A} l'ombre des jeune filles en fleurs}|

or (better)

|title = {Du côté de chez Swann}|
\end{center}

\index{title, capitalization}\index{capitalization!database, in|see{database}}\index{database!capitalization}
Data fields may be modified. For instance, some style files capitalize
only the first word of a title, so that an entry of
\begin{center}
|The German Law of Obligations|
\end{center}
would become
\begin{center}
The german law of obligations
\end{center}
Where you know that a particular letter must never be changed, you should `protect' it with braces:
\begin{center}
\texttt{The \{German\}\footnote{You will sometimes see this done as \texttt{\{G\}erman}, but the method given in the text is better, because it ensures proper kerning.} Law of Obligations}
\end{center}
Otherwise it's best to enter titles, at least if you write in English,
with `significant words capitalised', since \biblatex\ can easily
remove all but the first capital, but cannot be relied on to carry out
more sophisticated changes.\footnote{In other words, if asked to
  capitalise the first letters of \texttt{The German law of
    obligations} it would likely end up with \emph{The German Law Of
    Obligations}, which is not correct.} As noted above, some styles
put titles into `sentence case'---with only the first word
capitalised. If you don't want that, the right thing to do is not to
attempt to outfox the style by manually protecting every capital with
braces, because that makes your database file useless if you ever do
want such changes. The right thing to do is to choose a different
style, or modify it, to preserve your capitalization.

\subsection{Metadata}
\marginpar{\begin{CJK}{UTF8}{gbsn}\footnotesize 元数据。
\end{CJK}}
There is a wide variety of metadata that you may, for various reasons,
include in a file, including options to guide \biblatex\ in formatting
citations, and keywords and options to help structure biographies. In
general there is no particular difficulty with this sort of
information. It is just plain text, still included in braces, and
(sometimes) a comma-separated list.

Of these various fields, three stand out for special mention at this
point.

\index{database!languages}\index{language!specification in database|see{database}}
\paragraph{Specifying hyphenation patterns}Different languages are
hyphenated in different ways.\intref{Chapter~\ref{ch:languages}}
If you are writing a document in English, but cite a book whose title
is in French, \LaTeX\ will have trouble hyphenating it correctly. You
can specify the language in the |langid| field. Then, \emph{provided
  you set the package option} |autolang=hyphen|, the entry will be
hyphenated using the appropriate language.

\paragraph{Keywords} The |keywords| field enables you to provide one
or more keywords (separated by commas). These can be useful if you
want to separate out different kinds of source. For instance, if you
wanted to produce different bibliographies by topic\intref{Chapter~\ref{ch:subdivisions}} for an annotated
bibliography, you could assign sources to particular topics using
keywords.

Other meta-data fields include |options|, |sorttitle|, |labeltitle|,
|indextitle| and |indexsorttitle|. These are all used for specialised
purposes to enable you to over-ride defaults in the way that
\biblatex\ would construct citations, sort the bibliography, or index
an entry. Some of these are discussed as appropriate later in this
book,\intref{Chapter~\ref{ch:sorting}} where their use is explained.

\paragraph{Sub-types} Some bibliography styles want to handle, say,
an article in a newspaper differently from an article in a journal. For this
purpose there is an \verb|entrysubtype| field. This can also be useful when
preparing subdivided bibliographies, where it can be used to select which
works get printed in which section.

%\section{Common entry types}
\section{常见条目类型}
\marginpar{\begin{CJK}{UTF8}{gbsn}\footnotesize 常见的条目类型。
\end{CJK}}
This section is going to take a look at four of the commonly used
types of entry: books, articles, and the |incollection| and
|inbook| types. This is not intended to be a complete formal
description of every possibility. For that you can consult the
documentation.\intref{\emph{Manual}~\S~2.1.} It is supposed to give
practical guidance, especially for neophytes. Most of what is said
should seem like common sense for anyone who has ever written an
academic bibliography.

\subsection{Books}
\marginpar{\begin{CJK}{UTF8}{gbsn}\footnotesize 专著(book)。
\end{CJK}}

\index{database!book type}\index{book!database entry|see{database}}
The bare minimum for any book is some sort of title. \emph{The Bible}
is a book, timeless as it is and un-named as its Author customarily
remains. Most books, however, have more information, and most
bibliographical styles require more information to be provided, if it
is available. Typically aim to record the following:

\paragraph{The author(s) name(s)} in the |author| field. If there is
an |editor| but no |author| leave the field blank.

\paragraph{The editor(s) name(s)} in the |editor| field. If there is
no editor, leave the field blank. If there is both an author and an
editor, include the editor's name as well as the author's.

\paragraph{The title} in the |title| field. Mostly this is easy, but
some works exist to annoy us. Dr Edwin Poppie-Cocke has just completed
`Grammatical Solecisms: Dangling Participles and Misplaced Modifiers`,
which is volume 234 of his treatise on `Elementary Stylistics'. What,
exactly, is its title? For our purposes,\footnote{This is a
  simplification: there is also a \texttt{titleaddon} field that can
  be used to print something after the title in a different font. And
  \texttt{maintitle} can also have its own \texttt{mainsubtitle} and
  \texttt{maintitleaddon} too! Happily, these are rarely needed.} we
divide it into three:
\begin{itemize}
\item The |maintitle|, which is the title of the multi-volume
  treatise.
\item The |title|, which is the particular title of the book.
\item The |subtitle|.
\end{itemize}
We therefore enter:
\begin{Verbatim}
@book{poppycocke:2013,
  author    = {Poppie-Cocke, Edwin},
  maintitle = {Elementary Stylistics},
  title     = {Grammatical Solecisims},
  subtitle  = {Dangling Participles and Misplaced Modifiers},
  volume    = 234,
}
\end{Verbatim}
It is better to divide title and sub-title in this way because,
although it would work if we didn't, some citation styles use the
title when referring back to the work on second and subsequent
citations, and the title alone, without subtitle, is usually
sufficient for this purpose, and less of a mouthful.

Where you have a multiple volume work, you have two choices. (1) You
can have a single entry in your database for the work in question with
all its volumes. This would normally be appropriate where you are
dealing with one work, published at one time, which happened to be
subdivided. In that case, enter the number of volumes in the |volumes|
field -- and remember to identify the particular volume you are
referring to when you cite the work.\footnote{There is also a
  \texttt{mvbook} entry-type, which is specifically designed for
  multiple-volume works. It's probably better, though not compulsory,
  to use it for such works: see the \biblatex\ \emph{Manual}
  \S~2.1.1.} (2) You may have separate entries for each volume, where
each volume is effectively a separate work. That is normally
appropriate where the volumes are published at different times. In
that case, put the volume in the |volume| field. Some works have
`parts' instead of, or as well as, volumes: in that case put the part
in the |part| field.

Some self-contained works are published as part of a series. If you
want to record that information, put the series in the |series| field.

\paragraph{The edition} in the |edition| field. This is usually
omitted for the first edition if only one edition was published. If
this is a number, just enter the cardinal number: |{2}|, not |second|
or |2nd| --- the right suffix will be added by \biblatex. If it's
something like `revised' or `corrected', type that.

\paragraph{Publication information}

In the |date| field. The year will usually suffice. The older \bibtex\
style was to enter the date in the |year| field. That will work fine,
but it's probably better to use |date|,\footnote{Unless you need your
  \texttt{.bib} file to be usable in \bibtex\, in which case stick
  with year.} which is `correct'. In some cases you will want to
record two dates: the date of original publication, and the date of
the particular edition you are referring to. In that case, put the
date of the particular edition in the |date| field, and the date of
original publication in the |origdate| field.

The place of publication should be specified in the |location|
field. You can also use a field called |address|, which is provided
for backward compatibility purposes (it was the field used in the
`original' \bibtex.)

The publisher's name goes in the |publisher| field. This is a `list
field', so that you can have more than one publisher, in which case
you enter them as
\begin{center}
\ttfamily
publisher = \{Publisher A \textbf{and} Publisher B \textbf{and} Publisher C\},
\end{center}

In some disciplines it is not customary to print the publisher's name,
or the place of publication, and you may think that's pretty
sensible since in most cases they are of no interest. Even so, unless
you are very certain that you will never want this information, it's
better to fill it in. It's always fairly straightforward to tell
\biblatex\ not to use information that it has available --- but
impossible to tell it to create information you haven't provided. So,
if you think of the future, it's best to include both.

\newthought{Those fields cover the basic bibliographical data
  applicable to most books} but there are other pieces of information
that you may sometimes need or want to record.

\paragraph{Short titles and shorthands} If a title is being printed in
a bibliography, it makes sense to print it in its entirety, even if it
is very long. But some citation schemes (such as those that refer to
works by author and title) make repeated use of titles, and if the
title is very long such repeated references can become tiresome. It
may make sense, in such cases, to abbreviate the title of the work on
second and subsequent citation, so that---say---P.\ Kempees, \emph{A
  Systematic Guide to the Case-Law of the European Court of Human
  Rights 1960--1994} can become `Kempees, \emph{Systematic Guide}'. In
such cases, you can specify a |shorttitle|.

\indexstart{shorthand}\label{shorthands}
The \emph{shorthand} is a related, but subtly different idea. In some
works (for instance, commentaries or monographs devoted to the study
of a small number of books) or there may be such frequent
reference to a particular source that it makes sense to have an
abbreviation for it. You can define a special citation form, called a
`shorthand', which will be used to cite the work in question. The
package will also keep track of shorthands and you can print a list of
them, similarly to a bibliography, using the command
|\printshorthands| where you want the list to be produced.\intref{See
  the \emph{Manual} \S~3.6.3.} A shorthand is included in the
|shorthand| field.

Shorthands may be summarized in a list of shorthands. They need not,
therefore, be self-explanatory; the reader will encounter them
frequently, and (if they are not standard in the field) can be
expected to look them up. They should be fairly rare --- reserved for
special cases. Short titles on the other hand will not appear in any
separate list, and you should be careful to make sure that they will
be immediately identifiable to any reader from the bibliography, and
reserve them for cases where the title is long. In a specialist work
on Wittgenstein it may be sensible to define \emph{PI} as an
abbreviation for \emph{Philosophical Investigations}, but it wouldn't
make sense to define that as a short title, or indeed to define any
short title for general use. And in the example given above,
`Systematic Guide' makes a good short title, but `SGCL60--94' would be
quite confusing. It is usually better not to define any short title or
shorthand until you are sure that you need them.
\indexstop{shorthand}

\paragraph{Translators, adapters, revisers.} It is usual to have an
author, and common to have an editor as well. It's not uncommon to
have other people who should be credited: translators, commentators
and the like. There are dedicated fields for |translator|, |annotator|
or |commentator|, into which you can put the name(s). If you have some
oddity, you can make use of upt to two fields called |editora| and
|editorb|. These are `generic' fields, into which you can put
names. If you do that, you also need to fill in the |editoratype| (or
|editorbtype| ...) fields, with the `role' of the person. \biblatex\
recognises as possible roles: `editor', `compiler', `founder',
`reviser' and `collaborator', and the enigmatic figures of `redactor'
and `continuator' too. You can add others as well, though it's not a
totally straightforward task.\intref{See \emph{Manual} \S\S~2.3.6, 3.8
  and 4.9.1.}

The result, as the following example (whose output is shown in figure
\ref{redactors}) shows, can in theory include a very large number of
different roles, where that is needed.
\begin{Verbatim}
@book{team,
  author      = {Author, Arnold},
  editor      = {Editor, Edwin},
  translator  = {Translator, Theodore},
  commentator = {Commentator, Cuthbert},
  editora     = {Redaktor, Richard},
  editoratype = {redactor},
  editorb     = {Collaborator, Christopher},
  editorbtype = {collaborator},
  title       = {Team Players},
  date        = {2013},
  publisher   = {Pubco},
  address     = {Oxbridge},
}
\end{Verbatim}

\begin{figure}
\fbox{\includegraphics{./examples/database-eg1.pdf}}
\caption{Bibliographical entry showing author, editor, commentator, translator, redactor and collaborator\label{redactors}}
\end{figure}

\index{URL!database, in|see{database}}\index{database!URL}\index{database!electronic publication}
\paragraph{Electronic publication} Works which were, in the past,
exclusively printed are now often published electronically, either as
well as or instead of paper publication. The \biblatex\ package tries
to reflect that.

One possibility is that the work is available on the internet, at a
\URL. In that case, you can specify a \URL\ in the |url| field. Many
bibliographic schemes require that the \emph{date} on which that \URL\
was last checked is also given, and that date should therefore be
included in the |urldate| field.

\index{arXiv}\index{JSTOR@\textsc{jstor}}
A second possibility is that the work is available in a specialised
electronic repository, such as arXiv or \textsc{jstor}. If that is the case,
you can make use of the |eprint|, |eprinttype| and |eprintclass|
fields to provide a reference for the work. \intref{See
  \emph{Manual}~\S~3.11.7.} Not every bibliographic style will use
these details (and they can always be `turned off').\footnote{by
  setting the option \texttt{eprint = false} when loading \biblatex.}

Finally, you can if you wish, provide a Digital Object Identifier
\smallcaps{doi}, which may provide a more permanent record of an
electronic source than a \URL, since \smallcaps{doi}s do not
change.\sidenote{\url{http:\\www.doi.org}. Again this feature can be
  turned off by setting \texttt{doi = false}.}

\index{ISBN!database, in|see{database}}\index{database!ISBN}
\paragraph{ISBNs} If you think it is ever likely to be used, you can
included a book's \smallcaps{isbn} (International Standard Book
Number) in the |isbn| field. As with \smallcaps{doi}s, not every style
will print \smallcaps{isbn}s (and they can always be turned
off).\footnote{by setting the option \texttt{isbn = false} when loading
  \biblatex.}

\paragraph{Other information} The \biblatex\ package allows quite a
wide range of other information to be included in a |.bib| file,
though not all of it will be used in all styles. For instance, it is
possible to include information about the original language of a book,
its original title, the title of a reprint and so forth. Since most of
this information is of interest only in narrow fields of unusual
cases, the reader is referred to the manual for details. Three fairly
common fields are, however, worth noting:
\begin{itemize}
\item |pubstate| is used to provide information about the publication
  state of a work that has not yet been `properly' published. Standard
  styles should recognise (in decreasing ratio of hope to expectation)
  |inpreparation|, |submitted|, |forthcoming|, |inpress| and
  |prepublished| as valid options here, and print appropriate
  indications.
\item |note| and |addendum| may be used to provide necessary
  bibliographical information, in a free form, which would otherwise
  not have a `home', such as `reprinted with corrections from the 1724
  edition'. The difference is that |note| will usually get printed
  somewhere in the middle of an entry, while |addendum| gets printed
  at the end (though the precise position depends on the particular
  style). So
  \begin{Verbatim}
@book{generic,
  author   = {Author, Albert},
  ...
  note     = {With a note},
  addendum = {And an addendum},
}
\end{Verbatim}
produces something like figure \ref{addendum}.

\begin{figure}\fbox{\includegraphics{./examples/database-eg2.pdf}}
\caption{Note and addendum\label{addendum}}
\end{figure}
\item |annotation| may be used to provide a lengthy annotation, for
  instance for use in annotated bibliographies.
\end{itemize}
\indexstop{database!book type}

\subsection{Articles}

\marginpar{\begin{CJK}{UTF8}{gbsn}\footnotesize 期刊析出的文章(Article)。
\end{CJK}}

\indexstart{database!article type}\index{article!database entry|see{database}}
The article is the next basic form. Indeed, for most academic work it
is probably the most common. It is entered into the database using the
|article| entry type.

In general an article (as opposed to its reference) has only two
significant parts: the author(s) and title. Those are formatted just
as for books.\footnote{Except that \texttt{maintitle} is not used. The
  \texttt{subtitle} field could, however, be used.}
\begin{Verbatim}
@article{mueck,
  author  = {Mueck, A. O. and
             Seeger, H. and
             Wallwiener, D.},
  title   = {Comparison of the Proliferative Effects
             of Estradiol and Conjugated Equine
             Estrogens on Human Breast Cancer Cells
             and Impact of Continuous Combined
             Progestogen Addition},
}
\end{Verbatim}

That, as figure \ref{mueck} shows, gets us only part of the way: we
have the \emph{article}'s title and authors, but we still need to
provide the details of the journal in which it is published.

\begin{figure}
\fbox{\includegraphics{./examples/database-eg3.pdf}}
\caption{Article with author and title, but reference missing.\label{mueck}}
\end{figure}

The publication details are entered using fields for:
\begin{itemize}
\item |journaltitle| for the name of the journal.\footnote{For
    backwards compatibility reasons, \texttt{journal} will work too,
    but prefer \texttt{journaltitle} unless you need \bibtex\
    compatibility.} (There is also a |journalsubtitle| field, though
  it is very seldom required.)
\item |series| for the journal's series (if any).
\item |volume| for the journal volume (if any).
\item |number| for the journal number (if any).
\item |issue| for the journal's issue (if any) (such as `Spring' --
  there are also |issuetitle| and |issuesubtitle| fields, which can be
  used where a particular issue has a special title which ought to be
  cited). Use |number| strictly for numerical subdivisions and |issue|
  for anything else. No matter that the journal you are dealing with
  call this its `Christmas number': you will call it the |issue = {Christmas}|.
  And no matter the journal says this is volume 5, \emph{issue}
  2: you will call it |number = 2|.
\item |pages| for the pages occupied by the article in question.
\item |date| for the date of the publication.
\item |year| available only for backward compatibility with
  \bibtex. Unless you need that, use the |date| field.
\item |month| for the journal's month, if that matters for
  citation. This should be either an integer (1 = January, and so
  forth) \emph{or} a three letter code which should be entered
  \emph{without} braces: |jan| not |{jan}|. It is only there for
  backward compatibility with \bibtex. Unless you need that, you can just include the month as part of the |date|.

\end{itemize}

So we can complete our partial citation:
\begin{Verbatim}
@article{mueck,
  ...
  journaltitle = {Climacteric},
  volume       = {6},
  pages        = {221--227},
  date         = {2003},
}
\end{Verbatim}

And producing the result along the lines shown in figure \ref{mueck2}.

\begin{figure}
  \fbox{\includegraphics{./examples/database-eg4.pdf}}
  \caption{Article\label{mueck2}}
\end{figure}

\paragraph{Other information} Information such as \smallcaps{doi},
\smallcaps{issn} (the equivalent of \smallcaps{isbn} for journals) and
\smallcaps{annotations} is the same as for articles.
\indexstop{database!article type}

\subsection{Collections and Parts of Books}

\marginpar{\begin{CJK}{UTF8}{gbsn}\footnotesize 文集(Collection)和专著的析出部分(Inbook)
\end{CJK}}

\indexstart{database!incollection type}\index{database!inbook entry|see{database}}
\index{collection!article in|see{database, incollection}}
Papers are often published in books --- collected papers,
\emph{festschriften}, and collections devoted to a particular topic.

For such works, \biblatex\ provides two entry types: |inbook| and
|incollection|.\label{inbook:vs:incollection} The boundaries between
them are slightly hazy. In theory a |book| is a work which has, as a
work, one or more `authors' who take collective responsibility for it,
and |inbook| is the entry-type corresponding to a discrete and
self-contained part of that, whereas a |collection| is an assembly of
disparate contributions which will have an editor or editors, but no
author(s) as such, and |incollection| corresponds to a discrete part
of that. In practice --- as it lies on the shelf --- a |collection| is
a book. So the key difference is whether the `unifying feature' of the
work is its author (|inbook|) or its editors (|incollection|). But
almost everything said about the |book| type above can be assumed to
apply to the |collection| type, and everything said here about
|inbook| can be assumed to apply to |incollection| as well.

In practical terms, for both |inbook| and |incollection| types you
have two choices:
\begin{itemize}
\item You can include all the information in a single entry. In that
  case |author| and |title| are assumed to refer to the author of the
  discrete unit that is being cited, while |bookauthor|, |editor|, and
  |booktitle| refer to the larger work. You add either |pages| or
  |chapter| to indicate the part of the book occupied by the discrete
  unit.
\item You can separately specify the information for the larger work
  (as |book| or |collection|, using all the fields given above). You
  then specify the |author| and |title| of the sub-unit, together with
  |pages| or |chapter|, and use the |crossref| field to link the
  individual entry to the larger work.\index{crossref}\index{database!crossref}
\end{itemize}

So, for instance, using the first method we might have:
\begin{Verbatim}
@inbook{sedley:skulls,
  title        = {Skulls and Crossbones},
  author       = {Sedley, Stephen},
  bookauthor   = {Sedley, Stephen},
  booktitle    = {Ashes and Sparks},
  booksubtitle = {Essays on Law and Justice},
  date         = {2011},
  publisher    = {Cambridge UP},
  location     = {Cambridge},
  pages        = {131--138},
}
\end{Verbatim}

Alternatively, you could set things up as follows:
\begin{Verbatim}
@book{sedley:ashes,
  title        = {Ashes and Sparks},
  subtitle     = {Essays on Law and Justice},
  author       = {Sedley, Stephen},
  date         = {2011},
  publisher    = {Cambridge UP},
  location     = {Cambridge},
}
@inbook{sedley:skulls,
  title        = {Skulls and Crossbones},
  author       = {Sedley, Stephen},
  pages        = {131--138},
  crossref     = {sedley:ashes},
}
\end{Verbatim}

A citation of |\cite{sedley:skulls}| will produce exactly the same
output in either case. My fairly strong advice is to prefer the second
method, because it is more flexible -- it enables citation of the
whole book separately, and it enables you easily to add additional
chapters or papers as units of their own without difficulty. It also
means that, once a certain threshold is reached, the entire book will
be added to the bibliography, which is often a convenience to
readers. (By default, the `parent' work is added to the bibliography
if two `children' are cited. You can change this number by giving a
value to |mincrossrefs| in the options to \biblatex. So to set it to
include the parent only when four `children' are cited, you would
include |mincrossrefs = 4| when loading \biblatex.)
\indexstop{database!incollection type}

%\section{A summary by type of literature}
\section{参考文献条目类型总结}
\marginpar{\begin{CJK}{UTF8}{gbsn}\footnotesize 条目类型小结
\end{CJK}}

The preceding section of the chapter has surveyed the |book|,
|article|, |inbook| and |incollection| types, and indirectly
|collection| too, in some detail (though without exploring every tiny
corner, for which the manual is invaluable). There are many other
types -- some supported by the standard styles, and some used by
specialist styles (for instance, styles for legal citation or which
include support for citing sound recordings). Rather than work
relentlessly through them, table \ref{entry:summary} suggests what
type you might use to cite a number of different types of
literature. Where the entry type is marked by an asterisk, that means
that it is not supported in the standard styles, and you should
probably be looking for a specialised style that does support it.

\begin{table*}
\begin{tabularx}{\linewidth}{llX}
\toprule
\textsf{literature}     & \textsf{suggested entry type} &                                              \\
\midrule
article (journal)       & \texttt{article}                                                             \\
article (in collection) & \texttt{incollection} or \texttt{inbook}                                     \\
article (in newspaper)  & \texttt{article}              & (some styles offer special formatting)       \\
article (unpublished)   & \texttt{article}              & use pubstate                                 \\
book                    & book                                                                         \\
letter (unpublished)    & \texttt{misc}                                                                \\
letter (published)      & \texttt{article}                                                             \\
manual                  & \texttt{report}                                                              \\
online item             & \texttt{online}                                                              \\
thesis                  & \texttt{thesis}               & set the \texttt{type} for the type of thesis \\
\bottomrule
\end{tabularx}
\caption{Sources and entry types\label{entry:summary}}
\end{table*}

%\section{Replacing text}
\section{文本替换}

\marginpar{\begin{CJK}{UTF8}{gbsn}\footnotesize 有时需要替换文本。比如对于同一个期刊或出版商,有时需要用全称,有时需要缩写。那么需要具有替换域信息的能力。

有两种方法可以实现:

1. 在bib文件中实现,定义string,然后用string名作为域的内容,主要string名不要带花括号。

2. 利用biber实现,biber的动态数据修改可以实现,利用直接的文本查找或者正则表达式,然后实现替换。
\end{CJK}}

\indexstart{database!abbreviations}\index{abbreviations!in database|see{database}}
One curse of bibliography generation is the abbreviation. Suppose you
have a journal called the `New York University Legal Studies
Quarterly'. Sometimes you will be told not to abbreviate it at
all. Sometimes you will be told to abbreviate it to `NY University
Legal Studies Q'; sometimes to `NYU Legal Studies Q'. And so
forth. (Slightly similar problems can apply with publishers: is it
`Oxford Univ.\ Press' or `OUP' or `Oxford University Press' or `Oxford
UP'?)

If you work in a field where this is not a problem---where standard
abbreviations are absolutely set in stone---then you are in
clover. In other fields, some element of flexibility is needed. There
are two ways you can achieve this, and one way that you can make your
life as difficult as possible.

If you want to make your life difficult, be sure to enter articles in
a radically inconsistent way, using different forms of the journal
name or publisher.

The two ways you can make your life easy are as follows.

\index{database!strings@database!@string}
First, you can use \emph{strings} in the |.bib| file. To do this,
define a string at the top of the file as follows
\begin{Verbatim}
@string{NYULSQ = "NY University Legal Studies Q"}
\end{Verbatim}
and then, in the |journaltitle| field, simply put |NYULSQ| (without braces)
\begin{Verbatim}
@article{...
  journaltitle = NYULSQ,
  ...
}
\end{Verbatim}
the string definition will be substituted for the abbreviation before
processing. That means that if you want to change all the entries to
read `NYU Legal Studies Q', you just have to change one line (in the
|string| definition.

The alternative is to be, at least, completely consistent. Then you
can use the power of \package{Biber} to search for and replace strings
in your file. For instance, suppose that throughout your file you had
entered the above-mentioned (fictitious) journal as |NYU Legal StudiesQ|,
but you are now writing a paper in which the proper abbreviation
is `NYU Leg Studs Q'. The following lines, placed in the preamble to
your file, will accomplish the necessary change:\label{datamap}

\begin{Verbatim}
\DeclareSourcemap{
  \maps[datatype=bibtex]{                % for .bib files
    \map[overwrite=true]{                % (over-writing if need be)
      \step[fieldsource=journaltitle,    % if the journaltitle field
            match={NYU Legal Studies Q}, % reads "NYU Legal Studies Q"
            replace={NYU Leg Studs Q}]   % replace it with "NYU Leg Studs Q"
    }
  }
}
\end{Verbatim}

\index{sourcemap directive}
A sourcemap directive is a powerful thing.\intref{\emph{Manual}
  \S~4.5.2.} It tells \package{Biber} to make changes to the input
file data \emph{before} outputting them for \LaTeX\ to use. In this
case, it tells it to apply the changes to any |.bib|
file.\footnote{Don't suppose that \texttt{datatype = bibtex} means that
  the \bibtex\ \emph{program} can do this. It's a \package{Biber}
  technique only!}

 The changes consist of a filtering step where it examines the
|journaltitle| field in any source and, if that field matches\linebreak
|NYU Legal Studies Q|, \package{Biber} will replace it with
|NYU Leg Studs Q|. This does not actually change your
|.bib| \emph{file}; but it alters the data as it passes from that file into
\biblatex\ for formatting.

As you can see, this trick will only work if you have been
\emph{consistent} in the way you have named journals and
publishers. But so long as you are consistent, it's a powerful
technique.
\indexstop{database!abbreviations}


%%% Local Variables:
%%% coding: utf-8
%%% mode: LaTeX
%%% TeX-master: "biblatex-tutorial"
%%% End:


\chapter{Customisation: A Short Tour}\label{ch:customize1}

We are about to dive into the details of using \biblatex. As we go
through them, we are going to be thinking a bit about
customisation.

\textsf{Biblatex} is highly customisable. That, in the end, is its
point. At one extreme, if you are setting out to write a bibliography
style for general use you might be customising a great deal. But at
that point it gets very complicated. If you are trying to do anything
that difficult you will really want to understand the internals of
\biblatex\ --- which will involve reading the manual
carefully,\intref{\emph{Manual} \S~4} and also looking at some of the
source code.\sidenote{Especially \texttt{standard.def} and the
  \texttt{.bbx} and \texttt{.cbx} files for the various standard
  files.} This guide is not trying to teach you how to do that. But
there are lots of customisations that are well within the reach of the
`ordinary' user, and it makes sense to discuss them. This chapter
provides a general introduction. Further details are given throughout
the guide in the places where they make most sense, and there is a
chapter\intref{Chapter \ref{ch:recipes}} that deals specifically with
`recipes' for various commonly-needed changes.

In order to keep things simple, I am going quite often to explain
\emph{how} to obtain certain commonly needed changes, without
necessarily explaining exactly why it works---what internal mechanism
is in play.

However, to avoid frustration, you may need to understand some of the
common ways in which changes are made, so that you can include the
right commands in the right way. The final section of this chapter
explains the bare minimum that you need to know.

Finally, do bear in mind that, with \biblatex\ there are often several
ways to achieve a particular result. You may find other suggestions
elsewhere about how to get \biblatex\ to do something, and they may
well be just as good (or better) than the ways I have
suggested. Generally, I've tried to keep things as simple as possible.

\clearpage
\section{The common methods}

\indexstart{customisation!general techniques}
\index{Biblatex!package options|see{options}}
\index{options}
\paragraph{Using an option.}\intref{\emph{Manual} \S~3.1}
Many `customisations' can be achieved simply by setting one of
\biblatex's package options. This is done when you load \biblatex. For
instance, to change the way the bibliography is sorted, you might set
the \verb|sorting| option to \verb|none|, by loading
\begin{verbatim}
\usepackage[style=numeric,
            sorting=none]{biblatex}
\end{verbatim}

Apart from the package itself, some commands also have options. For
instance, there are a number of options that can be given to the
\cs{printbibliography}\intref{See chapter~\ref{ch:customize1}}
command to affect how the bibliography is printed.

\paragraph{Redefining a command.} It's quite common for a change to be
made by redefining a command. For instance, in many cases \biblatex\
uses a command to insert a piece of punctuation, so that if you want
to change the punctuation,\intref{See p~\pageref{sec:punctuation}}
you can redefine the command to produce a different punctuation mark.

For instance, there's a command called \cs{multicitedelim} which
inserts the punctuation between multiple citations. Styles set it
differently, but it's usually a comma or a semicolon. Suppose, for
some really odd reason we wanted to make it `AND'. To do this, you use
\cs{renewcommand}.
\begin{verbatim}
\renewcommand{\multicitedelim}{\addspace AND\space}
\end{verbatim}

\index{punctuation}
\paragraph{Punctuation.} At this point, one feature of
bibliography-bashing is worth attention. If you do much of it, you
will find yourself spending quite a bit of time fiddling with
punctuation. It's not easy, but \biblatex\ has some rather
sophisticated ways of helping you.

When setting punctuation in \biblatex\ commands, it's often best
\emph{not} to use punctuation marks directly: for although they will
work, they don't `play' nicely with some of the clever tricks
\biblatex\ uses to keep punctuation straight. Instead, when engaged in
customisation, use the various commands that \biblatex\ offers.

\csindex{adddot}\csindex{addperiod}\csindex{addcomma}\csindex{addcolon}\csindex{addsemicolon}
\csindex{addspace}\csindex{addnbspace}\csindex{addexclam}\csindex{addquestion}
\index{punctuation!period}\index{punctuation!comma}\index{punctuation!semicolon}\index{punctuation!colon}
\index{spaces}\index{punctuation!exclamation mark}\index{punctuation!question mark}
\begin{margintable}
\begin{tabular}{lll}
\toprule
\cs{adddot}       &  \textbf{.} & (abbreviation dot) \\
\cs{addperiod}    & \textbf{.}  & (regular period) \\
\cs{addcomma}     & \textbf{,}  \\
\cs{addcolon}     & \textbf{:}  \\
\cs{addsemicolon} & \textbf{;}  \\
\cs{addexclam} 	  & \textbf{!}  \\
\cs{addquestion}  & \textbf{?}  \\
\cs{addslash}     & \textbf{\slash} \\
\cs{addspace}     &             & (regular space) \\
\cs{addnbspace}   &             & (unbreakable space) \\
\bottomrule
\end{tabular}
\vspace{3pt}
\caption{The \textbackslash add... commands\label{addcommands}}
\end{margintable}

\index{spaces}
\intref[3ex]{For spacing commands see the \emph{Manual} \S~4.7.4}
These (see table \ref{addcommands}) all take the form
\cs{add...}. Where they often have the edge on simply using
punctuation marks directly, is that they are context sensitive: they
will not add marks if it is inappropriate (for instance, they won't
add a comma after a question-mark), and they will remove unnecessary
white space before the mark. The best way to think of them is as
`context-sensitive' punctuation.

\index{strings!commonly used}
\csindex{DefineBibliographyStrings}
\paragraph{Bibliography strings.} Apart from the bits
of\intref{\emph{Manual} \S~4.9.1} bibliographical information
assembled from the database, all sorts of bits of text get printed in
a bibliography: `ed.', `vol.', `pp.'  and so on. These are handled by
\biblatex\ using bibliography strings. One quite common customisation
is to change these. This is done using the command
\begin{quotation}
\ttfamily
\cs{DefineBibliographyStrings}%
  \{\angled{language}\}\\%
  \quad\{\angled{bibstring} = \angled{definition},\\
  \quad \angled{bibstring} = \angled{definition} \}
\end{quotation} For
instance, if we decided that instead of printing `edited by' or `ed
by', \biblatex\ would produce `conjured by' we could alter the
\verb|byeditor| string:
\begin{verbatim}
\DefineBibliographyStrings{english}% or your language
    { byeditor = {conjured by}, }
\end{verbatim}

\index{formatting!field formats}
\csindex{DeclareFieldFormat}
\csindex{DeclareFieldFormat}
\paragraph{Field formats.} Internally, when it prints some data from
the \verb|.bib| file, \biblatex\ passes the data through a filter that
is defined for that field and entry-type. Although there are some
extra complexities about fields that hold names or lists, the basic
filter is set up using a command \cs{DeclareFieldFormat}.\manref{\S~4.4.2} The basic
structure of this is
\begin{pseudoverb}
\cs{DeclareFieldFormat}[\angled{entrytype}]\{\angled{field}\}\{\angled{format}\}
\end{pseudoverb}
The \angled{format} should just be the definition of a command which
accepts a single input (\texttt{\#1}). At its simplest, it
might simply print this:
\begin{pseudoverb}
\cs{DeclareFieldFormat}\{title\}\{\#1\}
\end{pseudoverb}
would just print the field, whereas
\begin{pseudoverb}
\cs{DeclareFieldFormat}\{title\}\{***\#1***\}
\end{pseudoverb}
would print the title surrounded by three asterisks, or
\begin{pseudoverb}
\cs{DeclareFieldFormat}\{title\}\{\cs{textsc}\{\#1\}\}
\end{pseudoverb}
would print the title in small capitals.

\csindex{DeclareListFormat}\csindex{DeclareNameFormat}
Because of the way that \biblatex\ works there are also formatting
directives for lists and
names.\marginnote{\cs{DeclareListFormat}\\\cs{DeclareNameFormat}}
These are more complex; ordinary users are probably best to leave them
alone; but they may be encountered from time to time in particular recipes.

\index{customisation!hooks}
\paragraph{Hooks.} There\intref{\emph{Manual} \S~4.10.6} are a number
of `hooks' offered by \biblatex\ at which it is made relatively easy
to insert user-defined commands which might do something useful. For
instance \cs{AtEveryCitekey}\csindex{AtEveryCitekey} allows one to
execute some code whenever a source is about to be cited, at a time
when its data is available but has not yet been handled;
\cs{AtEveryBibitem}\csindex{Ateverybibitem} offers a similar facility
in relation to the printing of bibliography items. Particularly
useful, sometimes, is the hook \cs{AtDataInput},\csindex{AtDataInput}
which is executed when data is first `read in' to become available for
citation. From time to time you may be encouraged to use these hooks
for various purposes.

\index{sourcemap directive}
\paragraph{Source mapping.} In fact,\intref{\emph{Manual} \S~4.5.2}
\biblatex\ can `get at' the data even before it is read in---indeed,
while it is being processed by \package{Biber}. The facilities
providing for source mapping enable one to play various useful tricks
with the \verb|.bib| file. We've already seen an example of this on
page \pageref{datamap}.

\paragraph{Deeper and darker.}
All these techniques (options, redefinition of `hook' commands, and
the redefinition of bibliography strings) are fairly simple, and
intended for regular use. Some customisations take us deeper: into the
internal workings of \biblatex. These things include the redefinition
of citation commands, the redefinition of things called bibmacros, and
ultimately the rewriting of bibliography drivers.
\indexstop{customisation!general techniques}

\section{Where to put customisations}

\indexstart{customisation!file locations}
This all depends on whether your customisations are `one off' or
intended to be used in all your documents.
\begin{itemize}
\item For one off customisations:
\begin{itemize}
\item customisation by options should go in the options that you list
  when loading \biblatex.\index{options}
\item most other customisations should go in the document preamble, after
  \biblatex\ has been loaded but before |\begin{document}|. In a \emph{very
  few} cases it's better to put them at some other point. Where that is so
  I specifically mention it.\index{customisation!in preamble}
\end{itemize}
\item If you want a standard set of options to apply to all your
  documents, put them in a file called |biblatex.cfg|, which should be
  put in your \emph{local} \textsc{texmf} tree at
  |/tex/latex/biblatex|. If there is such a file it will be loaded
  after \biblatex. It can be used to redefine commands and to set
  (most, but not all) bibliography options, using
  \cs{ExecuteBibliographyOptions}\braced{\angled{options}}. You
  cannot, however, set the bibliography style there: that at least you
  must do when you load \biblatex.\index{customisation!biblatex.cfg@\texttt{biblatex.cfg}}\index{files!biblatex.cfg@\texttt{biblatex.cfg}}\csindex{ExecuteBibliographyOptions}
\item If\intref{\emph{Manual} \S~4.2} you get to the point that you
  are developing a comprehensive style, you will want to write |.bbx|,
  |.lbx| and |.cbx| files which can be loaded as a style. That,
  however, is beyond the scope of this document.
\end{itemize}
\indexstop{customisation!file locations}

%%% Local Variables:
%%% coding: utf-8
%%% mode: LaTeX
%%% TeX-master: "biblatex-tutorial"
%%% End:

\chapter{Styles: The Menu}\label{ch:bibstyles}

\indexstart{styles!overview}
There are many different citation and bibliography styles available
for \biblatex, and half the battle is finding one which meets your
needs as closely as possible. But before looking at them in detail,
it's a good idea to try to identify them broadly.

\tikzstyle{level 1}=[level distance=2.5cm, sibling distance=2cm]
\tikzstyle{level 2}=[level distance=2.5cm, sibling distance=2cm]
\tikzstyle{level 3}=[level distance=2.5cm, sibling distance=1cm]
\tikzstyle{bag}=[text width=4em, text centered]
\tikzstyle{end}=[circle, minimum width=3pt, fill, inner sep=0pt]

\begin{figure}
\sffamily
\begin{tikzpicture}[grow=right]
\node[end]{}
  child {
    node[bag] {labelled}
        child {
           node[bag] {abstract} 
              child {
           node[bag, color=red]{numeric}}
              child {
           node[bag, color=red]{alphanumeric}
           }
          }
        child  {
           node[bag] {semi-abstract}
              child  {
                node[bag, color=blue]{author/year} }
              child  {
                node[bag, color=blue]{author/title}}
         }}
  child {
    node[bag,color=red!20!blue!30!green] {bibliographical}
  };
\end{tikzpicture}
\caption{Citation: Family Tree}
\end{figure}

The first big division is between styles which rely on \emph{labels}
and those that place full bibliographical information directly into
citations. In labelled styles the bibliography contains the essential
information about a citation, and the text contains references which
are not self-contained, but are intended to allow the reader to
cross-refer to the bibliography. In contrast, in unlabelled styles,
bibliographical information can be found in the citations
themselves. It may be repeated in the bibliography. But the reader is
not expected to need to look anything up in the bibliography to find
information about the work cited. However, because this consumes so
much space, it is common to find a system of cross-references and
abbreviations (including, in extreme cases, a panoply of latin
gadgets: ibid., op.\ cit., loc.\ cit., supra, infra).

The next division is between \emph{abstract} labelling systems and
\emph{partly meaningful} labelling systems. In an abstract labelling
system, the label carries no bibliographical information at all.
It might be simply a number. To follow it up, the reader
\emph{must} look it up in the bibliography. In a partly meaningful
labelling system (typically an author/year system), the label does
provide some information which is likely to mean something:
it tells you who wrote the work, and when. But the information is
still not comprehensive, and the reader who wanted to track down the
specific source would still need to use the bibliography.

\begin{table*}[tbh]
\setfloatalignment{t}
\caption[][-5cm]{Examples of citations in different families}
\begin{tabular}{lp{6cm}p{6cm}}
\toprule
& \textsf{first citation} & \textsf{later citations} \\
\cmidrule(lr){2-2}\cmidrule(lr){3-3}
\textsf{bibliographic} & M.\ Nussbaum, \emph{Poetic Justice} (Beacon
Press: Boston, 1995) & Nussbaum, op.\ cit.\ supra n. \emph{x}\\
& & ibid. \\
\textsf{author/title} & Nussbaum, \emph{Poetic Justice} & Nussbaum,
\emph{Poetic Justice}\\
\textsf{author/year} & (Nussbaum 1995) & (Nussbaum 1995) \\
\textsf{alphabetic label} & [Nus85] & [Nus85]\\
\textsf{numeric} & [1] & [1] \\
\bottomrule
\end{tabular}
\end{table*}

Of course, within each family there is considerable
variation. Numerical systems\marginnote{[1] (1) $^1$ [1, 5--8, 10] [1,
  8, 10, 7, 6, 5]} may place numbers in brackets, or parentheses, or
as superscript; they may sort and compress ranges, or leave them
uncompressed; they may sort the bibliography alphabetically and then
put references in, or they may present the bibliography in the order
in which the works happen to appear in the text. \marginnote{(Author
  1980) (Author: 1980) (Author, 1980)}Author/year systems may format
their references in a variety of different ways. And
full-bibliographical systems differ tremendously\marginnote{supra
  n. $\theta$, n $\theta$ above, (n $\theta$)} in the range of latin
gadgets they regard as acceptable or necessary. But these differences,
though important, are less pronounced than the fundamental differences
between the types of citation.  \indexstop{styles!overview}

\package{Biblatex} styles also fall into these three families, and
there are a number of `standard' styles in each. In principle,
\biblatex\ distinguishes between the `citation
style'\marginnote{\cs{usepackage[citestyle=...]\{biblatex\}}} (which
determines how citations appear in the text) and the `bibliography
style'\marginnote{\cs{usepackage[bibstyle=...]\{biblatex\}}} (which
determines how the list of references is printed; it is possible to
specify each separately. In practice, however, there tends to be a
connection between these things, and most users will not specify them
separately, but will simply specify a \verb|style| option when loading
\biblatex\, which will load both citation style and bibliography
style. That is what we assume here.

Another line of division within \biblatex\ styles is between generic
styles and specific styles. Generic styles (like the standard
\biblatex\ styles) reflect common citation practices but do not insist
on conforming to any particular style guide. Specific styles aim to
provide accurate implementations of particular detailed guides. In
principle, there is much to be said for specific styles, which are
more likely to be accurate reflections of citation practices in the
field to which they relate.

\indexstart{bibliography style!specific styles}
If you are looking for a \biblatex\ style the first question should be
whether there is an existing style which aims to implement the
particular citation style you are following. There are many such
packages available on \smallcaps{ctan} (a selection is given in table
\ref{ctan:bespoke}). However, you do need to be careful, because while
some of the packages---for instance \package{biblatex-apa} and
\package{biblatex-chicago}---are very stable and solidly maintained,
others are less polished and mature. The only solution is to test.

\index{bibliography style!APA}\index{bibliography style!AIP}%
\index{bibliography style!APS}\index{bibliography style!American Chemical Society}%
\index{bibliography style!Angewandte  Chemie}%
\index{bibliography style!Historische Zeitschrift}\index{bibliography style!Chicago}%
\index{bibliography style!IEEE}\index{bibliography style!MLA}%
\index{bibliography style!Nature@\emph{Nature}}%
\index{bibliography style!NEJM}%
\index{bibliography style!Royal Society of  Chemistry}%
\index{bibliography style!Science@\textit{Science}}%
\index{bibliography style!OSCOLA}%
\index{bibliography style!philosophy}%
\index{bibliography style!SBL}%
\begin{table}
\caption{Packages for Particular Styles\label{ctan:bespoke}}
\small
\begin{tabular}{llll}
\toprule
\textsf{style}                            & \textsf{package}                 & \textsf{version} & \textsf{date} \\
\midrule AIP                     & \texttt{biblatex-phys}           & 1.0b             & 2016          \\
APA                              & \texttt{biblatex-apa}            & 7.4              & 2017          \\
APS                              & \texttt{biblatex-phys}           & 1.0b             & 2016          \\
American Chemical Society        & \texttt{biblatex-chem}           & 1.1s             & 2017          \\
Angewandte Chemie                & \texttt{biblatex-chem}           & 1.1s             & 2017          \\
\textit{Historische Zeitschrift} & \texttt{historische-zeitschrift} & 1.1a             & 2014          \\
Chicago                          & \texttt{biblatex-chicago}        & 1.0rc1           & 2016          \\
IEEE                             & \texttt{biblatex-ieee}           & 1.2a             & 2017          \\
MLA                              & \texttt{biblatex-mla}            & 1.9              & 2016          \\
\textit{Nature}                  & \texttt{biblatex-nature}         & 1.3b             & 2017          \\
NEJM                             & \texttt{biblatex-nejm}           & 0.4              & 2011          \\
Royal Society of Chemistry       & \texttt{biblatex-chem}           & 1.1s             & 2017          \\
\textit{Science}                 & \texttt{biblatex-science}        & 1.1g             & 2016          \\
OSCOLA                           & \texttt{oscola}                  & 1.5              & 2017          \\
Philosophy                       & \texttt{biblatex-philosophy}     & 1.9              & 2017          \\
SBL                              & \texttt{biblatex-sbl}            & 0.8.1            & 2017          \\  
\bottomrule
\end{tabular}
\end{table}

\indexstop{bibliography style!specific styles}\index{bibliography style!general styles}

The second type of style is not tied to any concrete style guide or
set of rules, but aims to reflect general academic practice and to
provide a reasonable style which can be used or
adapted.\marginnote{See, in particular, the \texttt{biblatex-dw}
  styles.} The standard styles that come with \biblatex\ are in this
category. But there are also other styles, available on
\smallcaps{ctan} which are also like this. These styles often aim at
providing a large amount of configurability and
flexibility.\index{bibliography
  style!biblatex-dw@\texttt{biblatex-dw}}

\index{bibliography styles!choosing a style}
So how should you choose a style? The ideal situation is that the
exact style you want already exists. There's a good chance that is the
case. For instance, in the humanities, the Chicago styles (which offer
both an author/year and a full bibliographical form) are very commonly
used; and between the APA style (which is an author/year style), the
MLA style (which is an author/title style) and the various styles for
scientific journals, you may well find you are covered. If so, use
that style (and consult its documentation carefully). Scientists are
also well covered by a variety of different numeric styles. A variety
have been illustrated with examples at the end of this
document.\intref{See page~\pageref{chapter:examples}} The \biblatex\
documentation also includes many examples, together with the document
that produced them.

If there isn't an exact style that is precisely what you want, you
will have to find one that comes close to your needs, and either live
with it or adapt it. Provided your requirements are not rigidly
fixed,\footnote[][-3ex]{If requirements \emph{are} absolutely
  rigid, it sometimes turns out to be impossible to meet them
  precisely without adaptations that are not easy to make. Generally,
  if the adaptation requires re-ordering the information that is
  printed, it is likely to be fairly hard to do, whereas if it
  requires simple re-styling, it should be quite simple.} you can
probably find something which is satisfactory.

I'm not going to say much more of the precisely tailored styles:
you'll know if those are what you want. The rest of this chapter
surveys some of the standard (and a few of the other available
`general' styles) to help you understand what they provide. It's
helpful to consider them family-by-family. But bear in mind that
the main strength and purpose of \biblatex\ is not in its in-built
styles, which are intended rather as starting points for adjustment
than as a bibliographic \emph{dernier cri}. Still, much of what is
said about them is generally applicable.

\indexstart{bibliography style!numeric!standard}
\subsection{Standard Numeric Styles}

Numeric styles, though rather tedious to produce in the days of
typewriters and hot metal type, are quite convenient for computers,
and we will start with those. There's really room for variation in
four areas:
\begin{itemize}
\item Whether the references are listed in order of (first) citation,
  or sorted into a logical order (usually, alphabetical by name of
  author).
\item Exactly how the bibliographical entries are formatted internally
  (for instance whether journal articles are printed in italics, or
  put in quotation marks).
\item The precise format of references in the text. Numbers, of
  course: but are they in brackets [1] or parentheses (1) or
  superscript$^{1}$? Are they perhaps in bold \textbf{[1]}. If there
  are multiple citations are they separated by commas [1, 2]; or
  perhaps by semi-colons [1]; [2]?
\item How to handle ranges. If I cite items 1, 3, 4, 8, and 5, should
  I end up with the citation [1, 3, 4, 8, 5], or should it be sorted
  [1, 3, 4, 5, 8], or should it be sorted and compressed [1, 3--5, 8]?
\end{itemize}

How do the \biblatex\ standard numeric styles deal with these issues?

\index{bibliography style!numeric!bibliography format}
It's easiest to deal with the actual formatting of the bibliographical
material first, because that doesn't change: all the numeric styles handle it
in the same way. Figure \ref{numeric-examples} shows two very ordinary
book and article references, formatted as the standard styles do.

\begin{figure}
\fbox{\includegraphics{./examples/cotton-numericu.pdf}}
\caption{Numeric bibliography style\label{numeric-examples}}
\end{figure}

If that doesn't look right for you, you will need either to try to
identify an existing numeric bibliography style which is correct, or
to adapt the standard in some way.

\index{bibliography style!numeric!sorting
  citations}\index{bibliography style!numeric!compressing citations}
\newthought{Sorting is the next thing} to consider. In \biblatex\ that
is dealt with by a package option which specifies the sorting method
to be used. If you are used to \bibtex\ this will be unfamiliar to
you, because traditionally \bibtex\ `baked in' the sorting mechanism
to the bibliography style, whereas in \biblatex\ it can easily be
changed.

By default, the numeric styles in \biblatex\ sort the references into
alphabetical order: looking first at the name of the author or editor,
then at the title, and then at the year. If you want citations to be
listed in \emph{citation order}, you will need to specify the option
\begin{center}
\verb|sorting=none|
\end{center}
when you load \biblatex.\intref{See chapter~\ref{ch:sorting} for much
  more information on sorting.}

\begin{margintable}[4cm]
\begin{tabular}{lll}
\toprule
                       & \textsf{numeric} & \parbox{6ex}{\textsf{numeric-comp}} \\
\midrule\cs{cite\{a\}} & [1]              & [1]                                 \\
\cs{cite\{a,b\}}       & [1, 2]           & [1, 2]                              \\
\cs{cite\{b,a\}}       & [2, 1]           & [1, 2]                              \\
\cs{cite\{a,b,c\}}     & [1, 2, 3]        & [1--3]                              \\
\cs{cite\{c,a,b\}}     & [3, 1, 2]        & [1--3]                              \\
\bottomrule
\end{tabular}
\vspace{3pt}
\caption{Effect of compressing and sorting}
\end{margintable}
\newthought{That takes us to the form and order of references}, and at
this point the standard styles do offer some options, particularly in
terms of order.
\begin{enumerate}
\item If you want references \emph{uncompressed} and \emph{not
    reordered}, just load the style \verb|numeric|. The package will
  then print the references you give in any citation in exactly the
  order you cited them in.
\item If you want references \emph{sorted and re-ordered} but
  \emph{not} compressed, then load the style \verb|numeric|, but also
  add the option |sortcites=true|. (This, at least, you will surely
  nearly always want!)
\item If you want references \emph{sorted and compressed}, then load
  the style \verb|numeric-comp|
\end{enumerate}
(There's also a style called
  \texttt{numeric-verb}. It's described at in the \biblatex\
  documentation; but it's rather rarely useful.)

\index{bibliography style!numeric!superscript}
When it comes to changing the citation format, the most common
requirements can be met as follows For \emph{superscript citations},
you have three options:
\begin{enumerate}
\item Use the \cs{supercite}
  command\marginnote{\texttt{blah.}\cs{supercite\{a\}}
    $\rightarrow$blah.$^{1}$} instead of \cs{cite}: this will produce
  superscript citations.
\item Load \biblatex\ with the option \verb|autocite=superscript| and
  then use the \cs{autocite} citation command instead of \cs{cite}.
\item Remap \cs{cite} to \cs{supercite} (or to \cs{autocite} with the
  \verb|superscript| option set).
\end{enumerate}

\index{bibliography style!numeric!punctuation}
Changing the \emph{punctuation between multiple citations} is easy. By
default it's a comma. You change it by redefining
\cs{multicitedelim}.\intref{See further, chapter \ref{ch:recipes}} So
if you want a semicolon, you would put (in your preamble)
\begin{center}
\verb|\renewcommand{\multicitedelim}{\addsemicolon\space}|
\end{center}

Occasionally one sees styles where the label is in bold or some other
funny font. If you have to do that, you need to provide field formats
for \verb|labelprefix|, \verb|labelnumber| and
\verb|entrysetcount|. Each takes the same form (in the preamble of
your document, assuming you wanted bold):
\begin{center}
\verb|\DeclareFieldFormat{labelnumber}{\textbf{#1}}|
\end{center}
Or whatever other format you want.

If you want to use \emph{parentheses} instead of brackets around
citation labels ---so you get (1) instead of [1]---or if you want to
dispense with these delimiters altogether, you need to do rather more
work. This topic is covered in some detail below.\intref{p~\pageref{recipe:brackets}}
\indexstop{bibliography styles!numeric!standard}
\indexstart{bibliography styles!alphanumeric!standard}
\section{Alphanumeric Labels}

Alphanumeric labels look like [JW86] or [Jon95]. They are a half-way
house between the elegant simplicity of numerical citations, and the
additional information provided by author/year styles.

If you need or want to use such a system, load either the style
\verb|alphabetic| or \verb|alphabetic-verb|. The only difference
between the styles is how multiple citations are treated: with
\verb|alphabetic| you get them enclosed in a single set of
brackets\marginnote{\cs{cite\{a, b\}}\gives [AA01, BB02]} separated by
commas whereas with \verb|alphabetic-verb| you get each reference in
its own set of brackets, separated by
semicolons.\marginnote{\cs{cite\{a, b\}}\gives [AA01]; [BB02]}

The printed bibliography is obviously essential to a style that uses
alphanumeric labels, because the reader has to have a way of looking
up what the labels mean. The bibliography will look something like
figure \ref{example:bibliography:alphabetic}

\begin{figure}
\caption{A bibliography with alphanumeric
  labels\label{example:bibliography:alphabetic}}
\fbox{\includegraphics{./examples/cotton-alphabeticu.pdf}}
\end{figure}

\index{bibliography styles!alphanumeric!citation sorting}
\newthought{Compression obviously makes no sense} for labels (how
would one do it?) but you can, if you like have them sorted by setting
the option \verb|sortcites=true|: they will be placed in the same
order they appear in the bibliography. But you may be interested in
how the label actually gets constructed.

If you don't like the labels that \biblatex\ produces in a particular
case, you can override its decision in one of two ways.

\index{bibliography style!alphanumeric!custom labels}
First, if the \verb|shorthand| field is set, then \biblatex\ will use
it instead of the label. So, for instance, if we set
\begin{verbatim}
@book{book1,
  author = {Author, A.},
  date   = {1989},
  ...
  shorthand = {Masterpiece},
}
\end{verbatim}
then \verb|\cite{book1}| will produce [Masterpiece], not [AA89].

Alternatively, you can directly set the \verb|label| field. In
practical terms, that will produce the same result.
\begin{verbatim}
@book{book1,
  author = {Author, A.} 
  ...
  label = {MyLabel}
}
\end{verbatim}
and \verb|\cite{book1}| produces [MyLabel89]. You can see the difference:
whereas the \verb|shorthand| field is used `as is', the \verb|label|
field substitutes only the alphabetic part of the label: the year part
is still added automatically.

A still more advanced change would be to alter altogether the way that
labels are generated. This is an advanced customization, which is not
covered in this document: consult the \biblatex\
documentation.\intref{\emph{Manual} \S\ 4.5.4}

\index{bibliography style!alphanumeric!punctuation}
\newthought{In terms of formatting} there's not much room for
variation.
\begin{itemize}
\item You can change the punctuation that gets printed between
  multiple sources (by default, a comma in \verb|alphabetic| and a
  semicolon in \verb|alphabetic-verb|) by redefining
  \cs{multicitedelim}.
\item You can alter the font in which the label gets printed: you will
  need to alter the field format for both \verb|labelalpha| and
  \verb|extraalpha|. So, for instance, for bold labels you want
\begin{quote}
\verb|\DeclareFieldFormat{labelalpha}{\mkbibbold{#1}}|\\
\verb|\DeclareFieldFormat{extraalpha}{\mkbibbold{#1}}|
\end{quote}
\end{itemize}
\index{bibliography style!alphanumeric!standard}

\section{Author/Year Styles}

Author/year styles set citations in a form such as (Bloggs 1982). They
are common in many humanities subjects. Where an author has more
than one work in the same year, alphabetic labels are added to make it
clear which work is which (Bloggs 1982a, Bloggs 1982b). Labels are
usually placed directly in the text, not in footnotes. The
bibliography is necessary in order to provide detailed
bibliographical information that is lacking in the text: it is usually
organised to as to put author and year first in each entry, where it
can easily be found.

\begin{figure}
\caption{Author/year bibliography\label{example:bibliography:authoryear}}
\fbox{\includegraphics{./examples/cotton-authoryearu.pdf}}
\end{figure}

\indexstart{bibliography style!author/year!standard} The standard
\biblatex\ styles offer a small selection of different author/year
citation styles. (Among the other available styles, the APA style, and
the Chicago style if loaded with the \verb|authordate|
option\footnote{The \package{Chicago} style is loaded with
  \cs{usepackage[...]\{biblatex-chicago\}} rather than
  \cs{usepackage[style=chicago]\{biblatex\}}.} provide very fully
developed author/year styles, which might well be preferable to the
standard styles for many people.)

\index{bibliography style!author/year!compressing}\index{bibliography style!author/year!ibid}
\index{ibid!author/year styles, in}
The essential questions you need to ask in order to choose a style
are:\begin{itemize}
\item Do You want citations \emph{compressed}? In other words, if you
  cite (in a single citation) two works by the same author, do you
  want to see `Doe 1982; Doe 1983' (uncompressed) or `Doe 1982, 1983'
  (compressed).
\begin{margintable}
\begin{tabular}{lll}
\toprule
                          & \textsf{compress} & \textsf{ibid} \\
\midrule
\texttt{authoryear} \\
\texttt{authoryear-comp}  & \textbullet \\
\texttt{authoryear-ibid}  &             & \textbullet \\
\texttt{authoryear-icomp} & \textbullet & \textbullet \\
\bottomrule
\end{tabular}
\vspace{3pt}
\caption{Author/Year styles\label{authoryear:styles}}
\end{margintable}
\item Do you want to use `ibid' for repeated citations of the same
  source? (In other words, should successive citations of a given work
  produce `(Bloggs 1982) \ldots\ (ibid., p.~100)' instead of `(Bloggs
  1982) \ldots\ (Blogs 1982, p.~100)'. 
\end{itemize}
Depending on the answer to those questions, load the appropriate style
as shown in table \ref{authoryear:styles}. (If you want multiple
citations sorted but not compressed, you can set the option
\verb|sortcites| while loading an uncompressed style.)

\newthought{Besides selecting} from the range of styles, there's not
much room for simple customization; but there are three things you may
wish to adjust.

\index{bibliography style!author/year!names}
First, the number of names that get printed.\intref{See further, p~\pageref{recipes:maxnames}}  By default, up to three names will be
printed, both in the citations and in the bibliography. If there are
more than three names, they will be abbreviated to one name `et al'
--- again, both in the citations and the bibliography. So, for
instance, the entry:
\begin{Verbatim}
@book{companion,
  author = {Goossens, Michel and Mittelbach, Frank 
            and Samarin, Alexander},
  title = {The LaTeX Companion},
  date  = {1994},
  ...
}
\end{Verbatim}
will produce `Goosens, Mittelbach, and Samarin 1994'; but if one
additional author were added, the citation would become `Goosens et
al'.

This behaviour can be altered,\intref{For further detail, see
  p~\pageref{recipes:maxnames}} and it can be altered separately for
references in the text and those in the bibliography list. If you
prefer shorter lists of names, then use the option \verb|maxnames|: if
\verb|maxnames=1| then no more than one name will be used, and so
forth. You can, however, still have more than one name in the
bibliography, because you can set \verb|maxbibnames| to a different
number. So if, for instance, \verb|maxnames=1| but
\verb|maxbibnames=3|, then the in-text citations will read `Goosens et
al', but the bibliography will still read `Goosens, Mittelbach, and
Samarin \ldots'.

\index{bibliography style!author/year!punctuation}
A second adjustment you might want to make is to
punctuation:\intref{See generally, chapter \ref{ch:recipes}.}
\begin{itemize}
\item You might change the punctuation between author and the year. By
  default it is a space; one way to change it is by renewing the
  \verb|\nameyeardelim|.\intref{For more information about delimiters,
    and some more complex ways of handling them, see \emph{Manual},
    \S~3.10.2} For instance, if you wanted to place a colon, rather
  than a comma, you would
\begin{verbatim}
\renewcommand{\nameyeardelim}{\addcolon\space}
\end{verbatim}
  and citations would become `Author: 1980', and so forth. Another way
  of changing it uses the command
  \cs{DeclareDelimFormat}\manref{3.10.2}. So the same result is
  achived with\label{declaredelim}
\begin{verbatim}
\DeclareDelimFormat{nameyeardelim}{\addcolon\space}
\end{verbatim}
  Note that if you used the |\DeclareDelimFormat| method, the name of
  the delimiter is not preceded by a backslash.
\item You can change the punctuation between multiple citations, by
  changing \verb|\multicitedelim|, either using |\renewcommand| or
  using |\DeclareDelimFormat|.
\end{itemize}

\index{bibliography style!author/year!ibid}\index{ibid!ibidtracker option@\texttt{ibidtracker} option}
\label{citations:ibid}
The third set of adjustments that you might like to make arise only if
you are using one of the style versions that makes use of `ibid'. You
can control the circumstances in which ibid is used by setting the
\verb|ibidtracker| option.\label{ibidtracking} There are five possible
settings for \verb|ibidtracker| described in the \biblatex\
documentation; but in my experience only three of them are useful
\begin{description}[font=\ttfamily\upshape]
\item[ibidtracker=false] Don't ever use ibid.
\item[ibidtracker=constrict] Use ibid, but be cautious to avoid
  ambiguity. Footnotes and the main text are tracked separately, and
  ibid is used only if the reference will certainly be unambiguous.
\item[ibidtracker=true] The sloppiest use of ibid: use ibid whenever
  the previous reference was to the work in question, without tracking
  text and footnotes separately.
\end{description}
In most cases, the use of any setting sloppier than \texttt{constrict}
seems questionable (and this is the default setting).
\indexstop{bibliography style!author/year!standard}

\indexstart{bibliography style!author/title!standard}
\section{Author/Title Styles}

The various author/title styles all use the author's name as the
primary citation label. Depending on the particular style, a title is
either always added, or is added only where it seems necessary to
disambiguate the citation (because more than one work is being cited
by the same author). A bibliography remains essential to the style,
and will look like figure
\ref{example:bibliography:authortitle}. However, the author/title
styles lurk especially close the the borderline between labelled and
fully bibliographical styles, and are appropriate for use either in
footnotes or in running text. (In fact, the default styles which come
with \biblatex\ differ as to whether they place citations generated
with the \cs{autocite} command in text or footnotes.)

\begin{figure}
\caption{Author/title bibliography\label{example:bibliography:authortitle}}
\fbox{\includegraphics{./examples/cotton-authortitleu.pdf}}
\end{figure}

Simple as this may seem, there are actually a large number of standard
styles dealing with this. They can however, be selected by answering
three questions:

\index{bibliography style!author/title!sorting citations}
\index{bibliography style!author/title!compressing citations}
\begin{enumerate}
\item Do you want citations \emph{sorted and compressed}. Suppose you
  have two works by Joseph Bloggs, his \emph{First Book} and his
  \emph{Second Book}, and one by John Smith, his \emph{Book}. If you
  \verb|\cite{bloggs1,smith,bloggs2}| an uncompressed style will give
  you `Bloggs, \emph{First Book}; Smith, \emph{Book}; Bloogs,
  \emph{Second Book}'. A compressed style will give you `Bloggs,
  \emph{First Book}, \emph{Second Book}; Smith, \emph{Book}'.
\item \index{bibliography style!author/title!terse forms}Do you want
  citations to be \emph{terse}. If you use terse citations, then
  titles will be included only if they are required in order uniquely
  to identify the source. So, in the above example, titles will be
  included for Bloggs (because you are citing two works by him), but
  not for Smith, because you are citing only one. So
  \cs{cite}\texttt{\{bloggs1, bloggs2, smith\}} will give you `Bloggs,
  \emph{First Book}; Bloggs, \emph{Second Book}; Smith'.
\item \index{bibliography style!author/title!ibid}\index{ibid!author/title styles}If you
  cite the same work successively, do you want to use `ibid'. If you
  do, then
\begin{quote}
\ttfamily
Some \verb|\parencite{smith}| have pointed out that this is an unnecessarily
complex range of options \verb|\parencite[100]{smith}|.
\end{quote}
will give you 
\begin{quote}
Some (Smith, \emph{Book}) have pointed out that this is an
unnecessarily complex range of options (ibid, p.~100).
\end{quote}
\begin{margintable}
\begin{tabular}{lccc}
\toprule
                            & \textsf{compress} & \textsf{terse} & \textsf{ibid} \\
\midrule \textsf{authortitle}                                                    \\
\texttt{authortitle-comp}   & \textbullet                                        \\
\texttt{authortitle-terse}  &                   & \textbullet                    \\
\texttt{authortitle-ibid}   &                   &                & \textbullet   \\
\texttt{authortitle-icomp}  & \textbullet       &                & \textbullet   \\
\texttt{authortitle-tcomp}  & \textbullet       & \textbullet                    \\
\texttt{authortitle-ticomp} & \textbullet       & \textbullet    & \textbullet   \\
\bottomrule
\end{tabular}
\vspace{3pt}
\caption{Author/title styles\label{authortitle:styles}}
\end{margintable}
\end{enumerate}

Based on the answer to these questions, you can choose from among the
various author/title styles that are available. See table
\ref{authortitle:styles}.\footnote{The one missing style seems to be
  the combination of terse citations with ibid.} (If you want multiple
citations sorted but not compressed, you can use an uncompressed style
but set the \verb|sortcites| option.)

\newthought{Customization} is not often required, but resembles that
for the author/year styles.\intref{And see chapter \ref{ch:recipes}}
\begin{itemize}
\item \index{bibliography style!author/title!names} Customization of
  the number of names printed in citations and the bibliography is
  achieved using \verb|maxnames| and \verb|maxbibnames|, just as
  described above.
\item \index{bibliography style!author/title!punctuation} The separator
  between author and title is usually a comma. To change it, redefine
  \cs{nametitledelim}. For instance, to change it to a colon,
\begin{verbatim}
\renewcommand{\nametitledelim}{\addcolon\space}
\end{verbatim}
\item The delimiter between multiple citations is usually a
  semicolon. To change it, redefine \cs{multicitedelim}.
\item If you are using ibid,\intref{Page \pageref{ibidtracking}} you
  can configure it using the options given above.
\end{itemize}
\indexstop{bibliography style!author/title!standard}

\index{bibliography style!author/title!authortitle-dw@\texttt{authortitle-dw}}
\newthought{Apart from the standard styles} that come with \biblatex,
there is a very sophisticated, and highly configurable, style written
by Dominik Waßenhoven, the \texttt{authortitle-dw} style, which is
part of the \package{biblatex-dw} style package.

\section{Verbose styles}

\indexstart{bibliography style!verbose!standard}
The essence of verbose or bibliographical citation styles is that they
include information which is strictly bibliographical in the citations
themselves. Because of this, they are principally designed to be used
in footnotes --- the information the provide would clutter text up
much too much. And, quite often, in order to avoid excessive
repetition of the same data, these styles make use of a variety of
abbreviations or shorthands, such as \emph{ibid.}, \emph{op.\ cit.},
\emph{loc.\ cit.}, \emph{supra} and so on, though exactly which of
these are used and how varies quite a bit from style to style.

Because these styles place the full bibliographical data in the text,
the bibliography is more of an adjunct than an essential reference
tool. (It actually looks, in the standard styles, just like the
author/title bibliography in figure
\ref{example:bibliography:authortitle}.)

\index{bibliography style!verbose!Chicago}\index{bibliography style!verbose!footnote-dw@\texttt{footnote-dw}}
Standard\marginnote{The standard verbose styles are, I think, not
  especially usable as they stand; but they form an excellent starting
  point for customization.} \biblatex\ offers no fewer than seven
different flavours of verbose citation style. For practical use, there
should be added to these the \verb|footnote-dw| style from Dominik
Waßenhoven's \package{biblatex-dw} bundle, the
\package{biblatex-chicago} package, with option \verb|notes| and the
\package{oscola} style --- all of which build on the standard styles
to handle quite complex citations. In many cases these may be found
more convenient, though the standard styles are definitely the right
place to start if you are looking to produce a custom style of your
own.

As in the author/year and author/title styles, the essential choice
from the menu of verbose styles can be made by asking a few simple
questions.
\begin{enumerate}
\item How do you want to handle repeat citations. The first time you
  cite a source, all the verbose styles will give you a full
  citation. The second time, all will give you a shortened
  citation. But do you want that shortened citation to be simply a
  short name? Or do you want a cross-reference added to the note in
  which the source was first cited?
\item Do you want to use \emph{ibid} where two immediately sequential
  citations are to the same source?
\item Do you want to use additional scholarly abbreviations such as
  \emph{op cit} and \emph{loc cit}?
\end{enumerate}

\begin{table*}[btp]
\small
\begin{tabular}{lp{1.7in}p{1.7in}p{1.7in}}
\toprule
                                               & \textsf{first citation} & \textsf{subsequent} & 
\textsf{repeated}                                      \\
\midrule verbose                               & Frank Albert Cotton et al. \emph{Advanced inorganic
  chemistry}. 6th ed. Chichester: Wiley, 1999  & Cotton et al.,
\emph{Advanced inorganic chemistry}            & Cotton et al., \emph{Advanced
  inorganic chemistry}                                 \\
verbose-ibid                                   & Frank Albert Cotton et al. \emph{Advanced inorganic
  chemistry}. 6th ed. Chichester: Wiley, 1999  & Cotton et al.,
\emph{Advanced inorganic chemistry}            & Ibid. \\
verbose-note                                   & Frank Albert Cotton et al. \emph{Advanced inorganic
  chemistry}. 6th ed. Chichester: Wiley, 1999  & Cotton et al.,
\emph{Advanced inorganic chemistry}, see n. ?? & Cotton et al.,
\emph{Advanced inorganic chemistry}, see n. ??         \\
verbose-inote                                  & Frank Albert Cotton et al. \emph{Advanced inorganic
  chemistry}. 6th ed. Chichester: Wiley, 1999  & Cotton et al.,
\emph{Advanced inorganic chemistry}, see n. ?? & Ibid. \\
verbose-trad1                                  & Frank Albert Cotton et al. \emph{Advanced inorganic
  chemistry}. 6th ed. Chichester: Wiley, 1999  & Cotton et al.,
op.\ cit.                                      & Ibid. \\
verbose-trad2                                  & Frank Albert Cotton et al. \emph{Advanced inorganic
  chemistry}. 6th ed. Chichester: Wiley, 1999  & Cotton et al.,
\emph{Advanced inorganic chemistry},
op.\ cit.                                      & Ibid. \\
\bottomrule
\end{tabular}
\caption[][1em]{Various verbose styles.\label{bibliography:examples:verbose}}
\end{table*}

\newthought{Once that choice has been made} there is little scope for
customization. Or rather, there is such \emph{vast} scope for
customization of ever aspect of the citation --- much of which
requires customization of actual bibliography drivers --- that it lies
well outside the ambit of this chapter.

\index{ibid!in verbose styles}\index{bibliography style!verbose!ibid}
However there are some options which can be adjusted: the behaviour of
\verb|ibid| (see above),\intref{p~\pageref{citations:ibid}} for
example, or sorting schemes (see chapter \ref{ch:sorting}).
\indexstop{bibliography style!verbose!standard}


%%% Local Variables:
%%% coding: utf-8
%%% mode: LaTeX
%%% TeX-master: "biblatex-tutorial"
%%% End:


\chapter{Citation Commands}\label{ch:citationcommands}

The heart of any citation package, from the user’s point of view, is
the citation commands. These are what you use most. \biblatex\ offers a
number of different commands. They follow a consistent pattern; so
once you've mastered one it’s quite easy to grasp the others. Their
exact output depends on the citation style you are using.

\index{citation!general form}
Let's start not with the commands themselves, but by looking at some
citations. Consider the following:

\begin{quote}
See also [1]

(Jones 1990, p. 24)

cf [JJ90, ch. 3]

see Jones, \emph{Combinatorial Aesthetics} (London: Pubco, 1990)

ibid. 202
\end{quote}

There's a common pattern behind the obvious differences. Each citation
consists of up to three parts: an (optional) `signal' that introduces
it; the citation itself, which directly or indirectly identifies a
source; and an (optional) `pinpoint' which locates a particular part
of the cited work.

\begin{margintable}
\begin{tabular}{lll}
\toprule
\textsf{signal} & \textsf{citation} & \textsf{pinpoint} \\
\midrule 
See also &   [1] \\
         &   Jones 1990   & p. 24 \\
cf.      &   JJ90         & ch. 3 \\
see      &   Jones \ldots\    \\ 
         &   ibid.        &  202 \\
\bottomrule
\end{tabular}
\vspace{3pt}
\caption{The structure of citations}
\end{margintable}

\index{citation!general pattern}\index{prenote}\index{postnote}
The citation commands in \biblatex\ follow the same pattern: they have
one mandatory argument, which specifies the `key' of the source cited,
and two optional arguments. The primary optional argument is called
the \texttt{postnote} and gives the `pinpoint', and the secondary optional
argument is called the \texttt{prenote} and specifies the `signal'.

\begin{table}
\begin{tabular}{llll}
\toprule

                   & \textsf{prenote}  & \textsf{postnote} &  \textsf{key} \\
\midrule
\cs{cite[See also][]\{jones\}}   & See also    &        & jones \\
\cs{cite[24]\{jones\}}           &             & p. 24  & jones \\
\cs{cite[cf.][ch. 3]\{jones\}}   & cf.         & ch. 3  & jones \\
\cs{cite[see][]\{jones\}}        & see         &        & jones \\
\bottomrule
\end{tabular}
\vspace{6pt}
\caption{Citation command with prenotes and postnotes}
\end{table}


Notice that if there is only one optional argument it is assumed to be
the postnote; but if there are two, then the first optional argument
is the prenote. If we want a prenote but no postnote, as in the first
and last examples, we have to leave the second argument blank. But if
we want a postnote and no prenote, we just use one optional
argument. All the citation commands follow this format.

\section{The Basic Five}
\indexstart{citation!commands!general}

There are five basic citation commands: \cs{cite}, \cs{footcite},
\cs{parencite}, \cs{autocite} and \cs{textcite}. Exactly what each one
does depends on the citation style. But the general idea of the
circumstances in which you would use each is more or less the same.

\paragraph{\textbackslash cite.}\csindex{cite} The \cs{cite} command is “basic”. It
simply prints the particular style’s idea of a simple citation. So,
for instance, with a numeric style, \cs{cite} prints [1], and with a
verbose style it prints full bibliographical details (or ibid., or
supra n. \ldots, if that would be correct).


\paragraph{\textbackslash footcite and \textbackslash parencite.}\csindex{parencite} The
\cs{footcite} command tries to put its citation in a footnote. So
\cs{footcite\{jones\}} is means the same thing as \cs{footnote\{\textbackslash
  cite\{jones\}.\}} --- syntactic sugar. The \cs{parencite} command
does a similar thing, but it wraps its citation in parentheses, so
that it is equivalent to \texttt{(\cs{cite\{jones\}})}.


\paragraph{\textbackslash autocite.}\csindex{autocite} This is all very well. But what
if you’re not sure whether you should use \cs{cite} or \cs{footcite}
or \cs{parencite}. Suppose you write a paper, expecting to use
numerical citations; but then you decide to use verbose citations
instead. Now you have to go through and convert all your \cs{cite}s to
\cs{footcite}s. The \cs{autocite} command is intended to avoid
that. It will use whatever citation command is most appropriate to the
style selected. So if you use \cs{autocite}, you can (try to) change
all your citations from in-text citations to, say, footnotes simply by
changing the style.

It does another thing that is, as the kids say, cool. It will shift
punctuation. It needs to be able to do this because where a citation
appears relative to punctuation may vary. In English and American
practice, for instance, footnotes appear after punctuation. So whereas
a correct numerical citation would be `blah [1].' a correct verbose one
would be `blah.$^1$' With `\cs{autocite\{jones\}}.' \biblatex\ will move
punctuation from after the citation command to put the footnote marker
in the right place.

There are, however, limitations. The \cs{autocite} command can cope
well with citations at the end of sentences; but if citations are
entangled in the text too much, the result will not be satisfactory. A
sentence like
\begin{quote}
  Jones (1990) and Smith (1991) `independently concluded' (Doe 2002,
  p.~19) that foos should not be permitted to bar.
\end{quote}
will probably require some manual intervention.

\paragraph{\textbackslash textcite.}\csindex{textcite} Finally there is
\cs{textcite}. It’s not uncommon to have sentences like
\begin{quote}
   \ldots\ as Jones (1990) remarks \ldots
\end{quote}
where the citation is `woven' into the text of the sentence.

In numerical and verbose styles this can be done manually:
\begin{quote}
   \verb|as Jones \cite{jones} remarks| \gives\ as Jones [1] remarks
\end{quote}
But that's a bit wordy, and in author/year styles, this trick won't work:
\begin{quote}
   \verb|as Jones \parencite{jones} remarks| \gives\ as Jones (Jones 1990) remarks
\end{quote}
\cs{textcite} is designed to deal with this. It offers a citation in a form appropriate for running text.
\begin{quote}
   \verb|as \textcite{jones} remarks| \gives\ as Jones (1990) remarks
\end{quote}

Although mainly intended for author/year styles, \cs{textcite} works
rationally in any style; in verbose styles, the proper form involves
printing something in the text (such as `Jones') and adding a
footnote; if necessary punctuation will be moved for this purpose, so
that
\begin{Verbatim}
... as \textcite[11]{Jones}, presciently, remarked ...
\end{Verbatim}
will produce \marginnote{\tikz[overlay]{\draw (-0.5em,-0.5em) -|
    ([shift={(0.5ex,2.3ex)}] pic cs:comma)}The comma is moved so that
  it appears before the footnote.}
\begin{quote}
  \dots as
  Jones\colorbox{red!30}{\tikzmark{comma},\textsuperscript{1}}
  presciently, remarked
\end{quote}

\subsection{Summary}

Table \ref{allcitations} shows how the various citation commands
operate with the various different standard styles.

\begin{table*}
\scriptsize
\begin{tabular}{llllll}
\toprule
                                         & \cs{cite[9]\{jones\}}     & \cs{footcite[9]\{jones\}}  & 
\cs{parencite[9]\{jones\}}               & \cs{autocite[9]\{jones\}} & \texttt{as} \cs{textcite[9]\{jones\}}
says                                                                                                                        \\
\midrule \textsf{numeric}                & [1, p.~9]                 & $^*$1,
  p.~9.                                  & [1, p.~9]                 & [1, p.~9]                  & as Jones [1, p.~9] says \\
\midrule \textsf{alphabetic}             & [Jon94, p.~9]             & $^*$Jon94, p.~9            & [Jon94, p.~9]
                                         & [Jon94, p.~9]             & as Jones [Jon94, p.~9] says                          \\
\midrule \textsf{authoryear}             & Jones 1994, p.~9          & $^*$Jones 1994, p.~9       & (Jones
1994, p.~9)                              & (Jones 1994, p.~9)        & as Jones (1994,
p.~9) says                                                                                                                  \\
\midrule \textsf{authortitle}            & Jones, ``Title'', p.~9    & $^*$Jones, ``Title'',
p.~9                                     & (Jones, ``Title'', p.~9)  & $^*$Jones, ``Title'', p.~9 & as
Jones (``Title'', p.~9) says                                                                                                \\
\midrule \textsf{verbose}                & \parbox{2cm}{Jones. ``Title''. In: \emph{Journal}
  100 (1994), p.~7, p.~9}                & \parbox{2.2cm}{$^*$Jones. ``Title''. In:
  \emph{Journal} 100 (1994), p.~7, p.~9} & \parbox{2cm}{(Jones. ``Title''. In: \emph{Journal}
  100 (1994), p.~7, p.~9)}               & \parbox{2cm}{Jones. ``Title''. In: \emph{Journal}
  100 (1994), p.~7, p.~9}                & \parbox{3cm}{as Jones$^*$ says                                                   \\---\\
  $^*$Jones. ``Title''. In: \emph{Journal}
  100 (1994), p.~7, p.~9.}                                                                                                  \\
\bottomrule
\end{tabular}
\vspace{5pt}
\caption{Common citation commands in standard styles\label{allcitations}}
\end{table*}
\indexstop{citation!commands!general}

\section{Capital Forms}

\indexstart{citation!commands!capital forms} Each of \cs{cite},
\cs{parencite} and \cs{autocite} have `capital' forms (\cs{Cite}, etc)
which are appropriate for use where a capital would be required. You
might wonder why, since most citations already begin with a capital,
or with a number. The only time capitalisation might be needed is if
the citation begins with something like `von', and you want it
capitalised, or if the footnote reads `ibid' and you need
`Ibid'. There is no special capital form of \cs{footcite} because it
always assumes that the citation will begin a sentence so that a
capital may be appropriate.  \indexstop{citation!commands!capital
  forms}

\section{Specific to Numerical Styles}

\index{citation!commands!for numerical styles}\csindex{supercite}
For numerical styles, the command \cs{supercite} prints citations as
superscript numbers. It’s rather unusual to want both regular
full-size labels and superscript labels in the same document; but if
you do you can mix \cs{cite} and \cs{supercite} as you wish. Generally
speaking you want one or the other. For this, the safest course is to
use the option \texttt{autocite=super}, and use \cs{autocite}.

There is a limitation to \cs{supercite}. It doesn't make sense to try
to cram signals and pinpoints into superscript numbers. So
\cs{supercite} will discards prenote and postnote parts
\begin{quote}
\cs{supercite[see][25]\{jones\}} \gives\ $^1$
\end{quote}
If this happens, a warning will be given.

\section{Using particular fields}

\indexstart{citation!commands!particular fields}
There are a number of citation commands that are not intended to print
complete citations, but to pull potentially useful snippets out of a
source record, and treat it is a citation. They are mostly, though not
exclusively, useful in author/title and author/year styles, where
citations are often quite tightly woven into the text.

\csindex{citeauthor}\csindex{citetitle}\index{citation!commands!author}\index{author!citing name}\index{citation!commands!title}\index{title!citing title}
\paragraph{Author or title alone.} The commands \cs{citeauthor} and
\cs{Citeauthor} print the author's name (the latter with capitals
forced, even if they would otherwise not be used). The commands
\cs{citetitle} and \cs{citetitle*} print the title: abbreviated if
possible in the regular form, but always in full in the starred form.

\index{citation!commands!year}\csindex{citeyear}\csindex{citedate}\index{date!citing date}\index{year|see{date}}
\paragraph{Year or date.} The\marginnote{{\ttfamily
    \cs{citeauthor\{key\}} published his roman a clef
    (\cs{citeyear*}\{key\}) in \cs{citeyear}\{key\}} \gives\ Locke
  published his roman a clef (1905b) in 1905} commands \cs{citeyear}
and \cs{citeyear*}, and \cs{citedate} and \cs{citedate*}, print year
or date. The difference is that whereas the unstarred versions just
print the calendar year or date, the starred versions print any extra
label that is added to that for the particular work. So if you had,
say, five works by a particular author in one year,
\verb|\citeyear{prolific86c}| would print `1986', whereas
\verb|citeyear{prolific86c}| would print `1986c'.

\index{citation!commands!url}\index{URL!citing}\csindex{citeurl}
\paragraph{URLs.} The \cs{citeurl} command will print any url.

\paragraph{Others.} Apart from these commands, which the standard
styles already provide, it is possible to construct your own `partial'
commands,\manref{\S~3.8.7} using \cs{citename}, \cs{citelist} and
\cs{citefield}. There are three commands because \biblatex\
distinguishes between \verb|name| fields (like \texttt{author} and
\texttt{editor}), list fields (like \texttt{institution} and
\texttt{publisher}) and other fields. The commands are not really
suitable or intended for direct use: they would need to be `wrapped'
in some more user-friendly command for use in a document.
\indexstop{citation!commands!particular fields}

\section{Entry sets}

\index{citation!commands!entry sets}\index{sets of sources cited together|see{entryset}}\index{entryset@\texttt{entryset}}
In some disciplines it is common to have a single numeric citation
refer to a group of sources. This is supported in \biblatex\ by the
entry-set.\manref{\S~3.12.5}

\index{entryset@\texttt{entryset}!bib file in@in \texttt{.bib} file}
There are two ways to define such a set. One way is to define them as
a |key| in the |.bib| file. Thus, for instance, your |.bib| file might
define
\begin{Verbatim}
@set{set1,
  entryset={augustine, cotton}}
\end{Verbatim}
You can then use |\cite{set1}| to get output as in figure \ref{entryset1}.
\begin{figure}
\fbox{\includegraphics{examples/entryset1.pdf}}
\caption{An \texttt{entryset}\label{entryset1}}
\end{figure}

\csindex{defbibentryset}\index{entryset@\texttt{entryset}!using defbibentryset@using \cs{defbibentryset}}
The alternative, and usually more convenient approach, is to define
such sets `on the fly' in each document. This can be done using the
command
\begin{pseudoverb}
  \centering\cs{defbibentryset}\{\angled{set-key}\}\{\angled{key$_{1}$
    \ldots\ key$_{n}$}\}
\end{pseudoverb}
so that
\begin{pseudoverb}
  \centering
\cs{defbibentryset}\{set1\}\{augustine, cotton\}
\end{pseudoverb}
will produce the same output as figure \ref{entryset1}.

\section{Pinpoints}

\indexstart{citation!postnote}\index{pinpoints|see{citation, postnote}}
It usually makes sense to give pinpoint citations to a particular work
primarily by reference to one type of marker. So, for instance, books
are normally cited by page. How such references are styled depend on
the particular scheme. Some schemes want `p. 23', others want `p 23',
others just want `23'. To save effort, \biblatex\ will insert
appropriate default labels. So if you just type \cs{cite[23]\{key\}},
\biblatex\ will produce `p. 23' (or whatever is appropriate to your
style). It works also for multiple pages: \cs{cite[22, 23]\{key\}}
will print `pp. 22, 23' and \cs{cite[22--25]\{key\}} will print
`pp. 22--25', and do forth.


What if you want some different prefix? If the problem is with a
particular citation, you can just enter the prefix manually. For
instance, if you want to cite chapter 3, just type
\cs{cite[ch.\textasciitilde 3]\{key\}}. Since \biblatex\ only adds a
prefix where you have provided a reference which consists only of
numbers, your postnote will be printed as you have entered it.

\newthought{This is neat and tidy.} But, as so often there are special
cases and customisations.

\index{citation!postnote!suppress prefixes}
\paragraph{No prefixes} You don't want prefixes, even when you enter a
purely numerical postnote: you want \cs{cite[22]\{key\}} to produce
`\angled{key}, 22'. There are a number of ways to achieve this,
depending on how generally you want to disable the use of prefixes.
For a single citation, start the postnote with the special command
\cs{nopp}.  \marginnote {{\ttfamily \cs{cite}[\cs{nopp}22]\{key\}}
  \gives\ \angled{key}, 22} This is most appropriate where you have a
single citation that is giving you problems.

To\marginnote{%
  {\ttfamily @book\{key,...\\
    \quad pagination = \{none\}\}} \\%
  \cs{cite[22]\{key\}} \gives\ \angled{key}, 22} stop all additions
for a particular source, set that source’s pagination to ’none’. This
is most appropriate when you generally want to use prefixes, but you
have a particular source where they are not required.

To stop all additions, for every source:
\begin{center}
\verb|\DeclareFieldFormat{postnote}{#1}|
\end{center}
This is most appropriate when you are customising \biblatex\ for a
style which does not use any prefixes in general.

\index{citation!postnote!force prefixes}
\paragraph{Forcing prefixes} You want to enter a postnote with a
non-numerical character, and still get a prefix added. For instance,
you have a source with pages which have letters, like `1234a'. Again,
there are a number of solutions.

\csindex{pno}\csindex{ppno}
The\marginnote{\cs{cite[p.\textasciitilde 12a]\{key\}} \gives\
  \angled{key}, p.~12a} easiest thing to do in such cases is simply to
type the prefix yourself: \cs{cite[p.\textasciitilde
  1234a]\{key\}}. But if you think it’s worth getting \biblatex\ to do
it for you, you can use \cs{pno}\marginnote{\cs{cite[\textbackslash
    pno 12a]\{key\}} \gives\ \angled{key}, p.~12a} (for a single page)
and \cs{ppno}\marginnote{\cs{cite[\textbackslash ppno
    12a-{}-b]\{key\}} \gives\ \angled{key}, pp.~12a--b} (for multiple
pages). (In theory, the \cs{pno} and \cs{ppno} commands could be
`better' because you could change the style of prefixes and they would
adjust automatically. In the real world, there's probably not much
practical advantage.

\index{citation!postnote!Latin gadgets}\csindex{psq}\csindex{psqq}
\paragraph{Latin gadgets for multiple pages}There is one special
common case. Suppose you want `p.~20~ff.' or `p.~20 et seq.'  You
could of course enter the whole postnote manually. But you can also
use the special commands \cs{psq}\marginnote{\cs{cite[20\textbackslash
    psqq]\{key\}} \gives\ \angled{key}, pp.~20 sqq.} and \cs{psqq} to
add the indication. The advantage is that the particular phrase to
refer to subsequent page(s) can then be set consistently. You should
not put any space between these commands and the number:
\cs{cite[20\textbackslash psqq]\{key\}}.

By default, in the standard styles, \cs{psq} gives `sq.' and \cs{psqq}
gives `sqq.' To redefine them, redefine the bibliography strings
\verb|sequens| (for a single following page), and \verb|sequentes|
(for more than one). For instance, to use `et seq.' for both, one
would (assuming we are using American English)
\begin{verbatim}
\DefineBibliographyStrings{american}{%
    sequens = {et seq\adddot},
    sequentes = {et seq\adddot}}
\end{verbatim}
The space before this addition is added by a command called
\cs{sqspace}. Some citation styles don't add any space at this point,
preferring `10f.' to `10 f.' If you don't want any space, redefine
this command:
\begin{center}
\verb|\renewcommand{\sqspace}{}|
\end{center}

\paragraph{Correcting \biblatex's guesses} \biblatex\ tries to
guess whether to add `p.' or `pp.' by working out whether you have
cited a single page or a range of pages. But occasionally the form of
a citation might confuse \biblatex, so that it adds `p.' when you need
`pp.', or vice versa. In such cases, either type the whole postnote
yourself, or use the \cs{pno} or \cs{ppno} commands, as explained
above.

\index{citation!postnote!customisation}\index{customisation!postnote}
\paragraph{You want something \ldots\ but not `pp.'} Some sources
don't use pages for pinpoint citations. For instance, suppose you are
citing a work which is normally referred to by section, not page. In
such a case, you can still use the convenience of having automatic
prefixes. What you need to do is to set the pagination field of the
source in the \texttt{.bib} file to `section'. A list of the types of
pagination recognised, and their results, is shown in table
\ref{pagination:types}.\index{pagination!in postnote}
\begin{margintable}
\begin{tabular}{lll}
  \toprule
  \textsf{pagination} & \textsf{singular} & \textsf{plural} \\
  \midrule
  \texttt{page}       & p.                & pp.             \\
  \texttt{column}     & col.              & cols.           \\
  \texttt{section}    & \S                & \S\S            \\
  \texttt{paragraph}  & par.              & pars.           \\
  \texttt{verse}      & v.                & vv.             \\
  \texttt{line}       & l.                & ll.             \\
  \texttt{none}                                             \\
  \bottomrule
\end{tabular}
\vspace{3pt}
\caption{Standard values for \texttt{pagination}\label{pagination:types}}
\end{margintable}


That’s fine so long as \biblatex\ recognises the pagination type. But if
it doesn't, you will get odd results. For instance, suppose we want
references in the form `Art. x', so that \cs{cite[2]\{key\}} gives us
`Art. 2'. If we enter the pagination as \texttt{article}, however, we
just get `p.', because \texttt{article} is not a
pagination type that \biblatex\ knows.

How can we `teach' \biblatex\ a new pagination type? In this example,
we simply need to define two bibliography strings---article and
articles:

\begin{verbatim}
\NewBibliographyString{article, articles}
\DefineBibliographyStrings{american}{% or your language
   article  = {Art\adddot}
   articles = {Arts\adddot}}
\end{verbatim}
Now all will work as intended. 

\paragraph{Modifying standard prefixes} To change the standard
prefixes, you can use essentially the same trick -- only you don't
need to define the relevant bibliography strings because they already
exist. So, for instance, if you wanted the abbreviation for page to be
“pg”, you could redefine the relevant strings as follows

\begin{verbatim}
\DefineBibliographyStrings{american}{% or your language
  page  = {pg},
  pages = {pgs}}
\end{verbatim}
\indexstop{citation!postnote}

\section{Multiple citations}

\indexstart{citation!multiple}\csindex{cites}
You've probably worked out by now that even the standard citation
commands will handle multiple citations: \cs{cite\{key1, key2\}} will
produce `[1, 2]' or `Jones 1990; Smith 2010', or whatever is in
keeping with the style you have chosen.

\begin{margintable}
\begin{tabular}{ll}
\toprule
\textsf{regular} & \textsf{multi-cite version} \\
\midrule
\cs{cite}        & \cs{cites} \\
\cs{footcite}    & \cs{footcites} \\
\cs{parencite}   & \cs{parencites} \\
\cs{autocite}    & \cs{autocites} \\
\cs{textcite}    & \cs{textcites} \\
\cs{Cite}        & \cs{Cites} \\
\cs{Parencite}   & \cs{Parencites} \\
\cs{Autocite}    & \cs{Autocites} \\
\cs{Textcite}    & \cs{Textcites}\\
\bottomrule
\end{tabular}
\vspace{3pt}
\caption{Multiple citation commands\label{multicites}}
\end{margintable}


This is fine; but it doesn't handle pre- and postnotes very nicely. If
we enter \cs{cite[see][11]\{key1,key2\}} we get something `[see 1;
2, p.~11]' --- which will only make sense if the page reference is
to the last work cited. For this purpose, \biblatex\ offers a
selection of commands tailored for multiple citations.

The pattern is set by the \cs{cites} command. If we type
\begin{center}
\verb|\cites[See][10]{key1}[cf][20]{key2}[30]{key3]|
\end{center}
we would get (in a numeric style)
\begin{center}
[see 1, p.\ 10; cf 2, p.\ 20; 3, p. 30]
\end{center}
or (in an alphabetic style)
\begin{center}
see Smith 1990, p.\ 10; cf Jones 2000, p.\ 20; Bloggs 2010, p.\ 30
\end{center}

It's not hard to see what's happening here: basically \biblatex\ is
interpreting \cs{cites} as a set of individual \cs{cite} commands,
each with its own pre- and/or postnote, but also understanding that
they form a unit (so, for instance, the numeric style places them in
one set of brackets).
\begin{center}
\ttfamily
\cs{cites}\colorbox{red!30}{[see][10]\{key1\}}\,%
\colorbox{green!30}{[cf][20]\{key2\}}\,%
\colorbox{blue!30}{[30]\{key3\}}
\end{center}

To make things even more exciting, you can also have a pre- or
postnote for the \emph{entire} multiple citation: you just place them
(before anything else) in parentheses:
\begin{center}
\ttfamily
\cs{cites}(see:)(among others)[10]{key 1}[20]{key 2}
\end{center}
giving
\begin{center}
see: Smith 1990, p.\ 10; Jones 2000, p.\ 20, among others
\end{center}

But there's one catch. As it scans forwards, the \cs{cites} command will
be `tricked' if it sees a square bracket (\texttt[) or a brace
(\texttt\{) which is not intended to start a citation. For instance,
in:
\begin{center}
\ttfamily
\cs{cites}\colorbox{red!30}{[10]\{key
  1\}}\,\colorbox{green!30}{[20]\{key2\}}
\colorbox{blue!30}{[`{}`the principal works'{}']}
\end{center}
a human can quickly see that the blue portion is not another
citation, but some text in brackets. But \biblatex\ won't appreciate
this (a space is not enough), and will complain that you have a
defective citation. The trick in such cases is to put \cs{relax} at
the end of the list:
\begin{center}
\verb|\cites[10]{key1}[20]{key2}\relax [`the principal works']|
\end{center}

All the main citation commands (and their capitalised forms) have
multiple versions, consistently named: see table \ref{multicites}.
\indexstop{citation!multiple}

\section{Citation commands that \ldots\ don't}

\csindex{nocite}
Finally we have to consider a small family of odd citation
commands. Normally the point of including a citation is to have
\emph{some sort} of information about the source in question. It
might, therefore, seem odd that there are citation commands which
don't print anything. But they are sometimes needed. For instance, you
might want to include a work in the bibliography without actually
printing any citation in the text. To do that you can use the \cs{nocite}
command. So \cs{nocite\{angled{key}\}} will place \emph{key} in the
bibliography without citing it. You can also use \cs{nocite\{*\}},
which will put \emph{every source} in your bibliography file into the
bibliography.

\csindex{mancite}
Another occasionally useful command is \cs{mancite}. You use this
command to tell \biblatex\ that you have manually cited a work. You
might like to do this if you want to weave the citation into the text
in a way that is too complex to achieve using the various citation
commands. It may be worth doing, because it will make sure that
\biblatex\ keeps track of internal housekeeping that matters. For
instance, suppose you had the following:
\begin{verbatim}
... is clear \parencite{jones}. Smith's `rebuttal' (see 
above) hardly meets this point; but this is not Jones's 
main objection \parencite[45]{jones}. ...
\end{verbatim}
If you were using a citation style which used `ibid', the second
reference to Jones would be printed as `ibid', since as far as
\biblatex\ is concerned there are two consecutive citations to it. But
the sentence describing Smith's rebuttal is a sort of implicit
citation, and you might think `ibid' ambiguous in this context. So you
could type
\begin{pseudoverb}
... is clear \cs{parencite}\{jones\}. Smith's `rebuttal' (see\\
above){\bfseries\cs{mancite}\{smith\}} hardly meets this point; but this is\\
not Jones's main objection \cs{parencite}[45]\{jones\}. ...
\end{pseudoverb}
Generally, however, it's better to avoid such implicit citations.

%%% Local Variables:
%%% coding: utf-8
%%% mode: LaTeX
%%% TeX-master: "biblatex-tutorial"
%%% End:

%\chapter{The Bibliography}
\chapter{参考文献表}
\label{ch:bibliographyformat}
\index{customization!bibliography title|see{bibliography, title}}
\index{customization!bibliography heading|see{bibliography, heading}}

\index{bibliography!printing}
Citations written, we come to the actual printing of the
bibliography. The magic command is:\csindex{printbibliography}\manref{\S~3.7.7}
\begin{center}
\cs{printbibliography}\manref{\S~3.7.2}
\end{center}
And, for simple cases, it may even be as easy as that! But there are
quite a number of refinements.

The refinements can be classified as follows:\marginpar{\footnotesize printbibliography 是打印文献表的基本命令,对其还有做如下几类修改:标题、条目格式、排序、筛选、多文献表。
}
\begin{itemize}
\item Changes to the way the bibliography is headed: what is it
  called? Is it treated as if it were a new chapter, or a new section?
  How is the heading printed? Is it included in the table of contents?
\item Changes to the appearance of the bibliography items: what font
  are they printed in? Are they indented? How? How much?
\item Changes to the way the bibliography is sorted.
\item Selection of the material to be printed in the bibliography: are
  all sources cited, or only some? How are they selected?
\item Provisions for creating multiple bibliographies, such as
  different bibliographies for different chapters, or different
  bibliographies for different topics.
\end{itemize}
The last two items overlap a bit, because in one sense producing
multiple biographies is a matter of selecting what works go in which
bibliography. It's simpler, however, to defer most discussion of this
to a chapter of its own.\intref{Chapter~\ref{ch:subdivisions}} We are
also going to leave sorting for a separate discussion. So this chapter
is really going to concentrate on the headings, the physical format of
the bibliography.

%\section{Headings}
\section{标题}

\indexstart{bibliography!heading}
In a \LaTeX\ document, a heading is more than just its words: a
heading command (such as \cs{chapter} or \cs{section}) doesn't just
print its text, formatted a certain way. It may add a number, and
advance a counter. It may change what gets printed in the running
headers or footers.

\index{bibliography!title} Let's start with the actual title. A
default\footnote{The default is `Bibliography' for books and articles,
  and `References' for articles.} will be set by the |heading| option,
described below. The easiest way to change it is using the |title|
option in the \cs{printbibliography} command.

So, for instance, if we wanted to call our bibliography `List of
Sources', we could put the following:
\begin{center}
\cs{printbibliography[title=\{List of Sources\}]}
\end{center}

This is fairly straightforward. Beyond that, some degree of complexity
lurks.

It's all fairly easy if you want to have the bibliography title
treated as if it were a chapter heading (for the |book| and |report|
classes) or a section heading (for the |article| class). This is,
after all, the ordinary case, and \biblatex\ then provides you with
the following options:
\begin{itemize}
\item If \marginnote{\gives\ \textbf{References}\\(not in
    table of contents)}you want the heading inserted, but without any
  number and without being added to the table of contents, you're all
  set. This is what \biblatex\ will do by default.
\item If\marginnote{\gives\ \textbf{References}\\(in table of
    contents)} you want the heading added to the table of contents,
  but you don't want it numbered, then add the option
  |heading=bibintoc| to your \cs{printbibliography} command.
\item If\marginnote{\gives\ \textbf{6.\ References}\\(in table
    of contents)} you want the heading numbered and added to the table
  of contents, then add the option |heading=bibnumbered| to your
  \cs{printbibliography} command.
\end{itemize}

It's also easy enough if you want the bibliography dealt with at
\emph{one level below} the default: in other words, you want it
treated as a section (in the |book| or |report| classes) or
as a sub-section (in the |article| class:
\begin{itemize}
\item To have just the heading (no table of contents or numbering),
  choose the option |heading=subbibliography|.
\item To have the heading included in the table of contents, but not
  numbered, choose the option |heading=subbibintoc|.
\item To have the heading numbered and included in the table of
  contents, choose the option |heading=subbibnumbered|.
\end{itemize}

Finally (and perhaps easiest of all), if you want no heading at all,
just choose |heading=none|.

\newthought{These are all the options} that \biblatex\ gives you by
default. Combined with |title=...| (to overrride the default titles)
this will usually suffice. If, however, you want to define a special
heading for your bibliography, you will need to proceed as
follows:\intref{\emph{Manual} \S\ 3.6.7}
\begin{enumerate}
\item Define a bibliography heading style using\csindex{defbibheading}
\begin{center}
  \cs[\angled{title}][\angled{definition}]{defbibheading\{\angled{name}\}}
\end{center}
The \angled{name} is a name you will use, as an option passed to\linebreak
\cs{printbibliography}, to refer to the type of heading that you have
defined; it can be anything you choose. The \angled{title} is the
default title for your heading. The definition part is a complete set
of commands to create a heading and anything (such as the marking of
headers and footers, table of contents, and so forth) that needs to go
with it. It takes the form of a macro definition which assumes it will
be passed the title of the bibliography as its single argument, |#1|.
\item Use the name of the |heading| you have defined as the
  |heading| option to your \cs{printbibliography} command.
\end{enumerate}

So, for instance, suppose we wanted to set up our bibliography to be
formatted as a subsection, but without any number; we also want it
added to the table of contents, and we want it to mark both recto and
verso pages. The default title is to be `Works Cited':
\begin{verbatim}
\defbibheading{myheading}[Works Cited]{%
  \subsection*{#1}%
  \addcontentsline{toc}{subsection}{#1}%
  \markboth{#1}{#1}}
\end{verbatim}

There's one tiny trick to bibliography headings. It's normally best to
do all your customization in the preamble of your document, before you
(or \LaTeX) gets to \cs[document]{begin}. But for bibliography
headings (and title, and notes) it's actually better to set them up
after \cs[document]{begin}.
\indexstop{bibliography!heading}

%\section{Bibliography appearance}
\section{文献著录格式}

\indexstart{bibliography!appearance}
The appearance of a bibliography is largely controlled by a
pre-defined bibliography environment. You can change the default by
defining a modified bibliography environment of your own, and then
passing the name of that environment as
\begin{center}
\cs{printbibliography[env=\angled{your environment}]}
\end{center}

\csindex{defbibenvironment}
\index{customization!bibliography appearance}
The environment needs to be set up in advance, for that purpose, you
use the command \cs{defbibenvironment}:
\begin{pseudoverb}
\cs{defbibenvironment}\{\angled{name}\}\{\angled{start}\}\{\angled{end}\}\{\angled{per-item}\}
\end{pseudoverb}
The way this works is as follows:
\begin{description}
\item[\angled{name}] is the name you choose to refer to this style of
  environment: it's what you will use with the option \texttt{env=} to
  typeset a bibliography using this environment.
\item[\angled{start}] is code to be executed at the beginning of the
  bibliography, before any entries are printed. It's customary (though
  not essential) to set a bibliography using a list environment of
  some sort, and this lets you set that up.
\item[\angled{item}] is code to be executed at the beginning of each
  \emph{entry} in the bibliography: for instance \cs{item}.
\item[\angled{end}] is code to be executed at the end of the
  bibliography, for instance to end the environment you began with
  \angled{start}.
\end{description}

Let's give a practical example. People writing CVs sometimes want to
number the items while using a non-numeric bibliography style, such as
verbose or author-year. Suppose you loaded the |authoryear| style, but
then defined a bibliography environment as follows:
\begin{verbatim}
\defbibenvironment{myenv}
  {\begin{enumerate}}
  {\end{enumerate}}
  {\item}
\end{verbatim}
Now, using
\begin{pseudoverb}
\cs{printbibliography}\relax[env=myenv,heading=myheading]
\end{pseudoverb}
we get

\begin{figure}
\framebox{\includegraphics{./examples/cotton-springer-myenv.pdf}}
\caption{Customized bibliography environment\label{custom-env}}
\end{figure}
\indexstop{bibliography!appearance}

%\section{Other lists and indexing}
\section{其它列表和索引}

\index{shorthand!list of shorthands}
If you use `shorthands'\intref{See p~\pageref{shorthands}} to define abbreviations for particular works, you can print these with the command:
\begin{pseudoverb}
  \centering
  \cs{printbiblist}[\angled{options}]\{shorthand\}
\end{pseudoverb}

\index{indexing}
You will almost certainly want to provide a suitable title using the
option |title=...|. As the general form of the command suggests, it is
in fact an instance of a general way of printing special forms of
bibliographical list.\manref{\S~3.7.3}

\biblatex\ also enables styles to permit the creation of an
\emph{index} of sources. The details are elaborate, and you will need
to understand something about how \LaTeX's indexing methods work. Some
styles (such as \package{oscola}) make very extensive and complex use
of this facility. But, as it is a rather advanced topic, the curious
reader is referred to the manual, both for \biblatex\manref{\S~3.1.1}
and for any package that is being used.

\index{backreferences|see{bibliography, references}}\index{bibliography!references}
Slightly simpler, however, than an index (which is a separate
document), it is possible also with standard styles to print, in the
bibliography, a list of the pages where the source in question was
referred to. To do this, you pass \biblatex\ the option
|backref=true|. There are various options\manref{\S~3.1.1} for
|backrefstyle| which will determine how back references are actually
printed (for instance, when references in consecutive pages are
compressed).

%%% Local Variables:
%%% coding: utf-8
%%% mode: LaTeX
%%% TeX-master: "biblatex-tutorial"
%%% End: 

\chapter{Multiple Bibliographies}\label{ch:subdivisions}
\index{bibliography!multiple!see multiple bibliographies}

We now come to consider the various methods of producing
\emph{multiple} bibliographies in a single TeX document. The detail
can become overwhelming unless you first understand the logic, so that
is where we will start.

There are two main patterns to split bibliographies:

\index{multiple bibliographies!general types}
\begin{enumerate}
\item Division that is determined by characteristics of the
  \emph{source}, where some types of source get put in one bibliography
  and other types of source get put in another. The dividing line
  might be between primary and secondary sources. It might be between
  works about foology and everything else (`topic based'). It might
  be between the most important works and the rest. It might be
  between the works of Dr Drofnats and everyone else's works. But
  however the dividing-line is drawn, it is based on something about
  the underlying source, which is why I'm going to call these
  \emph{source-based} divisions.
\item Division that is determined by the source's \emph{citation}, and
  in particular by \emph{where} the source is cited --- so that, for
  instance, there are separate bibliographies for Chapter 1, Chapter
  2, and so forth. In an `extreme form' the bibliographies need to
  be completely self-contained; that's often needed, for instance, if
  a single document is being used to typeset papers by different
  authors into a collection or journal issue. I'm going to call these
  \emph{document-based} divisions.
\end{enumerate}
These categories are not, of course, mutually exclusive. You
\emph{could} have, in a single document, both subdivision by source
(say into primary and secondary sources) and subdivision by location
in the document.

\biblatex\ provides mechanisms to handle both types of division. It is
designed to be comprehensive. Before \biblatex\ there were many
competing packages (\package{multibib}, \package{splitbib},
\package{chapterbib}, \package{bibunits}, \package{bibtopic}) which
provided these facilities with biblatex. These are \emph{not}
compatible with \biblatex; you cannot use them together --- and you
don't need to.

\section{Source-based division}

\indexstart{multiple bibliographies!source-based division}
In principle, source-based division is simple. You provide some way for
\biblatex\ to identify the different types of source, and then you print
separate bibliographies for each. Although the most interesting features
are those that determine how works are selected, it's convenient to start
by explaining how the actual printing is handled.
\index{multiple bibliographies!source-based division!printing}

\begin{pseudoverb}
\cs{printbibliography}[title="Primary Sources",
                       \angled{selector}]
\cs{printbibliography}[title="Secondary Literature",
                       \angled{selector}]
\end{pseudoverb}
Usually in these cases, however, the separate bibliographies are seen
as `sub-bibliographies', in which case a little bit more work is
done to get the formatting right:

\begin{pseudoverb}
\cs{printbibheading}
\cs{printbibliography} [heading=subbibliography,
                      title="Primary Sources",
                      filter=\angled{selector}]
\cs{printbibliography} [heading=subbibliography,
                      title="Secondary Literature",
                      filter=\angled{selector}]
\end{pseudoverb}
We use \cs{printbibheading} simply as a convenience to print a
standard `top level' heading for the whole bibliography, but then use
the |heading| and |title| options when actually printing the
bibliography. As far as \biblatex\ is concerned, a document can contain
as many \cs{printbibliography} commands as you
like, and each can contain as many or as few sources as you like.  The
real trick lies in how you teach \biblatex\ to distinguish between
different sources. In all cases, the method used is to add an option
to the \cs{printbibliography} command which tells
\biblatex\ what filter to use.\csindex{printbibliography}

To simplify somewhat there are three basic ways you can do that, and a
convenient way of combining them. The basic ways involve:
\emph{type}, \emph{keywords} and \emph{categories}. The
combining device is the \emph{filter}, which enables complex
assemblies of conditions to be defined and used simply.

\index{multiple bibliographies!source-based division!selection}
\paragraph{How \biblatex\ chooses.} All these distinctions are brought
into operation using an option in the bibliography heading. These
follow a standard pattern:
\begin{quote}
\angled{selector} \texttt{=} \angled{value}
\end{quote}
(e.g. \texttt{type=book}) to include only books.

If you prefer to look on the negative side:
\begin{quote}
\texttt{not}\angled{selector} \texttt{=} \angled{value}
\end{quote}
(e.g. \texttt{nottype=online}): exclude from the bibliography entries of
this type). You can have multiple `excluders', which are applied
cumulatively: \texttt{{[}nottype=online,} \texttt{nottype=misc{]}} will
\emph{exclude} both \texttt{online} and \texttt{misc} entry types.

\index{multiple bibliographies!source-based division!by type}
\newthought{The easiest type of choice} to build in to a bibliography
is between different entry types. Every source necessarily has an
entry type, and this can be a convenient way of producing some kinds
of subdivided bibliography. For instance, if you wanted to distinguish
between printed sources (the bibliography) and online sources (the
`webography') it might be sufficient to distinguish between. Here you
can use:
\begin{itemize}
\item \texttt{type}, to include (only) particular entrytypes, e.g.
  \texttt{type=book} to include only books. You can only have
  \emph{one} positive type selector. If you want to include, say
  |book| and |article| types, you will need to define a |bibfilter|
  for that purpose;
\item \texttt{nottype}, to exclude particular entrytypes,
  e.g. \texttt{nottype=book} to exclude books. You can always have
  \emph{more than one} excluder, without the need to define a
  |bibfilter|.
\item \texttt{subtype}, to include particular entrysubtypes (some
  bibliographic styles make use of subtypes to further divide types,
  or to specialised types such as \texttt{misc})
\item \texttt{notsubtype}, to exclude particular entrysubtypes.
\end{itemize}
(The reason you can have multiple excluders, but not multiple
includers, is that generally \biblatex\ treats conditions as
cumulative. So (|nottype=book| and |nottype=article|) works to
identify sources which are neither books nor articles. But
(|type=book| and |type=article| `work' to identify sources which are
both books and articles. Inevitably, nothing matches.)

So, for example, we could build a separate `webography' and
`bibliography' along these lines:

\begin{Verbatim}
\printbibheading
\printbibliography[heading=subbibliography,
                   title="Webography",
                   type=online]
\printbibliography[heading=subbibliography,
                   title="Paperography",
                   nottype=online]
\end{Verbatim}

\newthought{However, type of source is often not sufficient}: many
potentially interesting divisions (such as between primary and
secondary sources, or by subject) cut across source types. The trouble
here is that, for obvious reasons, \biblatex\ is not going to know
about the underlying subject-matter of a source or whether it is
primary or secondary and so forth unless you tell it. There are two
ways to do this.

\index{multiple bibliographies!source-based division!keywords}
First, you can store the information \emph{in the .bib file} using the
  \emph{keywords} field,\index{database!keyword field@\texttt{keyword} field} and then filter inclusion in a particular
  bibliography using the keyword and notkeyword instructions. So, for
  instance, you might mark every piece of primary literature using the
  keyword `primary' in the \texttt{.bib} file. You could then
  distinguish between primary literature and secondary sources as
  follows:
\begin{Verbatim}
  \printbibliography[title = "Primary Sources",
                     keyword = primary]
  \printbibliography[title = "Secondary Material",
                     notkeyword = primary]
\end{Verbatim}

\index{multiple bibliographies!source-based division!categories}
That, however, is not very flexible.\manref{\S~3.12.4} The alternative approach allows
document-by-document flexibility. You can enter the information in the
\emph{document source (}|.tex|\emph{) file} using the
\emph{categories} system. To do this:
\begin{itemize}
\item You first define a category in the preamble of the document, and
  then use a command to identify which sources are within a particular
  category. For instance, suppose you wanted to print `crucial' works
  separately in some notes for your students, you would create that
  crucial category in your preamble:\csindex{DeclareBibliographyCategory}
  \begin{pseudoverb}
    \centering
    \cs{DeclareBibliographyCategory}\{crucial\}
  \end{pseudoverb}

\item You then add individual sources to this category anywhere in the
  document using \cs{addtocategory}, which takes two arguments: the
  category and a list of sources to be added to that category.\csindex{addtocategory}

  \begin{pseudoverb}
    \centering
    \cs{addtocategory}\{crucial\}\{\angled{key1}, ... \angled{key2}\}
  \end{pseudoverb}

You can have multiple keys and multiple
\texttt{\textbackslash{}addtocategory\{\}\{\}} commands as well.

\item Then, when it comes to printing the bibliography, you control what is
printed using \texttt{category=\angled{category}} and
\texttt{notcategory=\angled{category}}.
\begin{Verbatim}
\printbibliography[heading = "Key sources",
                   category = crucial]
\printbibliography[heading = "Other sources",
                   notcategory = crucial]
\end{Verbatim}
\end{itemize}

Both keywords and categories are ways of arbitrarily allocating
sources to groups. The difference between them is that keywords are
defined in the database, and categories in the document where the
database is used. This should give you a hint about the best way of
using them. A keyword should really say something permanently true
about a source: they are a pretty reasonable candidate for dividing
sources into primary and secondary, or original and translated, or
identifying their topic as foology rather than barography. Categories
need only be true for the particular document: they are a good
candidate for specifying a source as `important' in a set of student
notes for instance, or for quick-and-dirty work where you don't want
to alter a database file permanently.

\subsection{Combining criteria with user-defined filters}

\index{multiple bibliographies!source-based division!filters}
Suppose you want to combine filters in complicated ways: for instance
you want a sub-bibliography which contains only crucial primary
sources which are not online. For such purposes, one reaches for a
user-defined filter. Such a filter is defined using
\begin{pseudoverb}
  \centering
  \cs{defbibfilter}\{\angled{name}\}\{\angled{definition}\}
\end{pseudoverb}
\csindex{defbibfilter}
This enables you to combine the type, subtype, keyword and category
filters using logical operators (and, or, not) and parentheses for
grouping. Having defined the filter, it can then be used to construct
a bibliography with the option |filtername=|\angled{filter}

So, for example, we could define the selection filter mentioned above
as follows:

\begin{Verbatim}
\defbibfilter{croffprim}{
    keyword=primary
    and category=crucial
    and not type=online}
\end{Verbatim}
And we could the use it:

\begin{Verbatim}
\printbibheading
. . .
\printbibliography[heading=subbibliography,
                   title="Crucial offline primary sources",
                   filter=croffprim]
\end{Verbatim}

\subsection{Automation}

\index{multiple bibliographies!source-based division!automation}
Both keyword and category filters ultimately depend on direct user
intervention, key-by-key. Is there any way to automate this? For
instance, could you automatically assign (say) all the sources in a
particular \texttt{.bib} file to a given category? Or could could
assign all the works of a particular author?

The answer is that you often can do so. But it's rather an advanced
topic. Besides, the range of possible uses and circumstances is very
wide. This section, therefore, is going to sketch some ideas, rather
than provide a set of recipes.

\begin{itemize}
\item Define a low-level user-defined filter using
  \cs{defbibcheck},\csindex{defbibcheck} which then uses the
  \texttt{check=\angled{filter}} filter. For instance the \biblatex\
  manual\manref{\S~3.7.9} demonstrates using such a mechanism to
  examine the year of an entry and print a bibliography of `recent'
  literature.
\item Use\manref{\S~4.5.3} a |sourcemap| to add keywords automatically. This could easily be done, say,
  for a particular file, or for a particular author's work; you could
  then use a keyword filter.
\item Use\manref{\S~4.10.6} the
  \texttt{\textbackslash{}AtEveryCitekey} hook to execute code at
  every citation to check data and decide whether to add that key to a
  category.\csindex{AtEveryCitekey}
\item Use\manref{\S~4.10.6} the \texttt{\textbackslash{}AtDataInput}
  to execute code as each entry is read in and decide whether to add
  it to a category.
\end{itemize}
\indexstop{multiple bibliographies!source-based division}

\section{Document-based division}
\indexstart{multiple bibliographies!document-based division}

\biblatex\ offers two main devices for producing bibliographies that are
categorised based on where the citation is used in the LaTeX source
file:

\index{refsection!general description}
\index{refsegment!general description}
\begin{itemize}
\item A \emph{refsection} is (nearly) a fully self-contained `unit'
  which might have its own |.bib| files, and is expected to produce a
  basically self-contained bibliography. It is the obvious choice, for
  instance, if producing a collection of papers, each of which had a
  different author who had used his or her own \texttt{.bib} file,
  with its own keys.
\item A \emph{refsegment} is a less strongly-marked division, which
  represents a part of what is recognisably a larger work: a way of
  dividing a \emph{single} bibliography up for convenience. It might
  be the right choice if you wanted to produce both summary
  bibliographies by chapter and a complete bibliography for the whole
  document.
\end{itemize}
The conceptual difference between the two is best understood if you
imagine a bibliography style which uses numerical labels. If the same
work is cited in two different \emph{refsections} then it will
expected to appear in two different bibliographies with a
\emph{different} number: each bibliography is entirely self-contained,
and the labels are unique within each but not between each. If, on the
other hand, the same work is cited in two different \emph{refsegments}
then it will have a single, unique, label which will be common to
both.  The bibliography for each segment may print all and only the
sources referred to in that segment; but although a source that is
referred to in two different segments will be printed in both
segments' bibliographies, it will have the \emph{same} label in each.

Another key difference is that whereas \emph{refsegments} share the
same |.bib| files, \emph{refsections} need not.  Some resources can be
defined so as to be global, but individual refsections can also have
their own `private' resources. Why does this matter? Suppose that Dr
Drofnats and Dr Frobwangler each writes a paper for a collection. In
their field, the seminal work is Mangel-Wurzel's famous paper, `How to
Foo Bars'. Dr Drofnats has this paper in his personal |.bib| file as
`|mw:foo|'. Dr Frobwangler has it in her personal |.bib| file as
`|wurzel:bars|'. Each cites accordingly.

One possibility of course is to produce a unified |.bib| file for the
whole document, and edit the citations in each paper. And sometimes
that might be necessary. But if the bibliographies are \emph{truly}
self-contained and free-standing, it should be possible to use Dr
Drofnats' file for his paper, and Dr Frobwangler's file for hers,
without the trouble and risk of error that there is if files are
combined.

\subsection{Identifying segments and sections}

\index{multiple bibliographies!document-based division!refsegment}
\index{multiple bibliographies!document-based division!refsection}
Obviously, \biblatex\ needs to know where each segment or section begins
and ends. There are three ways you can do this:

\index{refsection environment@\texttt{refsection} environment}
\index{refsegment environment@\texttt{refsegment} environment}
\begin{enumerate}
\item
  \begin{marginfigure}
  \fbox{
  \begin{minipage}{0.9\marginparwidth}
  \texttt{\textbackslash begin\{refsection\}}\\
  \colorbox{red!50}{[section 1]} \\
  \texttt{\textbackslash end\{refsection\}} \\
  \texttt{\textbackslash begin\{refsection\}}\\
  \colorbox{blue!50}{[section 2]} \\
  \texttt{\textbackslash end\{refsection\}}
  \end{minipage}}
  \end{marginfigure}
  Most explicitly using
  |\begin{refsection}| \ldots |\end{refsection}|
  or
  |\begin{refsegment}| \ldots |\end{refsegment}|
.
\item
  \csindex{newrefsegment}\csindex{newrefsection}
  \begin{marginfigure}
  \fbox{
  \begin{minipage}{0.9\marginparwidth}
  \texttt{\textbackslash newrefsection}\\
  \colorbox{red!50}{[section 1]} \\
  \texttt{\textbackslash newrefsection}\\
  \colorbox{blue!50}{[section 2]} \\
  \texttt{\textbackslash endrefsection}
  \end{minipage}}
  \end{marginfigure}
  Using the commands |\newrefsection| and |\newrefsegment|. Each of
  these ends the existing refsegment or refsection (if need be) and
  starts a new one. You can mark the end of refsegments or refsections
  with |\endrefsegment| or |\endrefsection|.
\item By telling \biblatex\ automatically to start a new refsegment or
  refsection whenever a new part, or chapter, or section begins. To do
  this you pass the option |part|, |chapter|, |section| or |subsection| as a
  value for the refsection or refsegment option when loading \biblatex.
  So to make each chapter a free-standing section, you would put
  \begin{marginfigure}
  \fbox{
  \begin{minipage}{0.9\marginparwidth}
  \texttt{\textbackslash chapter\{One\}}\\
  \colorbox{red!50}{[section 1]} \\
  \texttt{\textbackslash chapter\{Two\}}\\
  \colorbox{blue!50}{[section 2]} \\
  \texttt{\textbackslash endrefsection}
  \end{minipage}}
  \end{marginfigure}
\begin{Verbatim}
  \usepackage[...
              refsection=chapter]{biblatex}
\end{Verbatim}
\end{enumerate}

\subsection{Identifying resources for refsections}

\index{refsection!database}\index{refsegment!database}
Refsegments share the bibliography resources (|.bib| files) of the
entire document. Refsections can have their own resources. The full
details are rather complex. What follows is a simplified set of
instructions.

\begin{enumerate}
\item If you want one bibliography database or databases --- whether
  that be undivided, or divided into segments, or divided into
  sections but with the same |.bib| files used in all sections --- then
  just use |\addbibresource{}| in the preamble.
  \csindex{addbibresource}
\item If you are using refsections and you want each to have its
  \emph{own} |.bib| file(s), then proceed as follows:
  \begin{itemize}
  \item In the preamble, use |\addbibresource{}| to identify each
    resource, but make use of the optional label argument to give each
    a label:
\begin{Verbatim}
  \addbibresource[label=drofnats]{./bibfiles/drofnats.bib}
  \addbibresource[label=frobwangler]{./bibfiles/frobwangler.bib}
\end{Verbatim}
\item Use explicit refsection commands.\footnote{Either
    \cs[refsection]{begin} and \cs[refsection]{end} or
    \cs{newrefsection}.} As each section begins, include a list of the
  labels of the applicable resources as an optional argument:
\begin{Verbatim}
  \newrefsection[drofnats]
  ... 
  \newrefsection[frobwangler]
\end{Verbatim}
\item If there are |.bib| files that you want to have available in all
  sections then either identify them as shown above
\begin{Verbatim}
  \addbibresource[label=everyone]{./bibfiles/common.bib}
  ...
  \newrefsection[drofnats,everyone]
\end{Verbatim}
or, instead of \cs[\angled{filename}]{addbibresource} use
\cs[\angled{filename}]{addglobalbib} in the preamble: this will
declare a resource which is automatically made available to every
refsection. So
\begin{Verbatim}
  \addglobalbib{./bibfiles/common.bib}
\end{Verbatim} 
will make |common.bib| accessible to every refsection, without any
need to include it in the optional argument.\csindex{addglobalbib}
\end{itemize}
\end{enumerate}
\indexstop{multiple bibliographies!document-based division}

\subsection{Printing the bibliographies}

To print a bibliography for a single refsection, you do one of two
things. First, you can explicitly identify the section for which you
want the bibliography printed. You do this using the section's
\emph{number}:

\begin{Verbatim}
\printbibliography[section=1]
\end{Verbatim}

What is this number? Basically, all the sections you create (with
|\begin{refsection}...\end{refsection}|, or |\newrefsection|, or
implicitly at the start of chapters and so on) are numbered, starting
with 1. (Anything that happens \emph{before} the first section or
\emph{after} the last is regarded as belonging in a section numbered
0.) So you can specify the section whose bibliography you want
printed.

You can also do this for refsegments too in just the same way: the
first segment you create is refsegment 1, the next is 2 and so on --
and you can choose which gets printed:

\begin{Verbatim}
\printbibliography[segment=2]
\end{Verbatim}

With \emph{refsections only} (\emph{not} with segments) there is
another way. If you include a |\printbibliography| command
\emph{within} a refsection, \biblatex\ will assume that you only want
to include that section.

\begin{figure*}
\begin{minipage}[b]{0.5\textwidth}
\ttfamily
\cs[book]{documentclass}\\
\cs[biblatex]{usepackage}\\
\cs{addbibresource}[label=chapter1]\tikz{\node(labelnode)[shape=coordinate]{};}\{chaptera.bib\}\\
\cs{addbibresource}[label=chapter2]\{chapterb.bib\}\\

\cs[...]{chapter}\\ 
\colorbox{green!15}{\parbox{0.9\textwidth}{
\cs{newrefsection}\,\tikz{\node(refsectnode)[shape=coordinate]{};}[chapter1]\\[2ex]
... [lots of citations] ...\\[2ex]

\cs{printbibliography}\tikz{\node(pbnode)[shape=coordinate]{};}[heading=subbibliography]\\[2ex]
}}
\cs[...]{chapter}\\ 
\cs{newrefsection}[chapter2]\\[2ex]

... [lots of citations] ...\\[2ex]

\cs{printbibliography}[heading=subbibliography]
\end{minipage}
\begin{minipage}[b]{0.5\textwidth}
\sffamily
\tikz{\node(resnode) [text width=5cm] {resources are labelled: here, \texttt{chapter1} refers to \texttt{chaptera.bib}};}\strut\\[4ex]

\tikz{\node(exprefnode) [text width=5cm] {a refsection: within this section the resource indicated by the label is used --- so, here, citations are taken from \texttt{chaptera.bib}};}\strut\\[2ex]

\tikz{\node(exppbnode) [text width=5cm] {only the bibliography for this refsection will be printed};}

\vspace{9ex}
\end{minipage}
\begin{tikzpicture}[overlay,line width=1pt]
\draw[red] (resnode.south) -| (labelnode.north);
\draw[red] (exprefnode.north west) -| (refsectnode.east);
\draw[red] (exppbnode.north) -| (pbnode.south);
\end{tikzpicture}
\caption{Exemplary refsections\label{refsections}}
\end{figure*}

Finally, because a possible use case is to loop through each
section or segment, printing the bibliography for that section or
segment, and then doing the same for the next, there are special
commands designed to achieve this: |\bibbysection| and
|\bibbysegment|. They are `smart' in that they will not print a
bibliography if a particular segment or section contains no citations
at all.\csindex{bibbysection}\csindex{bibbysegment}

If you want \emph{both} subdivided bibliographies and a main,
consolidated, bibliography, there are two approaches. (1) You can use
|refsegments|. If you then print a bibliography with \emph{no}
|segment=...| selector, all segments will be printed. (2) If
\emph{only} your cited sources are in the |.bib| file, you can use a
|\nocite{*}| instruction within a |section| to print
everything. Unfortunately there is no way to tell \biblatex\ to print
a combined bibliography of multiple sections.

\subsection{Some advice}

\index{multiple bibliographies!considered harmful}
In many cases, excessively divided bibliographies are more harmful
than helpful. Before deciding to divide bibliographies at all, do
think seriously about how a reader uses a bibliography. Don't
introduce unnecessary complexity. Apart from the most obvious case
(self-contained papers which need their own bibliographies, for
instance) subdivision can easily be over done.

%%% Local Variables:
%%% coding: utf-8
%%% mode: LaTeX
%%% TeX-master: "biblatex-tutorial"
%%% End:

\chapter{Sorting}\label{ch:sorting}

\indexstart{sorting!general}
A bibliography style will usually define a sorting scheme which is
appropriate for that style. For instance, consider the following works:

\medskip
\colorbox{red!40}%
    {Smith, J. (2000) Zoology for the Amateur. Oxbridge: Pubco.}\\
\colorbox{green!40}%
    {Smith, J. (1999) Professional Zoology. Yarvard: Aldus.}\\
\colorbox{blue!40}%
    {Smith, J. (2010) Amateur Zoology. Camford: Otherco.}
\medskip

In an author/year system, it would make sense to put the works in that
order. On the other hand, in an author/title system, it would probably
make sense to list them as

\medskip
\colorbox{blue!40}%
    {Smith, J. \emph{Amateur Zoology}. Camford: Otherco 2010.}\\
\colorbox{green!40}%
    {Smith, J. \emph{Professional Zoology}. Yarvard: Aldus 1999.}\\
\colorbox{red!40}%
    {Smith, J. \emph{Zoology for the Amateur}. Oxbridge: Pubco 2000.}
\medskip

In an alphanumeric system, the correct order will normally be one
that make sense of the labels.

\medskip
\colorbox{red!40}
    {{[}Smi00{]} Smith J. \emph{Zoology for the Amateur} \ldots{}}\\
\colorbox{blue!40}
    {{[}Smi10{]} Smith J. \emph{Amateur Zoology} \ldots{}}\\
\colorbox{green!40}
    {{[}Smi99{]} Smith J. \emph{Professional Zoology} \ldots{}}
\medskip

A numeric system may either sort its bibliography in some way (usually
along the lines of the author/title system) or print the bibliography in
the order of citation in the text -- a system that is conveniently and
inaccurately described as involving an \emph{unsorted} bibliography, and
which one achieves by loading \biblatex\ with the option
\texttt{sorting=none}.

\biblatex\ allows for a wide variety of sorting schemes, and there is
considerable flexibility to define new schemes. But that largely lies
outside the scope of this book, which is aimed at the ordinary user. So
what we are going to do is to describe the common schemes, explain how
you can make simple and reasonable changes to them, and pay a bit of
attention to some special fields in the \texttt{.bib} file that can be
used to influence sorting.
\indexstop{sorting!general}

\section{The built-in schemes}

\indexstart{sorting!predefined schemes}
The following are the basic schemes,\manref{\S\S\ 3.1.2.1, 3.5} and a
basic definition of what they are intended to achieve. (Note that
individual bibliography styles may produce their own schemes.)

\begin{itemize}
\item
  \emph{Unsorted.} (\texttt{sorting=none}) --- a misnomer, really, in so far as
  it suggests that the bibliography might appear in some sort of random
  order. The bibliography will be printed in the order the works are
  first cited in the text.\index{sorting!citation order}\index{sorting!none}
\item
  \emph{Name/Title/Year} (\texttt{sorting=nty}). Works are first sorted by the
  name of the author (or editor), so all works by Albert Aardvark appear
  before anything written by Benjamin Badger. Within each name, the
  works are sorted by title. And if titles are identical, then the
  earlier in time is placed first.\index{sorting!name/title/year}
  \begin{marginfigure}[-20ex]
  \fbox{
  \begin{minipage}{0.95\marginparwidth}
    \colorbox{red!50}{\strut Aardvark, A.} My life. 1999.\\
%
    \colorbox{red!30}{\strut Badger, B.}%
    \colorbox{green!60}{\strut Memoirs.}%
    \colorbox{blue!50}{\strut 2005.}\\
%
    \colorbox{red!30}{\strut Badger, B.}%
    \colorbox{green!60}{\strut Memoirs}(2nd ed).%
    \colorbox{blue!30}{\strut 2010.}\\
%
    \colorbox{red!30}{\strut Badger, B.}%
    \colorbox{green!40}{\strut Memories.} 1999.\\
%
    \colorbox{red!30}{\strut Badger, B.}%
    \colorbox{green!30}{\strut Recollection.} 2005.
  \end{minipage}}
  \vspace{3pt}
  \caption{\texttt{nty} sorting}
  \end{marginfigure}
\item \emph{Name/Year/Title} (\texttt{sorting=nyt} or \texttt{sorting=nyvt}).
  Works are first sorted by the name of the author (or editor), so
  that all works by Albert Aardvark appear before anything written by
  Benjamin Badger. Within each name, the works are sorted first by
  year and then, if there is more than one work in any given year,
  alphabetically by title. Name/Year/Volume/Title
  (\texttt{sorting=nyvt}) is the same as Name/Year/Title, except that
  the volume is considered before title.\index{sorting!name/year/title}
\begin{marginfigure}[-15ex]
  \fbox{
   \begin{minipage}{0.95\marginparwidth}
     \colorbox{red!50}{\strut Aardvark, A.} (1999) My life.\\
%
     \colorbox{red!30}{\strut Badger, B.}%
     \colorbox{green!60}{\strut(1999)} Memories.\\
%
     \colorbox{red!30}{\strut Badger, B.}%
     \colorbox{green!40}{\strut(2005)}%
     \colorbox{blue!50}{Memoirs.}\\
%
     \colorbox{red!30}{\strut Badger, B.}%
     \colorbox{green!40}{\strut(2005)}%
     \colorbox{blue!30}{Recollection.}\\
%
     \colorbox{red!30}{\strut Badger, B.}%
     \colorbox{green!20}{\strut(2010)} Memoirs (2nd ed)
   \end{minipage}}
  \vspace{3pt}
  \caption{\texttt{nyt} sorting}
  \end{marginfigure}
\item
  \emph{Year/Name/Title} (\texttt{sorting=ynt} or \texttt{sorting=ydnt}). Sorts
  by the year first, then the name, then the title (can be useful for
  producing a chronologically organized bibliography). The difference
  between \texttt{ynt} and \texttt{ydng} is that \texttt{ynt} works
  upwards (2013 comes after 2000), whereas \texttt{ydnt} works downwards
  (2000 comes after 2013).\index{sorting!year/name/title}
  \begin{marginfigure}[1ex]
  \fbox{
  \begin{minipage}{0.95\marginparwidth}
      \colorbox{red!50}{\strut 1999.}%
      \colorbox{green!50}{\strut Aardvark, A.}My Life.\\
%
      \colorbox{red!50}{\strut1999.}%
      \colorbox{green!30}{\strut Badger, B.}Memories.\\
%
      \colorbox{red!40}{\strut 2001.}Badger, B. Memoirs.\\
%
      \colorbox{red!30}{\strut 2002.}Badger, B. Recollection.\\
%
      \colorbox{red!20}{\strut 2003.}Badger, B. Memoirs (2nd ed).
    \end{minipage}}
  \vspace{3pt}
\caption{\texttt{ynt} sorting}
\end{marginfigure}
\item
  \emph{Alphabetic Label/Name/Year/Title} (\texttt{sorting=anyt}). Sorts
  principally by the alphabetic label --- and obviously therefore
  intended only for alphabetic styles. The \biblatex~manual says that it
  then sorts by name, year and title: but so long as the labels are
  unique, as would usually be the case, these will never need to be
  consulted. There is also a style \texttt{anyvt} which
  considers volume information.\index{sorting!alphanumeric labels}
 \begin{marginfigure}[1ex]
  \fbox{
  \begin{minipage}{0.95\marginparwidth}
    \colorbox{red!50}{\strut[Aar00]}A.\,Aardvark.\,My Life.\.2000. \\
    \colorbox{red!40}{\strut[Bad05a]}B.\,Badger.\,Memoirs.\,2005.\\
    \colorbox{red!30}{\strut[Bad05b]}B.\,Badger.\,Recollection.\,2005. \\
    \colorbox{red!20}{\strut[Bad10]}B.\,Badger.\,Memoirs.\,2010. \\
    \colorbox{red!10}{\strut[Bad99]}B.\,Badger.\,Memories.\,1999.
  \end{minipage}}
\vspace{3pt}
  \caption{\texttt{anyt} sorting}
  \end{marginfigure}
\item \emph{By order in the \texttt{.bib} file}
  (\texttt{sorting=debug}) this order citations by their \emph{key},
  and is (as its name suggests) exclusively intended for styles which
  are used for debugging \texttt{.bib} files.\index{sorting!debug}\index{sorting!in .bib file order@in \texttt{.bib} file order}
\end{itemize}
\indexstop{sorting!predefined schemes}

\section{Ad hoc manipulations in the \texttt{.bib} file.}

\index{sorting!special cases}
Most of the time, \biblatex\ and \package{biber} will sort quite well,
but there are occasions when you may need to intervene.

\index{sorting!sortname@texttt{sortname}}\index{database!sortname
field@\texttt{sortname} field}
\paragraph{Helping out the sorting}\label{sorting:sortname} The first,
and most common, is when for some reason the name field that should be
printed is inappropriate for sorting, and you need to specify a
slightly different version for sorting purposes only. This can happen
for two main reasons.

You may just want a different name. For instance, suppose you have an
`institutional' author:
\begin{Verbatim}
author = {{The Magoo Trust}}
\end{Verbatim}
\biblatex\ is going to try to sort this under T for `The' --- but you
might think it better to have it sorted under M. In such a case, you
can specify a \texttt{sortname}
\begin{Verbatim}
sortname = {{Magoo Trust The}}
\end{Verbatim}

\index{sorting!sorttitle@texttt{sorttitle}}\index{database!sorttitle
field@\texttt{sorttitle} field}
Similar things can happen with titles (indeed, it's more common
there)\label{sorting:sorttitle}
\begin{Verbatim}
title      = {The General Principles of EC Law}
sortitle   = {General Principles of EC Law The}
\end{Verbatim}

\index{database!latex in@\LaTeX\ in}
If you have used a \LaTeX\ command in a field this may confuse the
sorting, and you can use an `unvarnished' version for sorting
\begin{Verbatim}
title    = {\TeX\}ing,
sortitle = {Texing}
\end{Verbatim}

You can set \texttt{sortname}, \texttt{sorttitle}, and \texttt{sortyear}
fields for these purposes.

\index{sorting!altering order}
\paragraph{Fiddling with the order} The second change you might sometimes
want to make is something which more drastically manipulates the order.

Take, for instance a book with no author or one with no date. In a
name/title/year system, the authorless book gravitates to the top of
the list, and in a year/name/title system, the dateless work ends up
at the end of the list. Suppose you want the reverse?

\index{database!presort field@\texttt{presort} field}\index{sorting!pre-sorting}
The trick is this. Every entry in your \texttt{.bib} file is assumed
to have a field called \texttt{presort} magically set to
\texttt{mm}. And, as the name suggests, that is the first field that
gets sorted. So if you set \texttt{presort} to something higher in the
alphabet than \texttt{mm} (like, say, \texttt{aa}) the work in
question will magically appear (alongside everything else with the
same presort code) above the rest of the list; and if you set it to
something lower in the alphabet than \texttt{mm} (like, say,
\texttt{zz}) it will drop to the bottom.

The \texttt{presort} field could be used --- and in the past sometimes
was --- for other purposes, and in particular for producing
topic-based bibliographies (for instance by giving all primary sources
a presort of \texttt{aa}, to move them to the top of the list). With
\biblatex\ there are better ways of achieving that sort of
result.

\index{database!sortkey@\texttt{sortkey}}\index{sorting!sortkey as last resort@\texttt{sortkey} as last resort}
In desperation you can also use \texttt{sortkey} to fix absolutely and
unequivocally the `key' by which a work will be sorted. Generally
speaking, though, this is a counsel of desperation, and it is hard to
think of a real-life situation in which it would be advised.

%%% Local Variables:
%%% coding: utf-8
%%% mode: LaTeX
%%% TeX-master: "biblatex-tutorial"
%%% End:

\chapter{Languages}\label{ch:languages}

Dealing with different languages is not straightforward; but
\biblatex\ is fairly well set up for the task. This section aims to
provide basic information; there is actually considerably more on
offer in \biblatex\ for sophisticated language use than this section
deals with. This section basically assumes the `normal' user who
produces documents in a single language, but may occasionally need to
deal with bibliographic sources in other languages.

\index{languages!possible changes}
There are a number of different ways that a bibliographic system may
need to adjust to different languages. Most basically, a number of
words and abbreviations may need to change: `ed' in English becomes
`éd' in French, or `Hrsg' in German. There may be differences in
punctuation (for instance in how quotations are typeset). There may be
differences in sorting. (For instance, is `Aardvark' a word you would
expect to appear at the beginning or the end of a dictionary? An
English person and a Swede would disagree!) At a still more basic
level -- unseen by the user until it goes wrong -- the hyphenation of
different languages varies. And, even further `under the skin' there
are issues about the `encoding' of an input file: how human-readable
glyphs are turned into machine-readable sequences of numbers.

\section{A dogmatic approach}

\index{languages!recommendations}
Later sections of this chapter try to provide a better explanation of
how all these various things can be done. But here is a dogmatic `just
do this' approach, which will work in many cases.

\begin{enumerate}
\item If you are writing in a language which requires accents and
  diacritics, prefer to use \smallcaps{utf-8} encoding in your source
  and |.bib| files. Set your editor to use that encoding and, unless
  you are using Xe\TeX\ or Lua\TeX\ (which handle unicode natively),
  use the \package{inputenc} package in your source file.
 \begin{pseudoverb}
 \cs{usepackage}[utf8]\{inputenc\}
 \end{pseudoverb}
\item Load \package{babel}, or, if you use it,
    \package{polyglossia}, with your preferred
  language.\marginnote{\newcommand{\english}{{\normalfont
        (English)}}The language options are: \ttfamily american
    \english, australian \english, austrian, brazil, british \english,
    canadian \english, canadien {\normalfont(French)}, catalan, czech,
    danish, dutch, finnish, french, german, greek, italian, naustrian,
    ngerman, norsk, nynorsk, portugues, russian, spanish, swedish.}
 \begin{pseudoverb}
   \cs{usepackage}[\angled{language}]\{babel\}
 \end{pseudoverb}
 This should be done, and any language selection made, \emph{before} loading \biblatex.
\item Load \package{csquotes} with an appropriate quotation style:
 \begin{pseudoverb}
   \cs{usepackage}[style=\angled{language}]\{csquotes\}
 \end{pseudoverb}
\item Load \biblatex\ with the style |autolang=hyphen|, which will
  make sure that appropriate hyphenation patterns are used based on
  the |langid| field of the entry, if it has been set. Set that field
  where you think it will be helpful.
\item If you are using anything other than the standard bibliography
  style, consult that style's documentation to find out whether you
  need to `connect' your language to a particular style-specific
  language style. If so, do that using
 \begin{pseudoverb}
   \cs{DeclareLanguageMapping}\{\angled{language}\}\{\angled{language-definition}\}
 \end{pseudoverb}
 For instance, in the \package{APA} style, I might have
 \begin{pseudoverb}
   \cs{DeclareLanguageMapping}\{british\}\{english-apa\}
 \end{pseudoverb}
\end{enumerate}

\section{Some more detail}

\subsection{Encodings}

\index{encoding}
\biblatex\ (or, more accurately, \package{Biber}) is perfectly
happy working with unicode (indeed, internally this is what it always
tries to do). It is generally best, unless your use of accents and
diacritics is minimal, to use a \smallcaps{utf-8} encoded |.bib|
file. This need not be the encoding used in your |.tex| file: the
software will attempt to detect that, and will output a |.bbl| file
accordingly. But in general it makes sense, nowadays, to use a
\smallcaps{utf-8} encoded |.tex| source. For this, you should either
use Xe\TeX\ or Lua\TeX, which handle unicode natively, or use the
\package{inputenc} package to set the encoding to |utf8|.

Occasionally, notwithstanding such precautions, there can be
difficulties caused by the way \TeX\ and \package{Biber} deal with
encodings. In such cases it may be necessary to set input and output
encodings explicitly; on this, you should consult the \package{Biber}
documentation.

\subsection{Language}

In general, in the \LaTeX\ world, language features are controlled
using the (older) \package{babel} or (newer) \package{polyglossia}
packages to set languages. The \biblatex\ package generally tries to
leverage these packages to work out what language is being used. It
then attempts to load a file which contains suitable bibliographical
features in that language.

This is in many cases enough. But internally the position has to be a
bit more complex, because \biblatex\ has to associate a given basic
language with a set of files which define suitable abbreviations,
formatting of dates, and so forth for that language. In the standard
styles, this is entirely automatic. In non-standard styles, there may
be cases where the style is language-specific, or in which you are
otherwise required to tell \biblatex\ explicitly what language
definitions to use.

This is the purpose of the \cs{DeclareLanguageMapping} command. What
it does is tell \biblatex\ what definitions to use \emph{for a
  particular language}. So, for instance
\begin{pseudoverb}
  \cs{DeclareLanguageMapping}\{british\}\{british-apa\}
\end{pseudoverb}
tells \biblatex\ that if the language being used for typesetting is
|british|-English, then it should look for the relevant definitions in
the |british-apa| langugage definition. Sometimes this has been done
for you by the writer of the bibliography style; sometimes it needs to
be explicitly done (especially if the style provides a limited range
of language definitions).

\indexstart{languages!multiple}
It is perfectly permissible to have a document in more than one
language: the citation and bibliography commands will use whichever
language is active at the relevant time. So, for instance (to take a
thoroughly feeble example) suppose we had the following:

\begin{Verbatim}[frame=single]
...
\usepackage[french,british]{babel}
\usepackage[style=numeric]{biblatex}
\addbibresource{biblatex-examples.bib}

\begin{document}
We make reference to Aristotle \cite{aristotle:anima}.

\printbibliography

\selectlanguage{french}

\printbibliography
...
\end{Verbatim}

\begin{figure*}
\fbox{\includegraphics{./examples/duallanguage.pdf}}
\vspace{3pt}%
\caption{A document using two languages\label{dual:language}}
\end{figure*}

That code will produce something which looks like figure
\ref{dual:language}. The first time the bibliography is printed,
English conventions are used, because English is the active
language. Then French is selected as the language, and this time the
bibliography, when printed, is formatting according to French
conventions and terminology.

In that example, the language is being selected explicitly in the
source file. But what about the |.bib| entries themselves? Can they
select languages? The short answer is, Yes. But it's really quite
important to give a longer answer.

A |.bib| entry may have the field |language| set. But this will not
affect the language conventions used to typeset it. The purpose of the
|language| and |origlanguage| fields is to record
\emph{bibliographical} data, which may (depending on the style) be
printed.\sidenote[][-4ex]{For instance, a style might print:
  Aristotle, \emph{De Anima} (Greek).} So |language| and
|origlanguage| relate to the \emph{source} not the
\emph{entry}. However ther is a field
|langid|\footnote{\texttt{langid} was previously called
  \texttt{hyphenation}.} which is specifically designed for this
purpose. You set the |langid| field to indicate that the language of
the entry should (potentially) be fixed in a particular way. For
instance, if you are citing an author and a title in a foreign
language, it may be natural to specify an appropriate language ID:

\begin{verbatim}
@book{swann,
  author     = {Proust, Marcel},
  maintitle  = {À la recherche du temps perdu},
  volume     = 1,
  title      = {Du côté de chez Swann},
  date       = {1913},
  langid     = {french},
}
\end{verbatim}

\index{database!langid field@\texttt{langid} field}
However, simply specifying the language ID does not directly make any
difference: the entry simply gets printed in the ordinary way (see
figure \ref{proust1}).
\begin{figure}
\framebox{\includegraphics{./examples/proust1}}
\caption{The language ID: not necessarily any difference\label{proust1}}
\end{figure}

However, you can make it make a difference by using an option when
\biblatex\ is loaded:
\begin{itemize}
\item |autolang=hyphen| will activate hyphenation rules for the
  specified language. This may or may not make any difference
  (depending on line breaks and so forth). In our example it would
  not. But it's generally a good idea to allow hyphenation of titles
  and so forth correctly.
\item |autolang=other| will apply the \emph{full} set of rules for the
  language in question when printing that particular entry, as can be
  seen in figure \ref{proust2}. Note in this case that the
  bibliography in general is in English format (the title is
  `References' not `Références', and so forth), but the particular
  entry takes a French format.
\end{itemize}

\begin{figure}
\fbox{\includegraphics{./examples/proust2}}
\caption{\texttt{autolang=other}\label{proust2}}
\end{figure}

In general terms, it seems to me, activating language-specific
hyphenation is normally a sensible idea, but activating full language
support seems to me generally a rather peculiar thing to do in general
(though there might be some multilingual works where it makes sense).

There are further and deeper complexities for users of
\package{polyglossia}, for whom still more complex features are
available. But those are beyond the scope of this book.
\indexstop{languages!multiple}

\subsection{Sorting}

\index{sorting!language specific}
There is a tendency to consider alphabetical order, drilled into us as
children, as if it were fixed and immutable. But in fact alphabetical
order can vary from language to language. For instance, in Czech and
Slovak accented letters are conventionally sorted after unaccented
letters (so \'{E}mil comes after Emily), in Norwegian `Aa' is
considered the same as \r{A}, and placed at the end of the alphabet
(so Aardvark comes after Zebra!).

This sort of detail is under the control of \package{Biber}, not of
\biblatex. Normally you can assume that an appropriate locale will be
set based on environment variables in your computer. But you may
sometimes need to specify the locale. You can do this by specifying a
\texttt{sortlocale} as an option when loading \biblatex. So, for
instance, to have a Norwegian sorting scheme, you would specify
\begin{verbatim}
\usepackage[sortlocale=nb_NO]{biblatex}
\end{verbatim}

You should\footnote{Though at the time of writing \package{Biber}
  seems to be broken.} see the effect of this in figures
\ref{zebra:en} and \ref{zebra:no}. The bibliography used is the same
in each case, but figure \ref{zebra:en} is sorted using
\texttt{sortlocale=en\_GB}, whereas figure \ref{zebra:no} is sorted
using \texttt{sortlocale=nb\_NO}.

\begin{figure}
\fbox{\includegraphics{./examples/sorting-eg1.pdf}}
\caption{Sorted with \texttt{sortlocale=en\_GB}\label{zebra:en}}
\end{figure}

\begin{figure}
\fbox{\includegraphics{./examples/sorting-eg2.pdf}}
\caption{Sorted with \texttt{sortlocale=nb\_NO}\label{zebra:no}}
\end{figure}

\section{Limitations}
Although \biblatex\ has many features designed to deal with different
languages, there are still limits. The limits are, as things stand,
reached when different languages have fundamentally different ideas
about the sort of information that should be output, or the order in
which it shoudl be output. To cope with this it is necessary to carry
out deep changes to the bibliography drivers, and the existing styles
do not support this.

%%% Local Variables:
%%% coding: utf-8
%%% mode: LaTeX
%%% TeX-master: "biblatex-tutorial"
%%% End:


%\chapter{Recipes}
\chapter{实用技巧}\label{ch:recipes}

A comprehensive guide to customization is outside the scope of this
book. But some requests for customization are very common, and it
seems worthwhile to have a short chapter that provides some common
recipes.

%\section{Names}
\section{姓名相关的技巧}

\index{customization!names}
\paragraph{I want at most 1/2/3/4 names printed before `and
  others'.}\label{recipes:maxnames} You need to change the value of
|maxbibnames|, |maxcitenames|, |minbibnames| and/or |mincitenames|. As
you might expect |maxbibnames| and |minbibnames| deal with what is
printed in the bibliography, and |maxcitenames| and |mincitenames|
deal with what is printed in a citation (if citations print
names). You can change both with |maxnames| and |minnames|, which is
what you normally want to do, and what we will do here. You set this
value when loading \biblatex:
\begin{pseudoverb}
\cs{usepackage}[maxnames=2, minnames=1]\{biblatex\}
\end{pseudoverb}

The |maxnames| number holds the number of names that will
\emph{trigger} truncation to `and others' (or `et al', or
whatever). The |minnames| number is the number of names \emph{to which
  the list will be truncated if truncation occurs} (which must,
obviously, be no more than the number that triggers truncation).

Suppose we had an entry with four authors:
\begin{Verbatim}
...
author = { Alpha, A. and Beta, B.
           and Gamma, C. and Delta, D. }
...
\end{Verbatim}

The effect of various different settings of |maxnames| and |minnames|
is shown in table \ref{maxnames}.

\begin{table}
\begin{tabularx}{\textwidth}{llX}
\toprule
\texttt{maxnames} & \texttt{minnames} & \textsf{result} \\
\midrule
1                 &  1                & A.\ Alpha et al \\
2                 &  1                & A.\ Alpha et al \\
2                 &  2                & A.\ Alpha, B. Beta, at al \\
3                 &  1                & A.\ Alpha et al \\
3                 &  2                & A.\ Alpha, B. Beta, et al \\
3                 &  3                & A.\ Alpha, B. Beta, C. Gamma, et al \\
4                 &  any              & A.\ Alpha, B. Beta, C. Gamma, and D. Delta \\
\bottomrule
\end{tabularx}
\caption{Effect of \texttt{maxnames} and \texttt{minnames}\label{maxnames}}
\end{table}

Sometimes you will find that you get results which are not what you
are expecting. When it truncates a name, \biblatex\ can be configured
to consider whether the resulting list is unique. If it isn't,
\biblatex\ may add extra names in order to produce a unique list. For
example, suppose you are citing one work by Tom, Dick, and Harry; and
one work by Tom, Dick, and Harriet, and have set |maxnames| to 2 and
|minnames| to 1. Simply applying the ordinary rules, each work would
end up with the authors given as `Tom et al.' But in some styles it
may sometimes make sense to bend the rules if by doing so it is
possible to produce a `unique' citation. This is done by setting the
option |uniquelist=true|.\manref{3.6.1} In fact many standard styles
(such as author/year and author/title styles \emph{already do
  that}. So if you think that \biblatex\ is `disobeying' your
instructions about truncation, you might try adding the option
|uniquelist=false| when loading it.

\index{customization!et al@\emph{et al}}
\paragraph{I want something different from `et al' printed when names
  are truncated.} What gets printed in this case depends on the
setting of two bibliography strings, and one delimeter:
\begin{itemize}
\item The bibstring |andothers| (by default `et al')
\item The delimiter |andothersdelim| (by default ` ')
\item The |\finalandcomma| (printed wherever a list would end in
  `and')
\end{itemize}
So to print `\& al.' rather than `, et al', one would put
\begin{Verbatim}
\DefineBibliographyStrings{english}% or your language
  { andothers = {\& al\adddot}}
...
\renewcommand{\finalandcomma}{}
\end{Verbatim}
The redefinition of |\finalandcomma| needs to come \emph{after}
|\begin{document}|.

Things are a bit more complicated if you want to redefine |et al| in
such a way that it will be different depending on how truncated the
name is: for instance if you wanted `and others' if more than two
names were truncated, but `and another' if only one was. For that
purpose you would need the more extensive changes shown in figure
\ref{andothers}.
\begin{figure}
\begin{Verbatim}[frame=single]
\NewBibliographyString{andanother}
\DefineBibliographyStrings{english}% or your language
{ andothers = {and others},
  andanother = {and another}}
\renewbibmacro*{name:andothers}{%
  \ifboolexpr{
    test {\ifnumequal{\value{listcount}}{\value{liststop}}}
    and
    test \ifmorenames
  }
    {\ifnumgreater{\value{liststop}}{1}
       {\finalandcomma}
       {}%
     \ifnumgreater{\value{listtotal} - \value{liststop}}{1}
       {\andothersdelim\bibstring{andothers}}
       {\andothersdelim\bibstring{andanother}}
    }
    {}}
\end{Verbatim}
\caption{Introduction of `and others' and `and another'\label{andothers}}
\end{figure}

\index{customization!dashes in bibliography}\index{dash!in bibliography|see{customization, dashes}}
\paragraph{I want/don't want multiple references to a single author to
  appear as a long dash.} In the author/year, author/title and verbose
styles, you can control this with the option |dashed=true| or
|dashed=false| when loading \biblatex. Non-standard styles are not
obliged to provide this option, but most will do so.

\index{customization!initials}
\paragraph{I want to have all names abbreviated to initials.} Set the
option |giveninits=true| when loading \biblatex.

\index{customization!initials}
\paragraph{I want to have initials printed in a different way.} By
default, the standard styles of \biblatex\ print initials with spaces
and abbreviation dots:
\begin{quote}
  P.~M.~Stanley
\end{quote}
But other methods are available. If you set |terseinits=true| the initials are printed in a condensed way:
\begin{quote}
  PM~Stanley
\end{quote}
If you want a `halfway house', you can redefine |\bibinitperiod|. By combining |terseinits| with |\renewcommand{\bibinitperiod}{\adddot}| you can produce
\begin{quote}
  P.M. Stanley
\end{quote}
While by combining |terseinits=false| with |\renewcommand{\bibinitperiod}{}| you can produce
\begin{quote}
  P~M~Stanley
\end{quote}

There is, however, a catch. None of these changes will operate unless
\biblatex\ is producing the initials. In other words, none works
unless the |giveninits| option is in effect. Where names are being
printed in full, initials that you have entered will be printed (as
entered) with full stops and spaces, regardless of the settings of
|terseinits| or |\bibinitperiod|. (There are ways around this, but
they are beyond the scope of this section: the curious might look at
the \package{\textsc{oscola}} package code, which does something along these lines.)

\index{customization!names}
\paragraph{I want names printed in bold/italic/small capitals etc.}
How names are printed depends on four standard commands:
\cs{mkbibnamegiven} (which prints the initials or first names),
\cs{mkbibnameprefix} (which prints the `de' or `von' parts),
\cs{mkbibnamefamily} (which prints the last name) and
\cs{mkbibnamesuffix} (which prints the `Jr' part). Each or all of these
can be redefined to alter the way names are printed. Each takes a
single argument (the relevant part of the name), and can be used to
format it.\footnote{For formatting emphatically or in bold, you should
  use \cs{mkbibbold} or \cs{mkbibemph}}.

Suppose, for instance, that we had the following (peculiar)
requirements: We want first names printed in bold, we want `von' and
`Jr' parts printed emphatically, and we want the last name in small
capitals. We define the following
\begin{Verbatim}
\renewcommand{\mkbibnamegiven}[1]{\mkbibbold{#1}}
\renewcommand{\mkbibnameprefix}[1]{\mkbibemph{#1}}
\renewcommand{\mkbibnamesuffix}[1]{\mkbibemph{#1}}
\renewcommand{\mkbibnamefamily}[1]{\textsc{#1}}
\end{Verbatim}
And obtain results as shown in figure \ref{namestyle}.
\begin{figure}
\begin{minipage}[t]{3in}
 \tikz[thick]{
 \node(givenname)[fill=red!40, xshift=25ex]{\strut\texttt{Klaus}};
 \node(prefixname)[fill=blue!40, xshift=32ex]{\strut\texttt{von}};
 \node(familyname)[fill=green!40, xshift=39.5ex]{\strut\texttt{Bulow}};
 \node(suffixname)[fill=orange!40, xshift=46ex]{\strut\texttt{III}};
 \node(mkbibgiven)[fill=red!40,yshift=-10ex]{\cs[Klaus]{mkbibnamegiven}};
 \node(mkbibbold)[fill=red!40,yshift=-20ex]{\cs[Klaus]{mkbibbold}};
 \node(boldklaus)[fill=red!40,yshift=-30ex,xshift=25ex]{\strut\textbf{Klaus}};
 \node(mkbibprefix)[fill=blue!40, xshift=19ex, yshift=-15ex]{\cs[von]{mkbibprefix}};
 \node(mkbibemph1)[fill=blue!40, xshift=19ex, yshift=-25ex]{\cs[von]{mkbibemph}};
 \node(emvon)[fill=blue!40, xshift=32ex, yshift=-30ex]{\emph{\strut von}};
 \node(mkbibfamily)[fill=green!40, xshift=39.5ex, yshift=-10ex]{\cs[Bulow]{mkbibnamefamily}};
 \node(scname)[fill=green!40, xshift=39.5ex, yshift=-20ex]{\cs[Bulow]{textsc}};
 \node(scbulow)[fill=green!40, xshift=39.5ex, yshift=-30ex]{\strut\textsc{Bulow}};
 \node(mkbibsuffix)[fill=orange!40, xshift=55ex, yshift=-15ex]{\cs[III]{mkbibsuffix}};
 \node(mkbibemph2)[fill=orange!40, xshift=55ex, yshift=-25ex]{\cs[III]{mkbibemph}};
 \node(emiii)[fill=orange!40, xshift=46ex, yshift=-30ex]{\strut\emph{III}};
 \path[-stealth] (givenname.west) edge [out=180, in=90] (mkbibgiven.north);
 \path[-stealth] (mkbibgiven.south) edge [out=-90, in=90] (mkbibbold.north);
 \path[-stealth] (mkbibbold.south) edge [out=-90, in=180] (boldklaus.west);
 \path[-stealth] (prefixname.south) edge [out=-90, in=90] (mkbibprefix.north);
 \path[-stealth] (mkbibprefix.south) edge [out=-90, in=90] (mkbibemph1.north);
 \path[-stealth] (mkbibemph1.east) edge [out=0, in=90] (emvon.north);
 \path[-stealth] (familyname.south) edge [out=-90, in=90] (mkbibfamily.north);
 \path[-stealth] (mkbibfamily.south) edge [out=-90, in=90] (scname.north);
 \path[-stealth] (scname.south) edge [out=-90, in=90] (scbulow.north);
 \path[-stealth] (suffixname.east) edge [out=0, in=90] (mkbibsuffix.north);
 \path[-stealth] (mkbibsuffix.south) edge [out=-90, in=90] (mkbibemph2.north);
 \path[-stealth] (mkbibemph2.south) edge [out=-90, in=0] (emiii.east);
 }
\end{minipage}
\caption{How names are styled\label{namestyle}}
\end{figure}

Of course, it's usually unnecessary to have so many different
definitions. A much more common approach would be to have all the
various parts of a name formatted in the same way. But the flexibility
is there if you need it.\intref{Biblatex Manual § 12}

\index{customization!names}
\paragraph{I want the names to be printed `Firstname Lastname', but
  I'm getting `Lastname, Firstname'} Although there is, in theory,
infinite flexibility, in practice there are three common patterns to
how names are printed:
\begin{itemize}
\item In `ordinary' order: given name then family name --- Joe Bloggs
  and John Doe.
\item In `reverse' order: family name then given name --- Bloggs, Joe
  and Doe, John.
\item With the given name in the list reversed, but the other in
  ordinary order --- Bloggs, Joe and John Doe.
\end{itemize}
Styles generally set these sensibly, but if you do not like the
choice, you can usually change it without much difficulty. The command
you need is
\begin{pseudoverb}
  \cs[sortname]{DeclareNameAlias}\{\angled{order}\}
\end{pseudoverb}
Where \angled{order} can be |given-family| (for ordinary order),
|family-given| (for reversed order) or |family-given/given-family| (for the
hybrid form). Some styles reset this at the start of the bibliography,
in which case you \emph{also} need (in your preamble)
\begin{pseudoverb}
\cs{AtBeginBibliography}\{\cs[...]{DeclareNameAlias}\}
\end{pseudoverb}

\index{customization!list of own papers}\index{CV!list of papers}
\paragraph{I want to produce a publication list of my own papers.}
People writing CVs often want to produce a bibliography of their own
papers, but modifying the name format so that instead of placing their
name wherever it might happen to be in the list (it could be with `and
others'!) the list reads `with \ldots'. For this purpose you can use
the \package{biblatex-publist} style, which is specifically designed
for it. You tell the package your name (using the \cs{omitname}) macro
(or package option).

\indexstart{punctuation}\index{customization!punctuation|see{punctuation}}
%\section{Punctuation issues}
\section{标点问题}\label{sec:punctuation}

There are many configurable punctuation marks in \biblatex---too many
to deal with comprehensively here. In this section I'm only going to
cover some particularly common cases.\marginnote[-8ex]{In general, I use the
  technique of redefining macros. An alternative method, using
  \cs{DeclareDelimFormat} is also possible, and may be better if you are
  writing a complex style. For an example see p
  \pageref{declaredelim}.}

\index{punctuation!before postnote}
\paragraph{I want different punctuation before a postnote.}
\begin{marginfigure}
\fbox{\includegraphics{examples/punctcite.pdf}}
\vspace{3pt}%
\caption{Citation punctuation}\label{punctcite}
\end{marginfigure}
The overall punctuation of a citation is shown in figure
\ref{punctcite}. The point marked (1) is the place between the
pre-note and the beginning of the citation where \biblatex\ will
insert \cs{prenotedelim}. Here that command is defined to print
nothing, so in fact nothing gets inserted. The point marked (2) is the
place between the citation and the post-note where \biblatex\ will
insert the \cs{postnotedelim}. In this case that is a spaced comma.

So if we redefine both these commands, we can alter what gets
inserted:\begin{marginfigure}[12ex]
\fbox{\includegraphics{examples/punctcite2.pdf}} \vspace{3pt}%
\caption{Citation punctuation revised}\label{punctcite2}
\end{marginfigure}
\begin{Verbatim}
     \renewcommand{\prenotedelim}{\addcolon\space}
     \renewcommand{\postnotedelim}{\space--\space}
\end{Verbatim}

\index{punctuation!author/year}
\paragraph{I am using an author/year style, and I want a colon/a
  comma/nothing between the author and the year.}
\begin{marginfigure}[4ex]
\fbox{\includegraphics{examples/punctcite3.pdf}}
\vspace{3pt}%
\caption{\cs{nameyeardelim}}\label{punctcite3}
\end{marginfigure}
This is controlled by the \cs{nameyeardelim} macro (see figure
\ref{punctcite3}). In the standard styles it is, as you can see, a
space. If we wanted to make it, say, a comma, we could do so as
follows:
\begin{Verbatim}
    \renewcommand{\nameyeardelim}{\addcomma\space}
\end{Verbatim}
to produce the revised punctuation that is marked in figure
\ref{punctcite4}.
\begin{marginfigure}[4ex]
\fbox{\includegraphics{examples/punctcite4.pdf}}
\vspace{3pt}%
\caption{\cs{nameyeardelim} revised}\label{punctcite4}
\end{marginfigure}

\index{punctuation!multiple citations}
\paragraph{I have multiple citations and I want different punctuation
  between them.} As can be seen from figure \ref{punctcite5}
\begin{marginfigure}[4ex]
\fbox{\includegraphics{examples/punctcite5.pdf}}
\vspace{3pt}%
\caption{Multiple citations}\label{punctcite5}
\end{marginfigure}
the mark placed between most multiple citations is defined either by
\cs{multicitedelim} or by \cs{bibrangedash}. You are very unlikely to
want to redefine the latter command (which is used in many other
places), but \cs{muliticitedelim} can easily be redefined (a comma and
a space, or a semicolon and a space being the most common). To give a
(silly) example
\begin{Verbatim}
   \renewcommand{\multicitedelim}{\slash}
\end{Verbatim}
would give us the rather bizarre results show in figure
\ref{punctcite6}.
\begin{marginfigure}[-8ex]
\fbox{\includegraphics{examples/punctcite6.pdf}}
\vspace{3pt}%
\caption{Multiple citations redefined}\label{punctcite6}
\end{marginfigure}

Between \smallcaps{superscript citations} the delimiter is not
\cs{multicitedelim}, but \cs{supercitedelim}, so if you use
superscript citations it is this that you will have to redefine.

\index{punctuation!after `In'}
\paragraph{I don't want punctuation after `In'.} Standard \biblatex\
styles print a colon after `In:' This is produced by
\cs{intitlepunct}, as shown in figure \ref{punctcite7}.
\begin{figure}
\fbox{\includegraphics{examples/punctcite7.pdf}}
\caption{In: $\rightarrow$ In}\label{punctcite7}
\end{figure}

You can therefore redefine \cs{intitlepunct}, for instance to
\begin{Verbatim}
   \renewcommand{\intitlepunct}{\addspace\nopunct}
\end{Verbatim}
to avoid any punctuation being added at that point.

\index{customization!in@`In'!removal}
\paragraph{I want to remove the reference to `in' altogether.}

In that case
\begin{verbatim}
\renewbibmacro{in:}{}
\end{verbatim}

The trouble with this is that it removes `in' not only from, say,
articles, but also from other types of source (such as |incollection|)
where it may still be needed. In that case, you need a more complex
revision.\footnote{Thanks to Herbert:
  \url{http://tex.stackexchange.com/questions/10682/suppress-in-biblatex}},
using \cs[\angled{type}]{ifentrytype}. The following, for instance will
suppress `in' for |article| and |inproceedings| only, and you should
be able to work out how to adapt it.
\begin{verbatim}
\renewbibmacro{in:}{%
  \ifboolexpr{%
     test {\ifentrytype{article}}%
     or
     test {\ifentrytype{inproceedings}}%
  }
   {}%<-nothing
   {\printtext{\bibstring{in}\intitlepunct}}%<-normal 'in' with punctuation
}
\end{verbatim}


\index{customization!labels in parentheses}
\paragraph{I am using a numeric style, but I want my citations to
  appear in parentheses (1) instead of brackets
  [1].}\label{recipe:brackets}
\begin{marginfigure}[8ex]
\fbox{\includegraphics{examples/punctcite8.pdf}}
\vspace{3pt}%
\caption{Brackets replaced by parentheses}\label{punctcite8}
\end{marginfigure}
This is a surprisingly difficult change to made. The code required is
as follows:\footnote{Using the clever method devised by Audrey
  Boruvka: \url{http://tex.stackexchange.com/a/16792/5404}.}

\begin{Verbatim}
\makeatletter

\newrobustcmd*{\parentexttrack}[1]{%
  \begingroup
  \blx@blxinit
  \blx@setsfcodes
  \blx@bibopenparen#1\blx@bibcloseparen%
  \endgroup}

\AtEveryCite{%
  \let\parentext=\parentexttrack%
  \let\bibopenparen=\bibopenbracket
  \let\bibcloseparen=\bibclosebracket}

\makeatother

\DeclareFieldFormat{labelnumberwidth}{\mkbibparens{#1}}
\DeclareFieldFormat{labelformat}{\mkbibbrackets{#1}}
\end{Verbatim}

\index{customization!citations in brackets}
\paragraph{I am using an author/year style: I want to replace round
  parentheses with square brackets: [Author 1978] instead of (Author
  1978).} This is more-or-less the opposite problem to the last one,
and again it takes a surprising amount of work to get his right. The
relevant code is as follows (to go in the preamble).
\begin{Verbatim}
\makeatletter

\newrobustcmd*{\parentexttrack}[1]{%
  \begingroup
  \blx@blxinit
  \blx@setsfcodes
  \blx@bibopenbracket#1\blx@bibclosebracket%
  \endgroup}

\AtEveryCite{%
  \let\parentext=\parentexttrack%
  \let\bibopenbracket=\bibopenparen%
  \let\bibclosebracket=\bibcloseparen}

\makeatother
\end{Verbatim}

If you also want the year references to be printed in square brackets
in the bibliography, you need this in addition:
\begin{Verbatim}
\renewbibmacro*{date+extrayear}{%
  \iffieldundef{\thefield{datelabelsource}year}
    {}
    {\printtext[brackets]{%
       \iffieldsequal{year}{\thefield{datelabelsource}year}
         {\printdateextralabel}%
         {\printfield{labelyear}%
          \printfield{extrayear}}}}}%
\renewbibmacro*{date}{}%
\renewbibmacro*{issue+date}{%
  \iffieldundef{issue}
    {}
    {\printtext[brackets]{\printfield{issue}}}%
  \newunit}
\end{Verbatim}

\index{punctuation!within entry}
\paragraph{I would like the parts of a bibliographical entry to be
  separated by commas, but the standard styles use full stops.} The
division between the main parts of an entry are marked by
\cs{newunitpunct}. As you can see from figure \ref{punctcite10}, it is
quite frequently used.
\begin{figure}
\fbox{\includegraphics{examples/punctcite10.pdf}}
\caption{The \cs{newunitpunct} punctuation\label{punctcite10}}
\end{figure}

If we modify it by
\begin{Verbatim}
\renewcommand{\newunitpunct}{\addcomma\space}
\end{Verbatim}
we get a less choppy result: figure \ref{punctcite11}. Notice that the
capitalization adjust automatically.\begin{figure}
  \fbox{\includegraphics{examples/punctcite11.pdf}}
\caption{The \cs{newunitpunct} punctuation\label{punctcite11}}
\end{figure} It may, however, sometimes be necessary to adjust other
punctuation marks to ensure a really smooth result.

\index{punctuation!quotation marks}
\paragraph{I am not happy with the way quotation marks are being put
  around article names.} This is largely a function of how
\package{csquotes} is loaded.
\begin{itemize}
\item Loaded with the option |style=british|, you will get single
   quotation marks: `Like This'.
\item Loaded with the option |style=american|, you will get doubt
   quotation marks: ``Like This''.
\item If, in addition to loading \package{csquotes} with the option
   |style=american| you load \package{babel} or \package{polyglossia} with the
   |american| language you will get the American style of handling punctuation
   and quotations, where punctuation is moved inside the quotation:
   ``like this,'' in most cases.
\end{itemize}
Of course, you can equally use \package{csquotes} with «\,French\,»,
,,German``, «Spanish», or any other style of quotation.

%''

%\section{Issues about what gets printed}
\section{关于打印什么内容的问题}

\index{customization!URL!font}\index{customization!font!url}
\paragraph{I don't want to print \smallcaps{urls}\slash
  \smallcaps{eprint}\slash \smallcaps{doi}s.} These can be controlled
with the options |url|, |doi| and |eprint|. By default all of these
are set |true|; to turn off any of them set the relevant option to
|false|. If you want links to be created you should load the
\package{hyperref} package.

\index{customization!date}
\index{date!customization}
\paragraph{I want to control how dates get printed.}
\package{Biblatex} may print a number of different dates at various
points in a citation, and the format for each of them can be
separately controlled, using the options given in table
\ref{dateopts}.
\begin{margintable}
\begin{tabularx}{\marginparwidth}{lX}
\toprule
\texttt{date}      & date of publication \\
\texttt{labeldate} & the `label' date (in author/year styles) \\
\texttt{urldate}   & the date a \smallcaps{url} was `last visited' \\
\texttt{eventdate} & the date of a conference \\
\texttt{origdate}  & an original date of publication \\
\texttt{alldates}  & all dates \\
\bottomrule
\end{tabularx}
\vspace{3pt}
\caption{The various options to set date\label{dateopts}}
\end{margintable}

There is a bewildering range of possible settings for these various
date fields. Exactly what will get printed depends may depend on the
language you have set in \package{babel} or \package{polyglossia}: a
date printed as `15th January 2000' in |british| will be printed as
`January 15, 2000' in |american| and 15 janvier 2000 in
|french|. However, tabel \ref{datesettings} gives examples for how a
date of |2000-01-15| would get printed in British English which should
be sufficient to give guidance for other languages.
\begin{table*}
  \begin{tabular}{lp{2.8in}lp{1in}}
    \toprule
                          & \texttt{dateabbrev=false}                                    & \texttt{dateabbrev=true} & \texttt{datezeros=false} \\
    \midrule\texttt{long} & 15th January 2000                                            & 15th Jan 2000                                       \\
                          & ranges written in full: 15th January 2000--17th January 2000 & 15th Jan.\ 2000--17 Jan.\ 2000                      \\
    \texttt{short}        & 15/01/2000                                                   &                          & 15/1/2000                \\
                          & ranges written in full: 15/01/2000--17/01/2000               &                          & 15/1/2000--17/1/2000     \\
    \texttt{year}         & 2000                                                                                                               \\
    \texttt{comp}         & 15th January 2000                                                                                                  \\
                          & ranges compressed: 15th--17th January 2000                   & 15th--17th Jan.\ 2000                               \\
    \texttt{terse}        & as 15/01/2000                                                                                                      \\
                          & ranges compressed: 15--17/01/2000                            &                          & 15--17/1/2000            \\
    \texttt{iso8601}      & 2000-01-15                                                                                                         \\
                          & with ranges as 2000-01-15/2000-01-17                                                                               \\
    \bottomrule
\end{tabular}
\caption{Various date settings\label{datesettings}}
\end{table*}
In addition, you set the |dateabbrev| option |true| to have dates
printed in an abbreviated form (15th Jan.\ 2000), and the |datezeros|
option |true| to have numeric dates printed with leading zeros
(2000-01-15), or |false| to have them printed without.

\index{customization!suppressing information}
\paragraph{There's some other information I don't want printed.} At
this point we are beginning to enter the realm of serious changes,
which are arguably beyond the sort of simple customization that this
chapter is concerned with.

But let's suppose, for instance, that you were absolutely happy with
the standard author/year style, but just wanted not to print the
publisher: just the place of publication. One way to do this is to
`kid' the style that you haven't entered any publisher at all.
\begin{figure}
\fbox{\includegraphics{examples/punctcite12.pdf}}
\vspace{0.5\baselineskip}

\fbox{\includegraphics{examples/punctcite13.pdf}}
\caption{Removing the publisher using a \texttt{sourcemap}\label{punctcite12}}
\end{figure}

For this purpose we use a \emph{sourcemap}, as follows

\begin{minipage}{\textwidth}
\begin{Verbatim}
\DeclareSourcemap{
  \maps[datatype=bibtex]{
    \map[overwrite=true]{
      \step[fieldset=publisher,null]
    }
  }
}
\end{Verbatim}
\end{minipage}

\vspace{1ex}
What this basically means is that for any |.bib| file, the publisher
field of each entry is to be set empty (|null|). This will be done by
\package{biber} before any data enters the system. No publisher is
therefore printed, and so long as the style is capable of coping with
the particular absence, all is well, as figure \ref{punctcite12}
shows.

%%% Local Variables:
%%% coding: utf-8
%%% mode: LaTeX
%%% TeX-master: "biblatex-tutorial"
%%% End: 

\chapter{Tools and Editors}\label{ch:tools}

Most people who use \LaTeX\ use a specialised editor (or a special
mode or plugin for an editor) when working with \LaTeX. There are also
special tools for maintaining bibliographies. These tools are not
identical. That's sometimes a strength (people can have very strong
views about their preferred editor!), but for newcomers it can be a
definite weakness. In addition, many of the tools assume that you will
be using \bibtex\ -- which was a fair assumption in the past, and some
need `coaxing' to encourage them to work easily with \biblatex.

\section{In the shell}

\index{compiling}
I think it is important, whatever editor you use, to know how to
compile a file from the command line. Even the best tools sometimes
fail, and it's surprising how often (especially with a complex
project) a basic compilation from a shell will help diagnose
problems. So every user should know how to open a shell, navigate to
the directory in which the source file is found, and run
\begin{pseudoverb}
pdflatex\footnote{or \texttt{xelatex} or \texttt{lualatex}} \angled{name}\\
biber \angled{name}\\
pdflatex \angled{name}\\
pdflatex \angled{name}
\end{pseudoverb}
The only trick is with \angled{name} component: |pdflatex| wants to
get to work on a file with a |.tex| extension; |Biber| wants to get to
work on a file with the extension |.bcf|\footnote{\texttt{bcf} stands
  for bibliography control file.} (`Old-timers' may erroneously try to
run |biber| on the |.aux| file, which is what \bibtex\ would use.)
But, so long as you use standard extensions, problems can be
avoided by running the program in question on the `basename': the
filename \emph{without any extension}. So the following are
equivalent, if our source code is in a file called |myfile.tex|:
\begin{pseudoverb}
\begin{tabbing}
pdflatex myfile.tex\qquad   \= pdflatex myfile\\
biber myfile.bcf \> biber myfile\\
pdflatex myfile.tex \> pdflatex myfile\\
pdflatex myfile.tex \> pdflatex myfile
\end{tabbing}
\end{pseudoverb}
You won't \emph{always} need all these runs. You generally only need
to run \package{Biber} if you have either cited a new source or
significantly changed your bibliography options (for instance the
sorting). And you won't always need to run \LaTeX\ twice after using
\package{biber}. In any case, you get useful messages at the end of
compilation, telling you what to do.

Nevertheless, for a project of any complexity, this sort of need to
carry out multiple runs becomes tiresome, and you may want to look at
automating it. There are a few ways you can do this. You could use a
standard compiler program like |make|. There's nothing wrong with
this, and if you are happy writing |makefiles| it's a perfectly
reasonable way to go. But there are two \LaTeX\ specific tools which
are more likely to help you.

\subsection{Latexmk}

\index{latexmk@\package{latexmk}}
First is \package{latexmk}, which is a Perl script. It therefore
requries Perl to be installed. Perl always will be on Linux or Mac OS X
machines, but Windows users will need to download it (e.g. from
\url{strawberryperl.com}). In principle, and for reasonably simple
projects at least, you can substitute all the commands given above
with the simple:
\begin{pseudoverb}
latexmk myfile
\end{pseudoverb}
The script will then run \LaTeX, and auxiliary programs such as
\package{biber} and \package{makeindex}, as many times as necessary to
produce a stable result (if that's possible). The \package{latexmk}
script is `\package{Biber} aware': it will detect if it needs to run
\package{Biber} (or \bibtex) and proceed accordingly. There are many
options (described in the documentation).

\subsection{Arara}

\index{Arara@\package{Arara}}
The alternative modern approach is a program called
\package{Arara}. This is a Java program (and therefore requires a Java
command line to be installed). Instead of attempting to `detect' what
needs to be done to produce the final printable version,
\package{Arara} enables you to place commented instructions in your
source file, which essentially tell the program how to go about
processing it. So, for instance, you might put the following at the
start of your file:
\begin{pseudoverb}
\% arara: pdflatex\\
\% arara: biber\\
\% arara: pdflatex\\
\% arara: pdflatex
\end{pseudoverb}
You then run (from a command line)
\begin{pseudoverb}
arara \angled{filename}
\end{pseudoverb}
The package then reads the comment lines in the file to work out what
it should do. Effectively, you are writing makefile-like instructions,
but in a conveniently simple form and in the file you will
compile. You have to be explicit about them, but you may find this an
advantage over the sort of `black box' that \package{latexmk}
offers. Again, there are all sorts of bells and whistles -- but
\package{Arara} is, as the example above shows, `\package{biber}
aware'.

\subsection{Texify}

\index{Texify@\package{Texify}}
The \package{mikTeX} distribution comes with a command called
|texify|, which is intended to do something along the lines of
|latexmk|, but without the need to install Perl. This is \emph{not}
\package{Biber}-friendly: it assumes you will run \bibtex. Although it
is reputed to be possible to convert it to run \package{Biber} by
setting the environment variable |BIBTEX| to |biber|, I have not had
any success doing so. In my view, it's better to install and use
either \package{latexmk} or \package{Arara}.

\section{Editors}

\index{editors}
There are many different editors available and commonly used for
\LaTeX, most of which offer at least some support for the use of
\package{Biber} and \biblatex. For some, such as
\package{Emacs} with \package{AUC\TeX}, support is already built
in. In other cases it is necessary to add or adjust the editor's
setting in order to configure it to use \package{Biber} rather than
\bibtex. Given the vast number of editors, each slightly different,
and the speed at which they change, it's not practical to give
detailed instructions here. Information, periodically updated, may be
found at
\url{http://tex.stackexchange.com/questions/154751/biblatex-with-biber-configuring-my-editor-to-avoid-undefined-citations}.

A practical alternative to configuring your editor to use
\package{Biber} is to configure it to use \package{latexmk} or
\package{Arara}, as described above.

\section{Maintaining Databases}

\index{BibDesk@\package{BibDesk}}\index{JabRef@\package{JabRef}}
You can always maintain a database yourself, through a decent text
editor. Some editors (such as \package{Emacs}) have a special mode for editing
\bibtex\ datatabase files, which makes the process much smoother. But
there is also software designed to do this for you. Two commonly used
tools are \package{JabRef} and (for Mac OSX) \package{BibDesk}. Each
provides a graphical interface for entering bibliographic data, and
useful reminders of the fields that you will need to complete.

Such packages are designed for maintaining a database file (or files)
on a particular computer. Increasingly there are web- and cloud-based
bibliography tools such as \package{Mendeley} and \package{Zotero}.
Other web-based services (such as Google Scholar) will often also
offer pre-formed \bibtex\ records. These need to be scrutinized with
care, because they are not always accurate or complete.

Support for \biblatex\ in these tools varies, and changes over
time. You may be able to find useful up-to-date information at
\url{http://tex.stackexchange.com/questions/23942/bibliography-tools-that-are-compatible-with-biblatex-and-biber}.

%%% Local Variables:
%%% coding: utf-8
%%% mode: LaTeX
%%% TeX-master: "biblatex-tutorial"
%%% End:

\chapter{When Things Go Wrong}\label{ch:troubleshooting}

Things go wrong, and when they go wrong it can be quite infuriating
trying to work out what has happened. This chapter aims to guide you
through the possible symptoms you might see to try to help you to an
answer.

In this chapter I am not going to deal with the detail of debugging
changes to \biblatex\ styles, but with the sort of thing that goes
rather obviously wrong for an ordinary user running properly written
styles. The manifestation of this problem is nearly always the same:
instead of citations, you have labels in square brackets in the
printed brackets (as in figure \ref{nussbaum1}; you see warnings in
the \LaTeX\ log file telling you
\begin{verbatim} 
Citation 'blah' on page x undefined on input line y
\end{verbatim}
and at the end of the log file you are encouraged to |(re)run Biber|.
\index{errors|see{debugging}}
\index{debugging!empty citations}

\section{A Systematic Approach}

\indexstart{debugging!general approach}
The first thing to do in this situation is, paradoxically, to start
with a clean slate.\marginnote{If you use \package{latexmk} you can
  clean all these files away with the command \texttt{latexmk -c}.}
When you are working on a long document, you will end up with all
sorts of files in your working directory that you don't actually need,
and sometimes they can make debugging harder. So start cleanly: delete
all `automatically created' files: in particular all |.aux|, |.bcf|,
|.blg| and |.bbl| files.

\subsection{Run \LaTeX}

Now, from a command line, run |pdflatex| \angled{filename} (or
whichever \LaTeX\ you use. Don't worry much about the warnings: this
is a first clean run, and you \emph{expect} undefined citations; but
if \LaTeX\ is encountering errors, you obviously need to sort them
out: for instance, if you have misspelled |biblatex| or attempted to
load a non-existent bibliography style, you can't expect to produce
citations. Once the document compiles (with, as expected, warnings
about undefined references), you can move on to the next stage.

\subsection{Run \package{Biber}}

The next task is to run \package{Biber}.  This time, you should
examine your output, either on the console or in the log file (which
will have the extension |.blg|). If all has gone well you will see
something like figure \ref{biber:run}.
\begin{figure}
\begin{Verbatim}[frame=single,fontsize=\small]
INFO - This is Biber 2.7
INFO - Logfile is 'punctcite13.blg'
INFO - Reading 'punctcite13.bcf'
INFO - Found 1 citekeys in bib section 0
INFO - Processing section 0
INFO - Looking for bibtex format file 'biblatex-examples.bib' 
       for section 0
INFO - Decoding LaTeX character macros into UTF-8
INFO - Found BibTeX data source '/biblatex-examples.bib'
INFO - Overriding locale 'en-US' defaults 'variable = shifted'
        with 'variable = non-ignorable'
INFO - Overriding locale 'en-US' defaults 'normalization = NFD' 
       with 'normalization = prenormalized'
INFO - Sorting list 'nty/global/' of type 'entry' with scheme 'nty' 
       and locale 'en-US'
INFO - No sort tailoring available for locale 'en-US'
INFO - Writing 'punctcite13.bbl' with encoding 'ascii'
INFO - Output to punctcite13.bbl
\end{Verbatim}
\caption{Log of a successful \package{biber} run\label{biber:run}}
\end{figure}

If \package{Biber} has run successfully, you can go on to the next
step. If you see any lines in the log which begin |ERROR| or |WARN|
then there was some sort of problem, and you will need to do more
digging.

\newthought{Once \package{Biber} has actually run}, even with
warnings, you will find that the log provides a useful diagnostic
tool. The most common errors are set out in table \ref{biber:errors},
together with their meanings and the action you should take. But this
is not a comprehensive list. Occasionally, you may get other errors:
in the worst case, the progam may simply fail to run fully. This is
very rare indeed. When it happens it is usually because of some occult
problem in your |.bib| file.

\begin{table*}
\begin{tabular}{p{5cm}p{5cm}p{5cm}}
\toprule
\ttfamily WARN - No data sources defined!                                                                                               & Your source file does not contain any \cs{addbibresource} commands.                                                                                                              & Check your source file, and make sure it includes \cs{addbibresource} commands for every data source you need.                                                                                                                                                                                                                                            \\
\midrule\ttfamily ERROR - Cannot find \angled{database file}!                                                                           & Biber was not able to find the |.bib| file you have referred to.                                                                                                                 & Make sure you have spelled the name correctly. Make sure the file is somewhere that \TeX\ can find it, either in a place in your \TeX-\textsc{mf} tree where it will be found, or in the same directory as the source file. (If you don't understand this, just play safe: put it with your source file!)                                                 \\
\midrule\ttfamily WARN - Entry \angled{label} does not parse correctly.
ERROR - BibTeX subsystem: \angled{filename} ... syntax error: found "title", expected end of entry ("\}" or ")") (skipping to next "@") & You have not written a particular entry correctly: usually (as in this case) by forgetting a comma between fields, sometimes an unescaped comment character (\%) is the culprit. & Find the entry that did not parse, and correct it.                                                                                                                                                                                                                                                                                                        \\
\midrule\ttfamily WARN - BibTeX subsystem: \angled{filename} warning: 1 characters of junk seen at toplevel                             & There is something \emph{outside} and entry, other than the beginning of a new entry. The usual reason for this is a stray comma between entries.                                & Check your |.bib| file around the place where the `junk' appeared. Note that if this happens the rest of the file will fail to parse, and you may well get missing citations too.                                                                                                                                                                         \\
\midrule\ttfamily WARN - I didn't find a database entry for \angled{label}                                                              & That citation label is not included in your \texttt{.bib} file (or it is included after an error in that file from which recovery was not possible)                              & Often this is a typo (in either the source file or the database): check there is a source with that label, spelled exactly identically. The other common source of (many!) such warnings is that if there is a spoiled entry in the \texttt{.bib} file (leading to the message about `junk' having been seen) all subsequent entries will be ignored too. \\
\bottomrule
\end{tabular}
\vspace{10pt}
\caption{Biber errors, and what they mean\label{biber:errors}}
\end{table*}

\index{debugging!errors in bib file@errors in \texttt{.bib} file}
To diagnose such an error, proceed as follows:
\begin{itemize}
\item Rename your existing |.bib| file.
\item Gradually copy your |.bib| file entries back into an empty file
  (with the old name). Either copy a few entries at a time, or use the
  technique of copying half the entries, then either removing half (if
  the first chunk failed) or progressively adding more.
\item Each time you do this, rerun \package{biber} (you shouldn't need
  to re-run \LaTeX). You should expect, and ignore, warnings about
  missing citations; you are simply looking to get a run that does not
  abort.
\item In this way you should be able to identify the problem
  entries. Remove these.
\item Retype the problematic entries by hand, checking that they are
  correct. Where an error sufficiently grave to prevent
  \package{biber} from running occurs, the problem seems usually to be
  with some unacceptable encoding in the file, and manually
  re-entering the entry will usually sort it out.
\end{itemize}

Once you have corrected any errors, re-run \package{Biber}. If you had
to correct an error in your \emph{source file} (e.g. to add or change
an \cs{addbibresource} command), you will need to re-run (pdf)-\LaTeX,
and then \package{Biber}. If you only had to change entries in the
|.bib| file, you should be OK with just running \package{Biber}.

\subsection{Finally, run \LaTeX\ again}

Once you manage to run \package{Biber} without errors or warnings,
re-run \LaTeX. This time you should get a complete compilation,
without any warnings.

There is one category of error that can sometimes occur at this
point. Just occasionally you will find that when \LaTeX\ is run you
get complaints about problems with some unicode character. The problem
occurs because \package{biber} works internally in one flavour of
unicode normalisation, but \LaTeX\ does not, and even (in fact,
especially) when unicode is used (via |\usepackage[utf8]{inputenc}|)
there can be occasional incompatibilities. If you see this sort of
trouble, try running \package{biber} with the option
|--output_safechars|.
\indexstop{debugging!general approach}

\section{A less systematic approach}

Sometimes a less systematic approach is appropriate. If a document and
bibliography have all been compiling correctly, but suddenly you start
to see problems, particularly with one or two entries, go back and
check those particular entries. If those look correct (the label is
correctly spelled in the |.tex| source, the entry seems correct in the
|.bib| file) it's worth carrying out a manual run of \package{Biber}
just to check it is working correctly.

\section{Problems with output}

\indexstart{debugging!output problems}
Problems with output have three possible sources. (a) It may be that
your bibliography style is, in some way, not to your taste. This is
not really a `bug' as such: it may well be that the style is doing
exactly what its author intended it to. There are, of course, various
ways in which you may be able to tweak a style to your liking, some of
which are discussed elsewhere in this document. But the problem is not
one of debugging. (b) It may be that there actually \emph{is} a bug in
the bibliography style you have used. These things are complex, and
bugs happen. If you are nervous about them, try to stick to
well-established and maintained styles. Other possible things to do
are to get help online\footnote{For instance at
  \url{tex.stackexchange.com}} or to contact the style's maintainer by
email. (c) It may be a problem with your |.bib| file.

It's only common sense to check the last point first. Among the common
errors in a file which will `work' are the following:
\begin{itemize}
 \item Incorrect lists of names. It's terribly easy to write
 \begin{pseudoverb}
John Smith, A. U. Thor, Joe Bloggs
 \end{pseudoverb}
 when you mean
 \begin{pseudoverb}
John Smith and A. U. Thor and Joe Bloggs
\end{pseudoverb}
\item Incorrect names, particularly forgetting to put a full stop
  after initials (especially since it often makes no difference):
  |Smith, J| instead of |Smith, J.|
\item Forgetting to put extra braces round institutional authors, so
  that |Press Complaints Commission| becomes `P. C. Commission'.
 \item Forgetting to put braces around a word whose capitalization should not be changed, so that |German| becomes |german|
 \item Inadvertently muddling similar fields, such as |pages| and
   |pagination|, |journaltitle| and |series|, |volume| and |part|.
\end{itemize}
\indexstop{debugging!output problems}

\section{Further help}
The main source of help on \biblatex\ is its manual, which can be
found either online or, on a properly installed \TeX\ system, by
typing |texdoc biblatex| at the command line. Individual styles also
have their own documentation, which is sometimes extensive. That too
can be found using the |texdoc| command, adding the name of the
package.

When the going gets tough, it's sometimes helpful to have an
experienced person look at the problem. If you don't have such a guru
on tap, you can always ask a question at
\url{http://tex.stackexchange.com}, where the |biblatex| tag appears
frequently (and the package's maintainers regularly check
questions). If you decide to do that, please consider the following:
\begin{itemize}
\item Make a good faith effort to solve the problem yourself first,
  including by making sensible searches for duplicate questions.
\item Tag your question |biblatex|.
\item Post code which illustrates the problem.
\item But \emph{don't} just post or provide a link to a huge
  file. Make a new file, using a standard class (such as |article|)
  which shows it. Similarly, remove all packages that seem to be
  irrelevant to the problem. Preparing such a document may well help
  you to isolate it. For instance, if you can't reproduce the problem
  unless you add some odd package or antiquated class file, the
  solution may be obvious.
\item Include a minimal bibliography file in the example
  itself. There's an easy way to do this:
  \begin{itemize}
  \item Add |\usepackage{filecontents}| to your preamble.
  \item Place the necessary bibliography entries, just as in your
    |.bib| file, within an environment that begins
    |\begin{\filecontents}{\jobname.bib}| and
      |\end{filecontents}|. This will cause \LaTeX\ to create such a
    file on its first run.
  \item Use |\addbibresource{\jobname.bib}| to load that file.
  \end{itemize}
\item Be absolutely specific about what the problem is, and what a
  solution would look like. Add an image which shows what is wrong and
  what you need.
\item Don't take it amiss if someone on the site manages to see that
  your question is a duplicate, and points that out. This is not an
  agressive gesture. It's sometimes hard for anyone other than an
  expert even to \emph{realise} that what look like two different
  problems are really just different manifestations of the same
  difficulty.
\end{itemize}

%%% Local Variables:
%%% coding: utf-8
%%% mode: LaTeX
%%% TeX-master: "biblatex-tutorial"
%%% End:

\backmatter
\chapter{Quick Start Guide}\label{ch:quickstart}
\marginpar{\begin{CJK}{UTF8}{gbsn}\footnotesize 快速入门\end{CJK}}

\biblatex\ is a modern, and vastly more flexible, replacement for
\bibtex\ for those needing to automate bibliography production in
\LaTeX. These pages provide a very quick `quick start' guide. A rather
more carefully explained guide is found in
chapter~\ref{ch:introduction}, and of course much more information
throughout the guide, and in the \biblatex\ manual.
\marginpar{
\begin{CJK}{UTF8}{gbsn}\footnotesize 
本章提供了最最简单的入门介绍,主要介绍参考文献生成需要准备的内容和步骤。详细入门介绍见第\ref{ch:introduction}章
\end{CJK}}


\section{Step 1: Create a \texttt{.bib} file}\marginpar{\begin{CJK}{UTF8}{gbsn}\footnotesize 第一步 准备bib文件
\end{CJK}}
Containing the bibliographical data. \biblatex\ follows the basic
structure of \bibtex, but expands on
it.\intref{Chapter~\ref{ch:database}} An example is shown in figure~\ref{quickstart:bib}.

\begin{figure}
\strut
  \begin{minipage}[t]{0.7\linewidth}
    \ttfamily
 @book\{\colorbox{blue!15}{work:1},\tikz{\node(iskeynode)[shape=coordinate]{};}\\
  author       = \{Aristotle\},\\
  title        = \{De Anima\},\\
  date         = 1907,\\
  editor       = \{Hicks, Robert Drew\},\\
  publisher    = \{CUP\},\\
  location     = \{Cambridge\},\\
\}\\
\end{minipage}\begin{minipage}[t]{0.7\linewidth}
  \small\sffamily\strut\vspace{1pc}

  \tikz{\node(iskeycommentnode)[shape=coordinate]{};} Key: a unique
key chosen by you and used in \cs{cite} commands to refer to this
source.
\end{minipage}
\begin{tikzpicture}[overlay, line width=1pt]
  \draw[red, arrows=<-] (iskeynode) -- (iskeycommentnode);
\end{tikzpicture}
\caption{A sample \texttt{.bib} file entry}
\label{quickstart:bib}
\end{figure}

\section{Step 2: Write source file}\marginpar{\begin{CJK}{UTF8}{gbsn}\footnotesize 第二步 准备tex源文件
\end{CJK}}

Prepare your source file, loading the \biblatex\ package, adding your
|.bib| file as a resource, and including citations and a
bibliography. An annotated same file appears in
figure~\ref{quickstart:tex}.

\begin{figure}
\begin{minipage}[t]{0.7\linewidth}
  \ttfamily
  \cs[article]{documentclass}\\
  \vspace{1pc}
  \colorbox{blue!15}{\parbox{0.67\textwidth}{
  \cs[english][babel]{usepackage}\tikz{\node(polyglossianode)[shape=coordinate]{};}}}\\
  \vspace{2pc}
  \colorbox{blue!15}{\cs[csquotes]{usepackage}}\tikz{\node(csquotenode)[shape=coordinate]{};}\\
  \vspace{1pc}
  \colorbox{blue!15}{\parbox{0.65\textwidth}{
  \cs[T1][fontenc]{usepackage}\\
  \cs[utf8][inputenc]{usepackage}}}\tikz{\node(encodingnode)[shape=coordinate]{};}\\
  \vspace{4pc}
  \colorbox{blue!15}{\cs[\colorbox{blue!30}{\tikz{\node(optionsnode)[shape=coordinate]{};}style=authoryear}][biblatex]{\raisebox{6pt}{\tikz{\node(biblatexnode)[shape=coordinate]{};}}usepackage}}\\
   \vspace{4pc}
  \colorbox{blue!15}{\cs[\colorbox{blue!30}{\tikz{\node(bibfilenamenode)[shape=coordinate]{};}refs.bib}]{\raisebox{6pt}{\tikz{\node(addbibnode)[shape=coordinate]{};}}addbibresource}}\\
  \vspace{1pc}
  \cs[document]{begin}\\
  \vspace{2pc}
  In \colorbox{blue!15}{\cs
    [\colorbox{blue!30}{work:1}]
    {\raisebox{6pt}{\tikz{\node(citenode)[shape=coordinate]{};}}cite}} someone
  said something interesting. And also in\\
  \colorbox{blue!15}{\cs[\colorbox{blue!30}{\tikz{\node(keynode)[shape=coordinate]{};}work:2}]{cite}}.\\
  \vspace{2pc}
  \colorbox{blue!15}{\cs{printbibliography}}\tikz{\node(printbibliographynode)[shape=coordinate]{};}\\
  \vspace{1pc}
  \cs[document]{end}
\end{minipage}%
\begin{minipage}[t]{0.7\linewidth}
  \sffamily\small
  \vspace{1pc}
  \tikz{\node(langnode)[shape=coordinate]{};} It is not necessary to use \texttt{polyglossia} or \texttt{babel},
  but if they are used they should be loaded first. For more
  information on \biblatex\ and languages, see chapter
  \ref{ch:languages}.
  \vspace{1pc}

  \tikz{\node(csqoptionalnode)[shape=coordinate]{};} Optional,
  but a good idea.

  \vspace{1pc} \tikz{\node(encodingcommentnode)[shape=coordinate]{};}
  Optional (and only applicable to \textsc{pdf}\TeX). But
  desirable to enable the use of unicode-encoded input files and
  modern fonts. An alternative is to use Xe\TeX\ or Lua\TeX\, which
  handle these matters natively.

  \vspace{1pc}

  \tikz{\node(biblatexcommentnode)[shape=coordinate]{};} Load \biblatex.
  \vspace{1pc}

  Here we use the
  \texttt{style} option to choose a style for citations
  \tikz{\node(optionscommentnode)[shape=coordinate]{};}and
  bibliography. There are numerous options to customize in various ways.

  \vspace{1pc}
  \strut\raisebox{6pt}{\tikz{\node(addbibcommentnode)[shape=coordinate]{};}}
  Identify the files where the bibliographical data will be found. Can
  have multiple \cs{addbibresource} statements.

  \vspace{0.5pc}
  \tikz{\node(bibfilenamecommentnode)[shape=coordinate]{};} Include the
  full filename, including extension (conventionally
  \texttt{.bib}. For detail about the format of this file see chapter~\ref{ch:database}.

  \vspace{1.3pc}
  \tikz{\node(citecommentnode)[shape=coordinate]{};} Citations use
  \cs{cite} commands: for other possibilities see
  chapter~\ref{ch:citationcommands}

  \vspace{2.2pc}
  \tikz{\node(keycommentnode)[shape=coordinate]{};} Key as defined in
  the |.bib| file.

  \vspace{2pc}
  \tikz{\node(printbibcommentnode)[shape=coordinate]{};} Print
  the bibliography. See chapter~\ref{ch:bibliographyformat}.

\end{minipage}
\begin{tikzpicture}[overlay,line width=1pt]
  \draw[red, arrows=<-] (polyglossianode.south) -- (langnode.west);
  \draw[red, arrows=<-] (csquotenode.east) -- (csqoptionalnode.west);
  \draw[red, arrows=<-] (encodingnode.east) --
  (encodingcommentnode.west);
  \draw[red, arrows=<-] (addbibnode.south) |- (addbibcommentnode.north);
  \draw[red, arrows=<-] (bibfilenamenode.south) |-  (bibfilenamecommentnode.south);
  \draw[red, arrows=<-] (biblatexnode.north) |-
  (biblatexcommentnode.north);
  \draw[red, arrows=<-] (optionsnode.south) |-
  (optionscommentnode.south);
  \draw[red, arrows=<-] (citenode.north) |- (citecommentnode.west);
  \draw[red, arrows=<-] (keynode.south) |- (keycommentnode.south);
  \draw[red, arrows=<-] (printbibliographynode.east) -- (printbibcommentnode.west);
\end{tikzpicture}
\caption{A sample \texttt{.tex} file.}
\label{quickstart:tex}
\end{figure}

\section{Step 3: Compile the \LaTeX\ source}\marginpar{\begin{CJK}{UTF8}{gbsn}\footnotesize 第三步 编译tex文档
\end{CJK}}

Compile the \LaTeX\ source file, using |pdflatex|, |xelatax|, or
|lualatex|.

\section{Step 4: Run \package{Biber}}\marginpar{\begin{CJK}{UTF8}{gbsn}\footnotesize 第四步 运行 biber \end{CJK}}

Run the \package{Biber} program to prepare bibliographical data. Do
this by running
\begin{pseudoverb}
  \centering
  biber \angled{basename}\\
  {\normalfont\itshape or} \\
  biber \angled{basename}.bcf
\end{pseudoverb}

\section{Steps 5 \ldots: Run \LaTeX\ at least once more}\marginpar{\begin{CJK}{UTF8}{gbsn}\footnotesize 第五步 至少再运行一次\LaTeX\ \end{CJK}}

Run \LaTeX\ at least once, and sometimes twice more. It will read in
the citation data that \package{Biber} has prepared, produce the
bibliography and citations, and prepare the final text.


%%% Local Variables:
%%% coding: utf-8
%%% mode: latex
%%% TeX-master: "biblatex-tutorial"
%%% End:


%\chapter{Aide Memoire or Cheat Sheet}
\chapter{备忘}
\label{ch:cheatsheet}

\paragraph{Load \biblatex\ and \package{csquotes}}
\begin{pseudoverb}
\cs{usepackage}[backend=biber,\\
style=\angled{style},\angled{options}]\{biblatex\}\\
\cs{usepackage}[style=\angled{style}]\{csquotes\}
\end{pseudoverb}

Always add a bibliography file with
\begin{pseudoverb}
\cs[\angled{bibfile.bib}]{addbibliography}
\end{pseudoverb}

\paragraph{Languages}
\begin{pseudoverb}
\cs{usepackage}[\angled{language}]\{babel\}\\
\emph{or} \cs{usepackage}\{polyglossia\}
\end{pseudoverb}
Load these before \biblatex.

\paragraph{Encodings and accents} Use \smallcaps{utf-8} in |.bib|
file, or use \TeX\ accent commands |\`| etc. Prefer unicode. Set input
encoding of |.tex| file with \package{inputenc}, or use Xe\TeX\ or Lua\TeX.

\paragraph{Citations}
Use |\autocite| and |\cite| in all styles. Use |\textcite| for
`running' citations. Use |\parencite| for parenthetical citations. Use
|\footcite| for footnote citations.

Citation commands take the form:
\begin{pseudoverb}
  \cs{cite}[\angled{postnote}]\{\angled{key}\}\\
  \emph{or}
  \cs{cite}[\angled{prenote}][\angled{postnote}]\{\angled{key}\}
\end{pseudoverb}

\paragraph{Printing the bibliography}
\begin{pseudoverb}
\cs{printbibliography}
\end{pseudoverb}

\paragraph{Compilation}
\begin{itemize}
\item Run \LaTeX.
\item Run \package{Biber}.
\item Run \LaTeX\ once or twice more.
\end{itemize}

\paragraph{Database} Entries take the form
\begin{pseudoverb}
@\angled{type}\{\angled{key},\\
 \angled{field} $=$ \{\angled{value}\}, \\
 \emph{or} \angled{field} $=$ "\angled{value}",
 \}
\end{pseudoverb}

\paragraph{Common types:}
\begin{itemize}
\item \texttt{@article} for journal article.
\item \texttt{@book} for book.
\item \texttt{@incollection} for edited collection.
\item \texttt{@inbook} for collection by single author.
\end{itemize}

\paragraph{Common fields:}
\begin{itemize}
\item \texttt{title} e.g. `Bleak House'
\item \texttt{journaltitle} e.g. `Nature'
\item \texttt{author} e.g. `Shakespeare, William'
\item \texttt{editor} e.g. `Houseman, A. E.'
\item \texttt{date} e.g. `2001-11-10', `1999'
\item \texttt{volume}, \texttt{issue}, \texttt{number}
\item \texttt{pages} e.g. `1-10'
\item \texttt{location} e.g. `London'
\item \texttt{publisher} e.g. `OUP'
\end{itemize}

\paragraph{Names}
\begin{pseudoverb}
von Lastname, Firstname I. \\
Lastname, F. I. \\
Firstname Lastname
\end{pseudoverb}

\paragraph{Multiple authors} Use `and' not `,':
\begin{pseudoverb}
A. Thor and Ann Other
\end{pseudoverb}

\paragraph{Debugging}
\begin{itemize}
\item Did \package{biber} run?
\item What error messages?
\item Is |bibresource| correctly identified?
\item Can \TeX\ find it?
\item Are all relevant |.bib| entries correct?
\end{itemize}

%%% Local Variables:
%%% coding: utf-8
%%% mode: latex
%%% TeX-master: "biblatex-tutorial"
%%% End: 

%!TEX root=biblatex-tutorial.tex
\appendix

\chapter{Examples}\label{chapter:examples}

The following pages give some simple examples. The examples are
typeset as a simple article format, using Times New Roman. In each
case (at least) three examplars are cited (drawn from the
|biblatex-examples.bib| database): one |book|, one |book| with an
editor, one |article| and one |incollection|: I've chosen these
because they seem to me to be typical of the sort of literature that
an academic article will cite, and once you see these it's quite easy
to predict how other entry types will be handled by the same style.

For reference, the |.bib| entries for the examples I have used are
given in figures \ref{eg:book}, \ref{eg:book2}, \ref{eg:incollection}
and \ref{eg:article} on pages \pageref{eg:book} to
\pageref{eg:article}.

In each example I have also shown the output generated by the four
main citation commands: \cs{cite}, \cs{autocite}, \cs{textcite} and
\cs{footcite}. Depending on the style, not all of these commands are
necessarily appropriate, of course.

\begin{figure}
\begin{Verbatim}[frame=single,fontsize=\small]
@book{worman,
  author       = {Worman, Nancy},
  title        = {The Cast of Character},
  date         = 2002,
  publisher    = {University of Texas Press},
  location     = {Austin},
  langid       = {english},
  langidopts   = {variant=american},
  sorttitle    = {Cast of Character},
  indextitle   = {Cast of Character, The},
  subtitle     = {Style in Greek Literature},
  shorttitle   = {Cast of Character},
}
\end{Verbatim}
\caption{A \texttt{book} entry: \texttt{worman}\label{eg:book}}
\end{figure}

\begin{figure}
\begin{Verbatim}[frame=single, fontsize=\small]
@book{aristotle:anima,
  author       = {Aristotle},
  title        = {De Anima},
  date         = 1907,
  editor       = {Hicks, Robert Drew},
  publisher    = cup,
  location     = {Cambridge},
  keywords     = {primary},
  langid       = {english},
  langidopts   = {variant=british},
}
\end{Verbatim}
\caption{A \texttt{book} with an editor: \texttt{aristotle:anima}\label{eg:book2}}
\end{figure}

\begin{figure}
\begin{Verbatim}[frame=single, fontsize=\small]
@collection{gaonkar,
  editor       = {Gaonkar, Dilip Parameshwar},
  title        = {Alternative Modernities},
  date         = 2001,
  publisher    = {Duke University Press},
  location     = {Durham and London},
  isbn         = {0-822-32714-7},
  langid       = {english},
  langidopts   = {variant=american},
}
@InCollection{gaonkar:in,
  author       = {Gaonkar, Dilip Parameshwar},
  editor       = {Gaonkar, Dilip Parameshwar},
  title        = {On Alternative Modernities},
  date         = 2001,
  booktitle    = {Alternative Modernities},
  publisher    = {Duke University Press},
  location     = {Durham and London},
  isbn         = {0-822-32714-7},
  pages        = {1-23},
}
\end{Verbatim}
\caption{An \texttt{incollection} entry: \texttt{gaonkar:in}\label{eg:incollection}}
\end{figure}

\begin{figure}
\begin{Verbatim}[frame=single, fontsize=\small]
@article{reese,
  author       = {Reese, Trevor R.},
  title        = {Georgia in Anglo-Spanish Diplomacy, 1736-1739},
  journaltitle = {William and Mary Quarterly},
  date         = 1958,
  series       = 3,
  volume       = 15,
  pages        = {168-190},
  langid       = {english},
  langidopts   = {variant=american},
}
\end{Verbatim}
\caption{An \texttt{article}: \texttt{reese}\label{eg:article}}
\end{figure}

\clearpage

\includepdf[pages={1,2}]{./examples/chicago-authordate.pdf}

\includepdf[pages={1,2}]{./examples/chicagonotes.pdf}

\includepdf[pages={1,2}]{./examples/oscola.pdf}

\includepdf{./examples/ieee.pdf}

\includepdf{./examples/apa.pdf}

\includepdf[pages={1,2}]{./examples/mla.pdf}

\includepdf{./examples/dw-authortitle.pdf}

\includepdf{./examples/dw-footnote.pdf}

\includepdf{./examples/nature.pdf}

%%% Local Variables:
%%% coding: utf-8
%%% mode: LaTeX
%%% TeX-master: "biblatex-tutorial"
%%% End:

%!TEX root=biblatex-tutorial.tex
\addcontentsline{toc}{chapter}{Index}
\printindex

%%% Local Variables:
%%% coding: utf-8
%%% mode: LaTeX
%%% TeX-master: "biblatex-tutorial"
%%% End:

\end{CJK}

\end{document}

%%% Local Variables:
%%% coding: utf-8
%%% mode: LaTeX
%%% End:
