\chapter{快速入门}\label{ch:quickstart}
\marginpar{\begin{CJK}{UTF8}{gbsn}\footnotesize 快速入门\end{CJK}}

\biblatex\ is a modern, and vastly more flexible, replacement for
\bibtex\ for those needing to automate bibliography production in
\LaTeX. These pages provide a very quick `quick start' guide. A rather
more carefully explained guide is found in
chapter~\ref{ch:introduction}, and of course much more information
throughout the guide, and in the \biblatex\ manual.
\marginpar{
\begin{CJK}{UTF8}{gbsn}\footnotesize
本章提供了最最简单的入门介绍,主要介绍参考文献生成需要准备的内容和步骤。详细入门介绍见第\ref{ch:introduction}章
\end{CJK}}


%\section{Step 1: Create a \texttt{.bib} file}
\section{第一步:准备bib文件}
\marginpar{\begin{CJK}{UTF8}{gbsn}\footnotesize 第一步 准备bib文件
\end{CJK}}
Containing the bibliographical data. \biblatex\ follows the basic
structure of \bibtex, but expands on
it.\intref{Chapter~\ref{ch:database}} An example is shown in figure~\ref{quickstart:bib}.

\begin{figure}
\strut
  \begin{minipage}[t]{0.7\linewidth}
    \ttfamily
 @book\{\colorbox{blue!15}{work:1},\tikz{\node(iskeynode)[shape=coordinate]{};}\\
  author       = \{Aristotle\},\\
  title        = \{De Anima\},\\
  date         = 1907,\\
  editor       = \{Hicks, Robert Drew\},\\
  publisher    = \{CUP\},\\
  location     = \{Cambridge\},\\
\}\\
\end{minipage}\begin{minipage}[t]{0.7\linewidth}
  \small\sffamily\strut\vspace{1pc}

  \tikz{\node(iskeycommentnode)[shape=coordinate]{};} Key: a unique
key chosen by you and used in \cs{cite} commands to refer to this
source.
\end{minipage}
\begin{tikzpicture}[overlay, line width=1pt]
  \draw[red, arrows=<-] (iskeynode) -- (iskeycommentnode);
\end{tikzpicture}
\caption{A sample \texttt{.bib} file entry}
\label{quickstart:bib}
\end{figure}

%\section{Step 2: Write source file}
\section{第二步:准备tex源文件}
\marginpar{\begin{CJK}{UTF8}{gbsn}\footnotesize 第二步 准备tex源文件
\end{CJK}}

Prepare your source file, loading the \biblatex\ package, adding your
|.bib| file as a resource, and including citations and a
bibliography. An annotated same file appears in
figure~\ref{quickstart:tex}.

\begin{figure}
\begin{minipage}[t]{0.7\linewidth}
  \ttfamily
  \cs[article]{documentclass}\\
  \vspace{1pc}
  \colorbox{blue!15}{\parbox{0.67\textwidth}{
  \cs[english][babel]{usepackage}\tikz{\node(polyglossianode)[shape=coordinate]{};}}}\\
  \vspace{2pc}
  \colorbox{blue!15}{\cs[csquotes]{usepackage}}\tikz{\node(csquotenode)[shape=coordinate]{};}\\
  \vspace{1pc}
  \colorbox{blue!15}{\parbox{0.65\textwidth}{
  \cs[T1][fontenc]{usepackage}\\
  \cs[utf8][inputenc]{usepackage}}}\tikz{\node(encodingnode)[shape=coordinate]{};}\\
  \vspace{4pc}
  \colorbox{blue!15}{\cs[\colorbox{blue!30}{\tikz{\node(optionsnode)[shape=coordinate]{};}style=authoryear}][biblatex]{\raisebox{6pt}{\tikz{\node(biblatexnode)[shape=coordinate]{};}}usepackage}}\\
   \vspace{4pc}
  \colorbox{blue!15}{\cs[\colorbox{blue!30}{\tikz{\node(bibfilenamenode)[shape=coordinate]{};}refs.bib}]{\raisebox{6pt}{\tikz{\node(addbibnode)[shape=coordinate]{};}}addbibresource}}\\
  \vspace{1pc}
  \cs[document]{begin}\\
  \vspace{2pc}
  In \colorbox{blue!15}{\cs
    [\colorbox{blue!30}{work:1}]
    {\raisebox{6pt}{\tikz{\node(citenode)[shape=coordinate]{};}}cite}} someone
  said something interesting. And also in\\
  \colorbox{blue!15}{\cs[\colorbox{blue!30}{\tikz{\node(keynode)[shape=coordinate]{};}work:2}]{cite}}.\\
  \vspace{2pc}
  \colorbox{blue!15}{\cs{printbibliography}}\tikz{\node(printbibliographynode)[shape=coordinate]{};}\\
  \vspace{1pc}
  \cs[document]{end}
\end{minipage}%
\begin{minipage}[t]{0.7\linewidth}
  \sffamily\small
  \vspace{1pc}
  \tikz{\node(langnode)[shape=coordinate]{};} It is not necessary to use \texttt{polyglossia} or \texttt{babel},
  but if they are used they should be loaded first. For more
  information on \biblatex\ and languages, see chapter
  \ref{ch:languages}.
  \vspace{1pc}

  \tikz{\node(csqoptionalnode)[shape=coordinate]{};} Optional,
  but a good idea.

  \vspace{1pc} \tikz{\node(encodingcommentnode)[shape=coordinate]{};}
  Optional (and only applicable to \textsc{pdf}\TeX). But
  desirable to enable the use of unicode-encoded input files and
  modern fonts. An alternative is to use Xe\TeX\ or Lua\TeX\, which
  handle these matters natively.

  \vspace{1pc}

  \tikz{\node(biblatexcommentnode)[shape=coordinate]{};} Load \biblatex.
  \vspace{1pc}

  Here we use the
  \texttt{style} option to choose a style for citations
  \tikz{\node(optionscommentnode)[shape=coordinate]{};}and
  bibliography. There are numerous options to customize in various ways.

  \vspace{1pc}
  \strut\raisebox{6pt}{\tikz{\node(addbibcommentnode)[shape=coordinate]{};}}
  Identify the files where the bibliographical data will be found. Can
  have multiple \cs{addbibresource} statements.

  \vspace{0.5pc}
  \tikz{\node(bibfilenamecommentnode)[shape=coordinate]{};} Include the
  full filename, including extension (conventionally
  \texttt{.bib}. For detail about the format of this file see chapter~\ref{ch:database}.

  \vspace{1.3pc}
  \tikz{\node(citecommentnode)[shape=coordinate]{};} Citations use
  \cs{cite} commands: for other possibilities see
  chapter~\ref{ch:citationcommands}

  \vspace{2.2pc}
  \tikz{\node(keycommentnode)[shape=coordinate]{};} Key as defined in
  the |.bib| file.

  \vspace{2pc}
  \tikz{\node(printbibcommentnode)[shape=coordinate]{};} Print
  the bibliography. See chapter~\ref{ch:bibliographyformat}.

\end{minipage}
\begin{tikzpicture}[overlay,line width=1pt]
  \draw[red, arrows=<-] (polyglossianode.south) -- (langnode.west);
  \draw[red, arrows=<-] (csquotenode.east) -- (csqoptionalnode.west);
  \draw[red, arrows=<-] (encodingnode.east) --
  (encodingcommentnode.west);
  \draw[red, arrows=<-] (addbibnode.south) |- (addbibcommentnode.north);
  \draw[red, arrows=<-] (bibfilenamenode.south) |-  (bibfilenamecommentnode.south);
  \draw[red, arrows=<-] (biblatexnode.north) |-
  (biblatexcommentnode.north);
  \draw[red, arrows=<-] (optionsnode.south) |-
  (optionscommentnode.south);
  \draw[red, arrows=<-] (citenode.north) |- (citecommentnode.west);
  \draw[red, arrows=<-] (keynode.south) |- (keycommentnode.south);
  \draw[red, arrows=<-] (printbibliographynode.east) -- (printbibcommentnode.west);
\end{tikzpicture}
\caption{A sample \texttt{.tex} file.}
\label{quickstart:tex}
\end{figure}

%\section{Step 3: Compile the \LaTeX\ source}
\section{第三步:编译tex文档}
\marginpar{\begin{CJK}{UTF8}{gbsn}\footnotesize 第三步 编译tex文档
\end{CJK}}

Compile the \LaTeX\ source file, using |pdflatex|, |xelatax|, or
|lualatex|.

%\section{Step 4: Run \package{Biber}}
\section{第四步:运行 biber}
\marginpar{\begin{CJK}{UTF8}{gbsn}\footnotesize 第四步 运行 biber \end{CJK}}

Run the \package{Biber} program to prepare bibliographical data. Do
this by running
\begin{pseudoverb}
  \centering
  biber \angled{basename}\\
  {\normalfont\itshape or} \\
  biber \angled{basename}.bcf
\end{pseudoverb}

%\section{Steps 5 \ldots: Run \LaTeX\ at least once more}
\section{第五步 至少再运行一次\LaTeX }
\marginpar{\begin{CJK}{UTF8}{gbsn}\footnotesize 第五步 至少再运行一次\LaTeX\ \end{CJK}}

Run \LaTeX\ at least once, and sometimes twice more. It will read in
the citation data that \package{Biber} has prepared, produce the
bibliography and citations, and prepare the final text.


%%% Local Variables:
%%% coding: utf-8
%%% mode: latex
%%% TeX-master: "biblatex-tutorial"
%%% End:
