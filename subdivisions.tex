\chapter{Multiple Bibliographies}\label{ch:subdivisions}
\index{bibliography!multiple!see multiple bibliographies}

We now come to consider the various methods of producing
\emph{multiple} bibliographies in a single TeX document. The detail
can become overwhelming unless you first understand the logic, so that
is where we will start.

There are two main patterns to split bibliographies:

\index{multiple bibliographies!general types}
\begin{enumerate}
\item Division that is determined by characteristics of the
  \emph{source}, where some types of source get put in one bibliography
  and other types of source get put in another. The dividing line
  might be between primary and secondary sources. It might be between
  works about foology and everything else (`topic based'). It might
  be between the most important works and the rest. It might be
  between the works of Dr Drofnats and everyone else's works. But
  however the dividing-line is drawn, it is based on something about
  the underlying source, which is why I'm going to call these
  \emph{source-based} divisions.
\item Division that is determined by the source's \emph{citation}, and
  in particular by \emph{where} the source is cited --- so that, for
  instance, there are separate bibliographies for Chapter 1, Chapter
  2, and so forth. In an `extreme form' the bibliographies need to
  be completely self-contained; that's often needed, for instance, if
  a single document is being used to typeset papers by different
  authors into a collection or journal issue. I'm going to call these
  \emph{document-based} divisions.
\end{enumerate}
These categories are not, of course, mutually exclusive. You
\emph{could} have, in a single document, both subdivision by source
(say into primary and secondary sources) and subdivision by location
in the document.

\biblatex\ provides mechanisms to handle both types of division. It is
designed to be comprehensive. Before \biblatex\ there were many
competing packages (\package{multibib}, \package{splitbib},
\package{chapterbib}, \package{bibunits}, \package{bibtopic}) which
provided these facilities with biblatex. These are \emph{not}
compatible with \biblatex; you cannot use them together --- and you
don't need to.

\section{Source-based division}

\indexstart{multiple bibliographies!source-based division}
In principle, source-based division is simple. You provide some way for
\biblatex\ to identify the different types of source, and then you print
separate bibliographies for each. Although the most interesting features
are those that determine how works are selected, it's convenient to start
by explaining how the actual printing is handled.
\index{multiple bibliographies!source-based division!printing}

\begin{pseudoverb}
\cs{printbibliography}[title="Primary Sources",
                       \angled{selector}]
\cs{printbibliography}[title="Secondary Literature",
                       \angled{selector}]
\end{pseudoverb}
Usually in these cases, however, the separate bibliographies are seen
as `sub-bibliographies', in which case a little bit more work is
done to get the formatting right:

\begin{pseudoverb}
\cs{printbibheading}
\cs{printbibliography} [heading=subbibliography,
                      title="Primary Sources",
                      filter=\angled{selector}]
\cs{printbibliography} [heading=subbibliography,
                      title="Secondary Literature",
                      filter=\angled{selector}]
\end{pseudoverb}
We use \cs{printbibheading} simply as a convenience to print a
standard `top level' heading for the whole bibliography, but then use
the |heading| and |title| options when actually printing the
bibliography. As far as \biblatex\ is concerned, a document can contain
as many \cs{printbibliography} commands as you
like, and each can contain as many or as few sources as you like.  The
real trick lies in how you teach \biblatex\ to distinguish between
different sources. In all cases, the method used is to add an option
to the \cs{printbibliography} command which tells
\biblatex\ what filter to use.\csindex{printbibliography}

To simplify somewhat there are three basic ways you can do that, and a
convenient way of combining them. The basic ways involve:
\emph{type}, \emph{keywords} and \emph{categories}. The
combining device is the \emph{filter}, which enables complex
assemblies of conditions to be defined and used simply.

\index{multiple bibliographies!source-based division!selection}
\paragraph{How \biblatex\ chooses.} All these distinctions are brought
into operation using an option in the bibliography heading. These
follow a standard pattern:
\begin{quote}
\angled{selector} \texttt{=} \angled{value}
\end{quote}
(e.g. \texttt{type=book}) to include only books.

If you prefer to look on the negative side:
\begin{quote}
\texttt{not}\angled{selector} \texttt{=} \angled{value}
\end{quote}
(e.g. \texttt{nottype=online}): exclude from the bibliography entries of
this type). You can have multiple `excluders', which are applied
cumulatively: \texttt{{[}nottype=online,} \texttt{nottype=misc{]}} will
\emph{exclude} both \texttt{online} and \texttt{misc} entry types.

\index{multiple bibliographies!source-based division!by type}
\newthought{The easiest type of choice} to build in to a bibliography
is between different entry types. Every source necessarily has an
entry type, and this can be a convenient way of producing some kinds
of subdivided bibliography. For instance, if you wanted to distinguish
between printed sources (the bibliography) and online sources (the
`webography') it might be sufficient to distinguish between. Here you
can use:
\begin{itemize}
\item \texttt{type}, to include (only) particular entrytypes, e.g.
  \texttt{type=book} to include only books. You can only have
  \emph{one} positive type selector. If you want to include, say
  |book| and |article| types, you will need to define a |bibfilter|
  for that purpose;
\item \texttt{nottype}, to exclude particular entrytypes,
  e.g. \texttt{nottype=book} to exclude books. You can always have
  \emph{more than one} excluder, without the need to define a
  |bibfilter|.
\item \texttt{subtype}, to include particular entrysubtypes (some
  bibliographic styles make use of subtypes to further divide types,
  or to specialised types such as \texttt{misc})
\item \texttt{notsubtype}, to exclude particular entrysubtypes.
\end{itemize}
(The reason you can have multiple excluders, but not multiple
includers, is that generally \biblatex\ treats conditions as
cumulative. So (|nottype=book| and |nottype=article|) works to
identify sources which are neither books nor articles. But
(|type=book| and |type=article| `work' to identify sources which are
both books and articles. Inevitably, nothing matches.)

So, for example, we could build a separate `webography' and
`bibliography' along these lines:

\begin{Verbatim}
\printbibheading
\printbibliography[heading=subbibliography,
                   title="Webography",
                   type=online]
\printbibliography[heading=subbibliography,
                   title="Paperography",
                   nottype=online]
\end{Verbatim}

\newthought{However, type of source is often not sufficient}: many
potentially interesting divisions (such as between primary and
secondary sources, or by subject) cut across source types. The trouble
here is that, for obvious reasons, \biblatex\ is not going to know
about the underlying subject-matter of a source or whether it is
primary or secondary and so forth unless you tell it. There are two
ways to do this.

\index{multiple bibliographies!source-based division!keywords}
First, you can store the information \emph{in the .bib file} using the
  \emph{keywords} field,\index{database!keyword field@\texttt{keyword} field} and then filter inclusion in a particular
  bibliography using the keyword and notkeyword instructions. So, for
  instance, you might mark every piece of primary literature using the
  keyword `primary' in the \texttt{.bib} file. You could then
  distinguish between primary literature and secondary sources as
  follows:
\begin{Verbatim}
  \printbibliography[title = "Primary Sources",
                     keyword = primary]
  \printbibliography[title = "Secondary Material",
                     notkeyword = primary]
\end{Verbatim}

\index{multiple bibliographies!source-based division!categories}
That, however, is not very flexible.\manref{\S~3.12.4} The alternative approach allows
document-by-document flexibility. You can enter the information in the
\emph{document source (}|.tex|\emph{) file} using the
\emph{categories} system. To do this:
\begin{itemize}
\item You first define a category in the preamble of the document, and
  then use a command to identify which sources are within a particular
  category. For instance, suppose you wanted to print `crucial' works
  separately in some notes for your students, you would create that
  crucial category in your preamble:\csindex{DeclareBibliographyCategory}
  \begin{pseudoverb}
    \centering
    \cs{DeclareBibliographyCategory}\{crucial\}
  \end{pseudoverb}

\item You then add individual sources to this category anywhere in the
  document using \cs{addtocategory}, which takes two arguments: the
  category and a list of sources to be added to that category.\csindex{addtocategory}

  \begin{pseudoverb}
    \centering
    \cs{addtocategory}\{crucial\}\{\angled{key1}, ... \angled{key2}\}
  \end{pseudoverb}

You can have multiple keys and multiple
\texttt{\textbackslash{}addtocategory\{\}\{\}} commands as well.

\item Then, when it comes to printing the bibliography, you control what is
printed using \texttt{category=\angled{category}} and
\texttt{notcategory=\angled{category}}.
\begin{Verbatim}
\printbibliography[heading = "Key sources",
                   category = crucial]
\printbibliography[heading = "Other sources",
                   notcategory = crucial]
\end{Verbatim}
\end{itemize}

Both keywords and categories are ways of arbitrarily allocating
sources to groups. The difference between them is that keywords are
defined in the database, and categories in the document where the
database is used. This should give you a hint about the best way of
using them. A keyword should really say something permanently true
about a source: they are a pretty reasonable candidate for dividing
sources into primary and secondary, or original and translated, or
identifying their topic as foology rather than barography. Categories
need only be true for the particular document: they are a good
candidate for specifying a source as `important' in a set of student
notes for instance, or for quick-and-dirty work where you don't want
to alter a database file permanently.

\subsection{Combining criteria with user-defined filters}

\index{multiple bibliographies!source-based division!filters}
Suppose you want to combine filters in complicated ways: for instance
you want a sub-bibliography which contains only crucial primary
sources which are not online. For such purposes, one reaches for a
user-defined filter. Such a filter is defined using
\begin{pseudoverb}
  \centering
  \cs{defbibfilter}\{\angled{name}\}\{\angled{definition}\}
\end{pseudoverb}
\csindex{defbibfilter}
This enables you to combine the type, subtype, keyword and category
filters using logical operators (and, or, not) and parentheses for
grouping. Having defined the filter, it can then be used to construct
a bibliography with the option |filtername=|\angled{filter}

So, for example, we could define the selection filter mentioned above
as follows:

\begin{Verbatim}
\defbibfilter{croffprim}{
    keyword=primary
    and category=crucial
    and not type=online}
\end{Verbatim}
And we could the use it:

\begin{Verbatim}
\printbibheading
. . .
\printbibliography[heading=subbibliography,
                   title="Crucial offline primary sources",
                   filter=croffprim]
\end{Verbatim}

\subsection{Automation}

\index{multiple bibliographies!source-based division!automation}
Both keyword and category filters ultimately depend on direct user
intervention, key-by-key. Is there any way to automate this? For
instance, could you automatically assign (say) all the sources in a
particular \texttt{.bib} file to a given category? Or could could
assign all the works of a particular author?

The answer is that you often can do so. But it's rather an advanced
topic. Besides, the range of possible uses and circumstances is very
wide. This section, therefore, is going to sketch some ideas, rather
than provide a set of recipes.

\begin{itemize}
\item Define a low-level user-defined filter using
  \cs{defbibcheck},\csindex{defbibcheck} which then uses the
  \texttt{check=\angled{filter}} filter. For instance the \biblatex\
  manual\manref{\S~3.7.9} demonstrates using such a mechanism to
  examine the year of an entry and print a bibliography of `recent'
  literature.
\item Use\manref{\S~4.5.3} a |sourcemap| to add keywords automatically. This could easily be done, say,
  for a particular file, or for a particular author's work; you could
  then use a keyword filter.
\item Use\manref{\S~4.10.6} the
  \texttt{\textbackslash{}AtEveryCitekey} hook to execute code at
  every citation to check data and decide whether to add that key to a
  category.\csindex{AtEveryCitekey}
\item Use\manref{\S~4.10.6} the \texttt{\textbackslash{}AtDataInput}
  to execute code as each entry is read in and decide whether to add
  it to a category.
\end{itemize}
\indexstop{multiple bibliographies!source-based division}

\section{Document-based division}
\indexstart{multiple bibliographies!document-based division}

\biblatex\ offers two main devices for producing bibliographies that are
categorised based on where the citation is used in the LaTeX source
file:

\index{refsection!general description}
\index{refsegment!general description}
\begin{itemize}
\item A \emph{refsection} is (nearly) a fully self-contained `unit'
  which might have its own |.bib| files, and is expected to produce a
  basically self-contained bibliography. It is the obvious choice, for
  instance, if producing a collection of papers, each of which had a
  different author who had used his or her own \texttt{.bib} file,
  with its own keys.
\item A \emph{refsegment} is a less strongly-marked division, which
  represents a part of what is recognisably a larger work: a way of
  dividing a \emph{single} bibliography up for convenience. It might
  be the right choice if you wanted to produce both summary
  bibliographies by chapter and a complete bibliography for the whole
  document.
\end{itemize}
The conceptual difference between the two is best understood if you
imagine a bibliography style which uses numerical labels. If the same
work is cited in two different \emph{refsections} then it will
expected to appear in two different bibliographies with a
\emph{different} number: each bibliography is entirely self-contained,
and the labels are unique within each but not between each. If, on the
other hand, the same work is cited in two different \emph{refsegments}
then it will have a single, unique, label which will be common to
both.  The bibliography for each segment may print all and only the
sources referred to in that segment; but although a source that is
referred to in two different segments will be printed in both
segments' bibliographies, it will have the \emph{same} label in each.

Another key difference is that whereas \emph{refsegments} share the
same |.bib| files, \emph{refsections} need not.  Some resources can be
defined so as to be global, but individual refsections can also have
their own `private' resources. Why does this matter? Suppose that Dr
Drofnats and Dr Frobwangler each writes a paper for a collection. In
their field, the seminal work is Mangel-Wurzel's famous paper, `How to
Foo Bars'. Dr Drofnats has this paper in his personal |.bib| file as
`|mw:foo|'. Dr Frobwangler has it in her personal |.bib| file as
`|wurzel:bars|'. Each cites accordingly.

One possibility of course is to produce a unified |.bib| file for the
whole document, and edit the citations in each paper. And sometimes
that might be necessary. But if the bibliographies are \emph{truly}
self-contained and free-standing, it should be possible to use Dr
Drofnats' file for his paper, and Dr Frobwangler's file for hers,
without the trouble and risk of error that there is if files are
combined.

\subsection{Identifying segments and sections}

\index{multiple bibliographies!document-based division!refsegment}
\index{multiple bibliographies!document-based division!refsection}
Obviously, \biblatex\ needs to know where each segment or section begins
and ends. There are three ways you can do this:

\index{refsection environment@\texttt{refsection} environment}
\index{refsegment environment@\texttt{refsegment} environment}
\begin{enumerate}
\item
  \begin{marginfigure}
  \fbox{
  \begin{minipage}{0.9\marginparwidth}
  \texttt{\textbackslash begin\{refsection\}}\\
  \colorbox{red!50}{[section 1]} \\
  \texttt{\textbackslash end\{refsection\}} \\
  \texttt{\textbackslash begin\{refsection\}}\\
  \colorbox{blue!50}{[section 2]} \\
  \texttt{\textbackslash end\{refsection\}}
  \end{minipage}}
  \end{marginfigure}
  Most explicitly using
  |\begin{refsection}| \ldots |\end{refsection}|
  or
  |\begin{refsegment}| \ldots |\end{refsegment}|
.
\item
  \csindex{newrefsegment}\csindex{newrefsection}
  \begin{marginfigure}
  \fbox{
  \begin{minipage}{0.9\marginparwidth}
  \texttt{\textbackslash newrefsection}\\
  \colorbox{red!50}{[section 1]} \\
  \texttt{\textbackslash newrefsection}\\
  \colorbox{blue!50}{[section 2]} \\
  \texttt{\textbackslash endrefsection}
  \end{minipage}}
  \end{marginfigure}
  Using the commands |\newrefsection| and |\newrefsegment|. Each of
  these ends the existing refsegment or refsection (if need be) and
  starts a new one. You can mark the end of refsegments or refsections
  with |\endrefsegment| or |\endrefsection|.
\item By telling \biblatex\ automatically to start a new refsegment or
  refsection whenever a new part, or chapter, or section begins. To do
  this you pass the option |part|, |chapter|, |section| or |subsection| as a
  value for the refsection or refsegment option when loading \biblatex.
  So to make each chapter a free-standing section, you would put
  \begin{marginfigure}
  \fbox{
  \begin{minipage}{0.9\marginparwidth}
  \texttt{\textbackslash chapter\{One\}}\\
  \colorbox{red!50}{[section 1]} \\
  \texttt{\textbackslash chapter\{Two\}}\\
  \colorbox{blue!50}{[section 2]} \\
  \texttt{\textbackslash endrefsection}
  \end{minipage}}
  \end{marginfigure}
\begin{Verbatim}
  \usepackage[...
              refsection=chapter]{biblatex}
\end{Verbatim}
\end{enumerate}

\subsection{Identifying resources for refsections}

\index{refsection!database}\index{refsegment!database}
Refsegments share the bibliography resources (|.bib| files) of the
entire document. Refsections can have their own resources. The full
details are rather complex. What follows is a simplified set of
instructions.

\begin{enumerate}
\item If you want one bibliography database or databases --- whether
  that be undivided, or divided into segments, or divided into
  sections but with the same |.bib| files used in all sections --- then
  just use |\addbibresource{}| in the preamble.
  \csindex{addbibresource}
\item If you are using refsections and you want each to have its
  \emph{own} |.bib| file(s), then proceed as follows:
  \begin{itemize}
  \item In the preamble, use |\addbibresource{}| to identify each
    resource, but make use of the optional label argument to give each
    a label:
\begin{Verbatim}
  \addbibresource[label=drofnats]{./bibfiles/drofnats.bib}
  \addbibresource[label=frobwangler]{./bibfiles/frobwangler.bib}
\end{Verbatim}
\item Use explicit refsection commands.\footnote{Either
    \cs[refsection]{begin} and \cs[refsection]{end} or
    \cs{newrefsection}.} As each section begins, include a list of the
  labels of the applicable resources as an optional argument:
\begin{Verbatim}
  \newrefsection[drofnats]
  ... 
  \newrefsection[frobwangler]
\end{Verbatim}
\item If there are |.bib| files that you want to have available in all
  sections then either identify them as shown above
\begin{Verbatim}
  \addbibresource[label=everyone]{./bibfiles/common.bib}
  ...
  \newrefsection[drofnats,everyone]
\end{Verbatim}
or, instead of \cs[\angled{filename}]{addbibresource} use
\cs[\angled{filename}]{addglobalbib} in the preamble: this will
declare a resource which is automatically made available to every
refsection. So
\begin{Verbatim}
  \addglobalbib{./bibfiles/common.bib}
\end{Verbatim} 
will make |common.bib| accessible to every refsection, without any
need to include it in the optional argument.\csindex{addglobalbib}
\end{itemize}
\end{enumerate}
\indexstop{multiple bibliographies!document-based division}

\subsection{Printing the bibliographies}

To print a bibliography for a single refsection, you do one of two
things. First, you can explicitly identify the section for which you
want the bibliography printed. You do this using the section's
\emph{number}:

\begin{Verbatim}
\printbibliography[section=1]
\end{Verbatim}

What is this number? Basically, all the sections you create (with
|\begin{refsection}...\end{refsection}|, or |\newrefsection|, or
implicitly at the start of chapters and so on) are numbered, starting
with 1. (Anything that happens \emph{before} the first section or
\emph{after} the last is regarded as belonging in a section numbered
0.) So you can specify the section whose bibliography you want
printed.

You can also do this for refsegments too in just the same way: the
first segment you create is refsegment 1, the next is 2 and so on --
and you can choose which gets printed:

\begin{Verbatim}
\printbibliography[segment=2]
\end{Verbatim}

With \emph{refsections only} (\emph{not} with segments) there is
another way. If you include a |\printbibliography| command
\emph{within} a refsection, \biblatex\ will assume that you only want
to include that section.

\begin{figure*}
\begin{minipage}[b]{0.5\textwidth}
\ttfamily
\cs[book]{documentclass}\\
\cs[biblatex]{usepackage}\\
\cs{addbibresource}[label=chapter1]\tikz{\node(labelnode)[shape=coordinate]{};}\{chaptera.bib\}\\
\cs{addbibresource}[label=chapter2]\{chapterb.bib\}\\

\cs[...]{chapter}\\ 
\colorbox{green!15}{\parbox{0.9\textwidth}{
\cs{newrefsection}\,\tikz{\node(refsectnode)[shape=coordinate]{};}[chapter1]\\[2ex]
... [lots of citations] ...\\[2ex]

\cs{printbibliography}\tikz{\node(pbnode)[shape=coordinate]{};}[heading=subbibliography]\\[2ex]
}}
\cs[...]{chapter}\\ 
\cs{newrefsection}[chapter2]\\[2ex]

... [lots of citations] ...\\[2ex]

\cs{printbibliography}[heading=subbibliography]
\end{minipage}
\begin{minipage}[b]{0.5\textwidth}
\sffamily
\tikz{\node(resnode) [text width=5cm] {resources are labelled: here, \texttt{chapter1} refers to \texttt{chaptera.bib}};}\strut\\[4ex]

\tikz{\node(exprefnode) [text width=5cm] {a refsection: within this section the resource indicated by the label is used --- so, here, citations are taken from \texttt{chaptera.bib}};}\strut\\[2ex]

\tikz{\node(exppbnode) [text width=5cm] {only the bibliography for this refsection will be printed};}

\vspace{9ex}
\end{minipage}
\begin{tikzpicture}[overlay,line width=1pt]
\draw[red] (resnode.south) -| (labelnode.north);
\draw[red] (exprefnode.north west) -| (refsectnode.east);
\draw[red] (exppbnode.north) -| (pbnode.south);
\end{tikzpicture}
\caption{Exemplary refsections\label{refsections}}
\end{figure*}

Finally, because a possible use case is to loop through each
section or segment, printing the bibliography for that section or
segment, and then doing the same for the next, there are special
commands designed to achieve this: |\bibbysection| and
|\bibbysegment|. They are `smart' in that they will not print a
bibliography if a particular segment or section contains no citations
at all.\csindex{bibbysection}\csindex{bibbysegment}

If you want \emph{both} subdivided bibliographies and a main,
consolidated, bibliography, there are two approaches. (1) You can use
|refsegments|. If you then print a bibliography with \emph{no}
|segment=...| selector, all segments will be printed. (2) If
\emph{only} your cited sources are in the |.bib| file, you can use a
|\nocite{*}| instruction within a |section| to print
everything. Unfortunately there is no way to tell \biblatex\ to print
a combined bibliography of multiple sections.

\subsection{Some advice}

\index{multiple bibliographies!considered harmful}
In many cases, excessively divided bibliographies are more harmful
than helpful. Before deciding to divide bibliographies at all, do
think seriously about how a reader uses a bibliography. Don't
introduce unnecessary complexity. Apart from the most obvious case
(self-contained papers which need their own bibliographies, for
instance) subdivision can easily be over done.

%%% Local Variables:
%%% coding: utf-8
%%% mode: LaTeX
%%% TeX-master: "biblatex-tutorial"
%%% End: