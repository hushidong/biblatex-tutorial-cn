%\chapter{Recipes}
\chapter{实用技巧}\label{ch:recipes}

A comprehensive guide to customization is outside the scope of this
book. But some requests for customization are very common, and it
seems worthwhile to have a short chapter that provides some common
recipes.

%\section{Names}
\section{姓名相关的技巧}

\index{customization!names}
\paragraph{I want at most 1/2/3/4 names printed before `and
  others'.}\label{recipes:maxnames} You need to change the value of
|maxbibnames|, |maxcitenames|, |minbibnames| and/or |mincitenames|. As
you might expect |maxbibnames| and |minbibnames| deal with what is
printed in the bibliography, and |maxcitenames| and |mincitenames|
deal with what is printed in a citation (if citations print
names). You can change both with |maxnames| and |minnames|, which is
what you normally want to do, and what we will do here. You set this
value when loading \biblatex:
\begin{pseudoverb}
\cs{usepackage}[maxnames=2, minnames=1]\{biblatex\}
\end{pseudoverb}

The |maxnames| number holds the number of names that will
\emph{trigger} truncation to `and others' (or `et al', or
whatever). The |minnames| number is the number of names \emph{to which
  the list will be truncated if truncation occurs} (which must,
obviously, be no more than the number that triggers truncation).

Suppose we had an entry with four authors:
\begin{Verbatim}
...
author = { Alpha, A. and Beta, B.
           and Gamma, C. and Delta, D. }
...
\end{Verbatim}

The effect of various different settings of |maxnames| and |minnames|
is shown in table \ref{maxnames}.

\begin{table}
\begin{tabularx}{\textwidth}{llX}
\toprule
\texttt{maxnames} & \texttt{minnames} & \textsf{result} \\
\midrule
1                 &  1                & A.\ Alpha et al \\
2                 &  1                & A.\ Alpha et al \\
2                 &  2                & A.\ Alpha, B. Beta, at al \\
3                 &  1                & A.\ Alpha et al \\
3                 &  2                & A.\ Alpha, B. Beta, et al \\
3                 &  3                & A.\ Alpha, B. Beta, C. Gamma, et al \\
4                 &  any              & A.\ Alpha, B. Beta, C. Gamma, and D. Delta \\
\bottomrule
\end{tabularx}
\caption{Effect of \texttt{maxnames} and \texttt{minnames}\label{maxnames}}
\end{table}

Sometimes you will find that you get results which are not what you
are expecting. When it truncates a name, \biblatex\ can be configured
to consider whether the resulting list is unique. If it isn't,
\biblatex\ may add extra names in order to produce a unique list. For
example, suppose you are citing one work by Tom, Dick, and Harry; and
one work by Tom, Dick, and Harriet, and have set |maxnames| to 2 and
|minnames| to 1. Simply applying the ordinary rules, each work would
end up with the authors given as `Tom et al.' But in some styles it
may sometimes make sense to bend the rules if by doing so it is
possible to produce a `unique' citation. This is done by setting the
option |uniquelist=true|.\manref{3.6.1} In fact many standard styles
(such as author/year and author/title styles \emph{already do
  that}. So if you think that \biblatex\ is `disobeying' your
instructions about truncation, you might try adding the option
|uniquelist=false| when loading it.

\index{customization!et al@\emph{et al}}
\paragraph{I want something different from `et al' printed when names
  are truncated.} What gets printed in this case depends on the
setting of two bibliography strings, and one delimeter:
\begin{itemize}
\item The bibstring |andothers| (by default `et al')
\item The delimiter |andothersdelim| (by default ` ')
\item The |\finalandcomma| (printed wherever a list would end in
  `and')
\end{itemize}
So to print `\& al.' rather than `, et al', one would put
\begin{Verbatim}
\DefineBibliographyStrings{english}% or your language
  { andothers = {\& al\adddot}}
...
\renewcommand{\finalandcomma}{}
\end{Verbatim}
The redefinition of |\finalandcomma| needs to come \emph{after}
|\begin{document}|.

Things are a bit more complicated if you want to redefine |et al| in
such a way that it will be different depending on how truncated the
name is: for instance if you wanted `and others' if more than two
names were truncated, but `and another' if only one was. For that
purpose you would need the more extensive changes shown in figure
\ref{andothers}.
\begin{figure}
\begin{Verbatim}[frame=single]
\NewBibliographyString{andanother}
\DefineBibliographyStrings{english}% or your language
{ andothers = {and others},
  andanother = {and another}}
\renewbibmacro*{name:andothers}{%
  \ifboolexpr{
    test {\ifnumequal{\value{listcount}}{\value{liststop}}}
    and
    test \ifmorenames
  }
    {\ifnumgreater{\value{liststop}}{1}
       {\finalandcomma}
       {}%
     \ifnumgreater{\value{listtotal} - \value{liststop}}{1}
       {\andothersdelim\bibstring{andothers}}
       {\andothersdelim\bibstring{andanother}}
    }
    {}}
\end{Verbatim}
\caption{Introduction of `and others' and `and another'\label{andothers}}
\end{figure}

\index{customization!dashes in bibliography}\index{dash!in bibliography|see{customization, dashes}}
\paragraph{I want/don't want multiple references to a single author to
  appear as a long dash.} In the author/year, author/title and verbose
styles, you can control this with the option |dashed=true| or
|dashed=false| when loading \biblatex. Non-standard styles are not
obliged to provide this option, but most will do so.

\index{customization!initials}
\paragraph{I want to have all names abbreviated to initials.} Set the
option |giveninits=true| when loading \biblatex.

\index{customization!initials}
\paragraph{I want to have initials printed in a different way.} By
default, the standard styles of \biblatex\ print initials with spaces
and abbreviation dots:
\begin{quote}
  P.~M.~Stanley
\end{quote}
But other methods are available. If you set |terseinits=true| the initials are printed in a condensed way:
\begin{quote}
  PM~Stanley
\end{quote}
If you want a `halfway house', you can redefine |\bibinitperiod|. By combining |terseinits| with |\renewcommand{\bibinitperiod}{\adddot}| you can produce
\begin{quote}
  P.M. Stanley
\end{quote}
While by combining |terseinits=false| with |\renewcommand{\bibinitperiod}{}| you can produce
\begin{quote}
  P~M~Stanley
\end{quote}

There is, however, a catch. None of these changes will operate unless
\biblatex\ is producing the initials. In other words, none works
unless the |giveninits| option is in effect. Where names are being
printed in full, initials that you have entered will be printed (as
entered) with full stops and spaces, regardless of the settings of
|terseinits| or |\bibinitperiod|. (There are ways around this, but
they are beyond the scope of this section: the curious might look at
the \package{\textsc{oscola}} package code, which does something along these lines.)

\index{customization!names}
\paragraph{I want names printed in bold/italic/small capitals etc.}
How names are printed depends on four standard commands:
\cs{mkbibnamegiven} (which prints the initials or first names),
\cs{mkbibnameprefix} (which prints the `de' or `von' parts),
\cs{mkbibnamefamily} (which prints the last name) and
\cs{mkbibnamesuffix} (which prints the `Jr' part). Each or all of these
can be redefined to alter the way names are printed. Each takes a
single argument (the relevant part of the name), and can be used to
format it.\footnote{For formatting emphatically or in bold, you should
  use \cs{mkbibbold} or \cs{mkbibemph}}.

Suppose, for instance, that we had the following (peculiar)
requirements: We want first names printed in bold, we want `von' and
`Jr' parts printed emphatically, and we want the last name in small
capitals. We define the following
\begin{Verbatim}
\renewcommand{\mkbibnamegiven}[1]{\mkbibbold{#1}}
\renewcommand{\mkbibnameprefix}[1]{\mkbibemph{#1}}
\renewcommand{\mkbibnamesuffix}[1]{\mkbibemph{#1}}
\renewcommand{\mkbibnamefamily}[1]{\textsc{#1}}
\end{Verbatim}
And obtain results as shown in figure \ref{namestyle}.
\begin{figure}
\begin{minipage}[t]{3in}
 \tikz[thick]{
 \node(givenname)[fill=red!40, xshift=25ex]{\strut\texttt{Klaus}};
 \node(prefixname)[fill=blue!40, xshift=32ex]{\strut\texttt{von}};
 \node(familyname)[fill=green!40, xshift=39.5ex]{\strut\texttt{Bulow}};
 \node(suffixname)[fill=orange!40, xshift=46ex]{\strut\texttt{III}};
 \node(mkbibgiven)[fill=red!40,yshift=-10ex]{\cs[Klaus]{mkbibnamegiven}};
 \node(mkbibbold)[fill=red!40,yshift=-20ex]{\cs[Klaus]{mkbibbold}};
 \node(boldklaus)[fill=red!40,yshift=-30ex,xshift=25ex]{\strut\textbf{Klaus}};
 \node(mkbibprefix)[fill=blue!40, xshift=19ex, yshift=-15ex]{\cs[von]{mkbibprefix}};
 \node(mkbibemph1)[fill=blue!40, xshift=19ex, yshift=-25ex]{\cs[von]{mkbibemph}};
 \node(emvon)[fill=blue!40, xshift=32ex, yshift=-30ex]{\emph{\strut von}};
 \node(mkbibfamily)[fill=green!40, xshift=39.5ex, yshift=-10ex]{\cs[Bulow]{mkbibnamefamily}};
 \node(scname)[fill=green!40, xshift=39.5ex, yshift=-20ex]{\cs[Bulow]{textsc}};
 \node(scbulow)[fill=green!40, xshift=39.5ex, yshift=-30ex]{\strut\textsc{Bulow}};
 \node(mkbibsuffix)[fill=orange!40, xshift=55ex, yshift=-15ex]{\cs[III]{mkbibsuffix}};
 \node(mkbibemph2)[fill=orange!40, xshift=55ex, yshift=-25ex]{\cs[III]{mkbibemph}};
 \node(emiii)[fill=orange!40, xshift=46ex, yshift=-30ex]{\strut\emph{III}};
 \path[-stealth] (givenname.west) edge [out=180, in=90] (mkbibgiven.north);
 \path[-stealth] (mkbibgiven.south) edge [out=-90, in=90] (mkbibbold.north);
 \path[-stealth] (mkbibbold.south) edge [out=-90, in=180] (boldklaus.west);
 \path[-stealth] (prefixname.south) edge [out=-90, in=90] (mkbibprefix.north);
 \path[-stealth] (mkbibprefix.south) edge [out=-90, in=90] (mkbibemph1.north);
 \path[-stealth] (mkbibemph1.east) edge [out=0, in=90] (emvon.north);
 \path[-stealth] (familyname.south) edge [out=-90, in=90] (mkbibfamily.north);
 \path[-stealth] (mkbibfamily.south) edge [out=-90, in=90] (scname.north);
 \path[-stealth] (scname.south) edge [out=-90, in=90] (scbulow.north);
 \path[-stealth] (suffixname.east) edge [out=0, in=90] (mkbibsuffix.north);
 \path[-stealth] (mkbibsuffix.south) edge [out=-90, in=90] (mkbibemph2.north);
 \path[-stealth] (mkbibemph2.south) edge [out=-90, in=0] (emiii.east);
 }
\end{minipage}
\caption{How names are styled\label{namestyle}}
\end{figure}

Of course, it's usually unnecessary to have so many different
definitions. A much more common approach would be to have all the
various parts of a name formatted in the same way. But the flexibility
is there if you need it.\intref{Biblatex Manual § 12}

\index{customization!names}
\paragraph{I want the names to be printed `Firstname Lastname', but
  I'm getting `Lastname, Firstname'} Although there is, in theory,
infinite flexibility, in practice there are three common patterns to
how names are printed:
\begin{itemize}
\item In `ordinary' order: given name then family name --- Joe Bloggs
  and John Doe.
\item In `reverse' order: family name then given name --- Bloggs, Joe
  and Doe, John.
\item With the given name in the list reversed, but the other in
  ordinary order --- Bloggs, Joe and John Doe.
\end{itemize}
Styles generally set these sensibly, but if you do not like the
choice, you can usually change it without much difficulty. The command
you need is
\begin{pseudoverb}
  \cs[sortname]{DeclareNameAlias}\{\angled{order}\}
\end{pseudoverb}
Where \angled{order} can be |given-family| (for ordinary order),
|family-given| (for reversed order) or |family-given/given-family| (for the
hybrid form). Some styles reset this at the start of the bibliography,
in which case you \emph{also} need (in your preamble)
\begin{pseudoverb}
\cs{AtBeginBibliography}\{\cs[...]{DeclareNameAlias}\}
\end{pseudoverb}

\index{customization!list of own papers}\index{CV!list of papers}
\paragraph{I want to produce a publication list of my own papers.}
People writing CVs often want to produce a bibliography of their own
papers, but modifying the name format so that instead of placing their
name wherever it might happen to be in the list (it could be with `and
others'!) the list reads `with \ldots'. For this purpose you can use
the \package{biblatex-publist} style, which is specifically designed
for it. You tell the package your name (using the \cs{omitname}) macro
(or package option).

\indexstart{punctuation}\index{customization!punctuation|see{punctuation}}
%\section{Punctuation issues}
\section{标点问题}\label{sec:punctuation}

There are many configurable punctuation marks in \biblatex---too many
to deal with comprehensively here. In this section I'm only going to
cover some particularly common cases.\marginnote[-8ex]{In general, I use the
  technique of redefining macros. An alternative method, using
  \cs{DeclareDelimFormat} is also possible, and may be better if you are
  writing a complex style. For an example see p
  \pageref{declaredelim}.}

\index{punctuation!before postnote}
\paragraph{I want different punctuation before a postnote.}
\begin{marginfigure}
\fbox{\includegraphics{examples/punctcite.pdf}}
\vspace{3pt}%
\caption{Citation punctuation}\label{punctcite}
\end{marginfigure}
The overall punctuation of a citation is shown in figure
\ref{punctcite}. The point marked (1) is the place between the
pre-note and the beginning of the citation where \biblatex\ will
insert \cs{prenotedelim}. Here that command is defined to print
nothing, so in fact nothing gets inserted. The point marked (2) is the
place between the citation and the post-note where \biblatex\ will
insert the \cs{postnotedelim}. In this case that is a spaced comma.

So if we redefine both these commands, we can alter what gets
inserted:\begin{marginfigure}[12ex]
\fbox{\includegraphics{examples/punctcite2.pdf}} \vspace{3pt}%
\caption{Citation punctuation revised}\label{punctcite2}
\end{marginfigure}
\begin{Verbatim}
     \renewcommand{\prenotedelim}{\addcolon\space}
     \renewcommand{\postnotedelim}{\space--\space}
\end{Verbatim}

\index{punctuation!author/year}
\paragraph{I am using an author/year style, and I want a colon/a
  comma/nothing between the author and the year.}
\begin{marginfigure}[4ex]
\fbox{\includegraphics{examples/punctcite3.pdf}}
\vspace{3pt}%
\caption{\cs{nameyeardelim}}\label{punctcite3}
\end{marginfigure}
This is controlled by the \cs{nameyeardelim} macro (see figure
\ref{punctcite3}). In the standard styles it is, as you can see, a
space. If we wanted to make it, say, a comma, we could do so as
follows:
\begin{Verbatim}
    \renewcommand{\nameyeardelim}{\addcomma\space}
\end{Verbatim}
to produce the revised punctuation that is marked in figure
\ref{punctcite4}.
\begin{marginfigure}[4ex]
\fbox{\includegraphics{examples/punctcite4.pdf}}
\vspace{3pt}%
\caption{\cs{nameyeardelim} revised}\label{punctcite4}
\end{marginfigure}

\index{punctuation!multiple citations}
\paragraph{I have multiple citations and I want different punctuation
  between them.} As can be seen from figure \ref{punctcite5}
\begin{marginfigure}[4ex]
\fbox{\includegraphics{examples/punctcite5.pdf}}
\vspace{3pt}%
\caption{Multiple citations}\label{punctcite5}
\end{marginfigure}
the mark placed between most multiple citations is defined either by
\cs{multicitedelim} or by \cs{bibrangedash}. You are very unlikely to
want to redefine the latter command (which is used in many other
places), but \cs{muliticitedelim} can easily be redefined (a comma and
a space, or a semicolon and a space being the most common). To give a
(silly) example
\begin{Verbatim}
   \renewcommand{\multicitedelim}{\slash}
\end{Verbatim}
would give us the rather bizarre results show in figure
\ref{punctcite6}.
\begin{marginfigure}[-8ex]
\fbox{\includegraphics{examples/punctcite6.pdf}}
\vspace{3pt}%
\caption{Multiple citations redefined}\label{punctcite6}
\end{marginfigure}

Between \smallcaps{superscript citations} the delimiter is not
\cs{multicitedelim}, but \cs{supercitedelim}, so if you use
superscript citations it is this that you will have to redefine.

\index{punctuation!after `In'}
\paragraph{I don't want punctuation after `In'.} Standard \biblatex\
styles print a colon after `In:' This is produced by
\cs{intitlepunct}, as shown in figure \ref{punctcite7}.
\begin{figure}
\fbox{\includegraphics{examples/punctcite7.pdf}}
\caption{In: $\rightarrow$ In}\label{punctcite7}
\end{figure}

You can therefore redefine \cs{intitlepunct}, for instance to
\begin{Verbatim}
   \renewcommand{\intitlepunct}{\addspace\nopunct}
\end{Verbatim}
to avoid any punctuation being added at that point.

\index{customization!in@`In'!removal}
\paragraph{I want to remove the reference to `in' altogether.}

In that case
\begin{verbatim}
\renewbibmacro{in:}{}
\end{verbatim}

The trouble with this is that it removes `in' not only from, say,
articles, but also from other types of source (such as |incollection|)
where it may still be needed. In that case, you need a more complex
revision.\footnote{Thanks to Herbert:
  \url{http://tex.stackexchange.com/questions/10682/suppress-in-biblatex}},
using \cs[\angled{type}]{ifentrytype}. The following, for instance will
suppress `in' for |article| and |inproceedings| only, and you should
be able to work out how to adapt it.
\begin{verbatim}
\renewbibmacro{in:}{%
  \ifboolexpr{%
     test {\ifentrytype{article}}%
     or
     test {\ifentrytype{inproceedings}}%
  }
   {}%<-nothing
   {\printtext{\bibstring{in}\intitlepunct}}%<-normal 'in' with punctuation
}
\end{verbatim}


\index{customization!labels in parentheses}
\paragraph{I am using a numeric style, but I want my citations to
  appear in parentheses (1) instead of brackets
  [1].}\label{recipe:brackets}
\begin{marginfigure}[8ex]
\fbox{\includegraphics{examples/punctcite8.pdf}}
\vspace{3pt}%
\caption{Brackets replaced by parentheses}\label{punctcite8}
\end{marginfigure}
This is a surprisingly difficult change to made. The code required is
as follows:\footnote{Using the clever method devised by Audrey
  Boruvka: \url{http://tex.stackexchange.com/a/16792/5404}.}

\begin{Verbatim}
\makeatletter

\newrobustcmd*{\parentexttrack}[1]{%
  \begingroup
  \blx@blxinit
  \blx@setsfcodes
  \blx@bibopenparen#1\blx@bibcloseparen%
  \endgroup}

\AtEveryCite{%
  \let\parentext=\parentexttrack%
  \let\bibopenparen=\bibopenbracket
  \let\bibcloseparen=\bibclosebracket}

\makeatother

\DeclareFieldFormat{labelnumberwidth}{\mkbibparens{#1}}
\DeclareFieldFormat{labelformat}{\mkbibbrackets{#1}}
\end{Verbatim}

\index{customization!citations in brackets}
\paragraph{I am using an author/year style: I want to replace round
  parentheses with square brackets: [Author 1978] instead of (Author
  1978).} This is more-or-less the opposite problem to the last one,
and again it takes a surprising amount of work to get his right. The
relevant code is as follows (to go in the preamble).
\begin{Verbatim}
\makeatletter

\newrobustcmd*{\parentexttrack}[1]{%
  \begingroup
  \blx@blxinit
  \blx@setsfcodes
  \blx@bibopenbracket#1\blx@bibclosebracket%
  \endgroup}

\AtEveryCite{%
  \let\parentext=\parentexttrack%
  \let\bibopenbracket=\bibopenparen%
  \let\bibclosebracket=\bibcloseparen}

\makeatother
\end{Verbatim}

If you also want the year references to be printed in square brackets
in the bibliography, you need this in addition:
\begin{Verbatim}
\renewbibmacro*{date+extrayear}{%
  \iffieldundef{\thefield{datelabelsource}year}
    {}
    {\printtext[brackets]{%
       \iffieldsequal{year}{\thefield{datelabelsource}year}
         {\printdateextralabel}%
         {\printfield{labelyear}%
          \printfield{extrayear}}}}}%
\renewbibmacro*{date}{}%
\renewbibmacro*{issue+date}{%
  \iffieldundef{issue}
    {}
    {\printtext[brackets]{\printfield{issue}}}%
  \newunit}
\end{Verbatim}

\index{punctuation!within entry}
\paragraph{I would like the parts of a bibliographical entry to be
  separated by commas, but the standard styles use full stops.} The
division between the main parts of an entry are marked by
\cs{newunitpunct}. As you can see from figure \ref{punctcite10}, it is
quite frequently used.
\begin{figure}
\fbox{\includegraphics{examples/punctcite10.pdf}}
\caption{The \cs{newunitpunct} punctuation\label{punctcite10}}
\end{figure}

If we modify it by
\begin{Verbatim}
\renewcommand{\newunitpunct}{\addcomma\space}
\end{Verbatim}
we get a less choppy result: figure \ref{punctcite11}. Notice that the
capitalization adjust automatically.\begin{figure}
  \fbox{\includegraphics{examples/punctcite11.pdf}}
\caption{The \cs{newunitpunct} punctuation\label{punctcite11}}
\end{figure} It may, however, sometimes be necessary to adjust other
punctuation marks to ensure a really smooth result.

\index{punctuation!quotation marks}
\paragraph{I am not happy with the way quotation marks are being put
  around article names.} This is largely a function of how
\package{csquotes} is loaded.
\begin{itemize}
\item Loaded with the option |style=british|, you will get single
   quotation marks: `Like This'.
\item Loaded with the option |style=american|, you will get doubt
   quotation marks: ``Like This''.
\item If, in addition to loading \package{csquotes} with the option
   |style=american| you load \package{babel} or \package{polyglossia} with the
   |american| language you will get the American style of handling punctuation
   and quotations, where punctuation is moved inside the quotation:
   ``like this,'' in most cases.
\end{itemize}
Of course, you can equally use \package{csquotes} with «\,French\,»,
,,German``, «Spanish», or any other style of quotation.

%''

%\section{Issues about what gets printed}
\section{关于打印什么内容的问题}

\index{customization!URL!font}\index{customization!font!url}
\paragraph{I don't want to print \smallcaps{urls}\slash
  \smallcaps{eprint}\slash \smallcaps{doi}s.} These can be controlled
with the options |url|, |doi| and |eprint|. By default all of these
are set |true|; to turn off any of them set the relevant option to
|false|. If you want links to be created you should load the
\package{hyperref} package.

\index{customization!date}
\index{date!customization}
\paragraph{I want to control how dates get printed.}
\package{Biblatex} may print a number of different dates at various
points in a citation, and the format for each of them can be
separately controlled, using the options given in table
\ref{dateopts}.
\begin{margintable}
\begin{tabularx}{\marginparwidth}{lX}
\toprule
\texttt{date}      & date of publication \\
\texttt{labeldate} & the `label' date (in author/year styles) \\
\texttt{urldate}   & the date a \smallcaps{url} was `last visited' \\
\texttt{eventdate} & the date of a conference \\
\texttt{origdate}  & an original date of publication \\
\texttt{alldates}  & all dates \\
\bottomrule
\end{tabularx}
\vspace{3pt}
\caption{The various options to set date\label{dateopts}}
\end{margintable}

There is a bewildering range of possible settings for these various
date fields. Exactly what will get printed depends may depend on the
language you have set in \package{babel} or \package{polyglossia}: a
date printed as `15th January 2000' in |british| will be printed as
`January 15, 2000' in |american| and 15 janvier 2000 in
|french|. However, tabel \ref{datesettings} gives examples for how a
date of |2000-01-15| would get printed in British English which should
be sufficient to give guidance for other languages.
\begin{table*}
  \begin{tabular}{lp{2.8in}lp{1in}}
    \toprule
                          & \texttt{dateabbrev=false}                                    & \texttt{dateabbrev=true} & \texttt{datezeros=false} \\
    \midrule\texttt{long} & 15th January 2000                                            & 15th Jan 2000                                       \\
                          & ranges written in full: 15th January 2000--17th January 2000 & 15th Jan.\ 2000--17 Jan.\ 2000                      \\
    \texttt{short}        & 15/01/2000                                                   &                          & 15/1/2000                \\
                          & ranges written in full: 15/01/2000--17/01/2000               &                          & 15/1/2000--17/1/2000     \\
    \texttt{year}         & 2000                                                                                                               \\
    \texttt{comp}         & 15th January 2000                                                                                                  \\
                          & ranges compressed: 15th--17th January 2000                   & 15th--17th Jan.\ 2000                               \\
    \texttt{terse}        & as 15/01/2000                                                                                                      \\
                          & ranges compressed: 15--17/01/2000                            &                          & 15--17/1/2000            \\
    \texttt{iso8601}      & 2000-01-15                                                                                                         \\
                          & with ranges as 2000-01-15/2000-01-17                                                                               \\
    \bottomrule
\end{tabular}
\caption{Various date settings\label{datesettings}}
\end{table*}
In addition, you set the |dateabbrev| option |true| to have dates
printed in an abbreviated form (15th Jan.\ 2000), and the |datezeros|
option |true| to have numeric dates printed with leading zeros
(2000-01-15), or |false| to have them printed without.

\index{customization!suppressing information}
\paragraph{There's some other information I don't want printed.} At
this point we are beginning to enter the realm of serious changes,
which are arguably beyond the sort of simple customization that this
chapter is concerned with.

But let's suppose, for instance, that you were absolutely happy with
the standard author/year style, but just wanted not to print the
publisher: just the place of publication. One way to do this is to
`kid' the style that you haven't entered any publisher at all.
\begin{figure}
\fbox{\includegraphics{examples/punctcite12.pdf}}
\vspace{0.5\baselineskip}

\fbox{\includegraphics{examples/punctcite13.pdf}}
\caption{Removing the publisher using a \texttt{sourcemap}\label{punctcite12}}
\end{figure}

For this purpose we use a \emph{sourcemap}, as follows

\begin{minipage}{\textwidth}
\begin{Verbatim}
\DeclareSourcemap{
  \maps[datatype=bibtex]{
    \map[overwrite=true]{
      \step[fieldset=publisher,null]
    }
  }
}
\end{Verbatim}
\end{minipage}

\vspace{1ex}
What this basically means is that for any |.bib| file, the publisher
field of each entry is to be set empty (|null|). This will be done by
\package{biber} before any data enters the system. No publisher is
therefore printed, and so long as the style is capable of coping with
the particular absence, all is well, as figure \ref{punctcite12}
shows.

%%% Local Variables:
%%% coding: utf-8
%%% mode: LaTeX
%%% TeX-master: "biblatex-tutorial"
%%% End: 